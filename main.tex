\documentclass[licencjacka]{pracamgr}
%% \usepackage{polski}
\usepackage[T1]{fontenc} 
\usepackage[utf8]{inputenc} 
\usepackage{amssymb}
\usepackage{amsmath}
\usepackage{amsthm}

\theoremstyle{definition}
\newtheorem{definition}{Definition}[section]

\theoremstyle{definition}
\newtheorem{remark}{Remark}[section]

\theoremstyle{plain}
\newtheorem{lemma}{Lemma}[section]

\theoremstyle{plain}
\newtheorem{theorem}{Theorem}[section]

%% Document global definitions
\def\cfm{\ensuremath\mathrm{c}^f\mathcal{M}}

\author{Michał Garmulewicz}

\nralbumu{304742}


\title{$\mathrm{L}_p$-cohomologies of Riemannian $f$-horns.}

\tytulang{An implementation of a difference blabalizer based on the theory 
  of $\sigma$ -- $\rho$ phetors}

\kierunek{Matematyka} % Praca wykonana pod kierunkiem:
% (podać tytuł/stopień imię i nazwisko opiekuna
% Instytut
% ew. Wydział ew. Uczelnia (jeżeli nie MIM UW))
\opiekun{dra hab. Andrzeja Webera\\
              Instytut Matematyki\\
        }

% miesiąc i~rok:
\date{Maj 2015}

%Podać dziedzinę wg klasyfikacji Socrates-Erasmus:
\dziedzina{ 
11.0 Matematyka, Informatyka:\\ 
11.1 Matematyka\\ 
}

%Klasyfikacja tematyczna wedlug AMS (matematyka) lub ACM (informatyka)
\klasyfikacja{14 Algebraic Geometry\\
  14F (Co)homology theory\\
  14F40 de Rham cohomology}

% Słowa kluczowe:
\keywords{
  kohomologie de Rhama, topologia różniczkowa
}

% Tu jest dobre miejsce na Twoje własne makra i~środowiska:
\newtheorem{defi}{Definicja}[section]

% koniec definicji

\begin{document}
\maketitle

%tu idzie streszczenie na strone poczatkowa
\begin{abstract}
  In this thesis $L_p$-cohomologies of Riemannian $f$-horn are calculated.
\end{abstract}

\tableofcontents
%\listoffigures
%\listoftables

\chapter{Introduction}
\addcontentsline{toc}{chapter}{Introduction}

In \cite{weber} the author considers a cone over Riemannian
pseudomanifold.  The cone is given a metric of the form $dt \otimes dt
\oplus t^2 g$ and $g$ is the metric on the nonsingular part of the
original pseudomanifold. In the mentioned work, $L_p$ cohomology of
this space is presented.

This line of research can 
be traced back to Cheeger \cite{cheeger}. 
A similar approach was presented by Youssin \cite{youssin}, where $f$-horns
were considered.

 We present a slight modification of this notions, by considering
 manifolds where the Riemannian metric is of the for

\chapter{Preliminaries}
\section{Vector spaces and tensors}
Let us recall some basic facts about behaviour of norm when scaling tensors.
If we consider a finite-dimensional vector space $V$ with a given metric $||
\cdot ||$ and define a new metric $|| x ||_r  = r \dot || x ||$. Then in the
space $(V, || \cdot||_r)^\ast$ dual to $(V, ||| \cdot |||)$, the normed is scaled
by the factor $\frac{1}{r}$, that is for any $\varphi \in V^\ast$ we get
$||\varphi||_r = \frac{1}{r} ||\varphi|| $. \\

We will make a simple observation that we will later use in the
computations.  We consider a Riemannian manifold $\mathrm{M}$ and
tagent $T_x\mathrm{M}$ and cotangent $T_x^\ast\mathrm{M}$ spaces in
the point $x$.  Let the bases of these spaces be $e_1, e_2, ..., e_n$
and dual $e_1^\ast, e_2^\ast, ..., e_n^\ast$. The volume form of this
manifold is $d\mathrm{vol} = \pm e_1^\ast \wedge e_2^\ast \wedge
... \wedge e_n^\ast $. \\

We now want to compute how forms from $\Lambda(\mathcal{M})$ are scaled
with respect to such a change in the norm. Suppose we are considering
space of $k$-forms on $\mathbb{M}$ at some arbitrary point. Then every
$k$-form can be locally expressed in a basis consisting of products of
covectors belonging to basis dual to the standard basis. That is every
$k$-form in the point $x$ using some local coordinates $(x_1, x_2,
... x_n)$ can be written $ \sum_{\mathrm{I} \in I } a_\mathrm{I}
dx_\mathrm{i}$, where $I$ is the set of $k$-indices of form
$\underbrace{(i_1, i_2, ..., i_k)}_{k~\mathrm{times}}$, with $i_1,
i_2, ... \in \{1, 2, ..., n \}$.  (following Einstein convention).
Let us see how basis vector is scaled:
\begin{multline*}
    ||dx_{i_1} \wedge dx_{i_2} \wedge ... \wedge dx_{i_k} ||_t =  
    ||dx_{i_1} ||_t \cdot ||  dx_{i_2} ||_t \cdot ... \cdot || dx_{i_k} ||_t =  \\
    \frac{1}{t}||dx_{i_1} || \cdot \frac{1}{t} ||  dx_{i_2} ||_t \cdot ...
     \cdot \frac{1}{t} || dx_{i_k} ||_t = 
    \frac{1}{t^k}||dx_{i_1} \wedge dx_{i_2} \wedge ... \wedge dx_{i_k} ||
\end{multline*} \\

This means that any $k$ form is scaled by $1/t^k$ when metric is scaled
by a factor of $t$.
This applies also to the volume form, so we obtain:
\[
d\mathrm{vol}_t = \frac{1}{t^n} d\mathrm{vol}
\]

\section{Differetial forms}

\textbf{Riemannian metric} is a smooth symmetric covariant 2-tensor field
on manifold $\mathcal{M}$ that is positive definite at each point.(attaching
a field of linear functions that takes two variables to every point of the
manifold).

Consulting page 328 of Lee gives us that in any smooth local coordinates
$(x^i)$, Riemannian metric can be written as:
\[
    g = g_{ij} dx^i \otimes dx^j = g_{ij} dx^i dx^j
\]
where $g_{ij}$ is a positive definite matrix of smooth functions. 

The simplest example of Riemannian metric is \emph{Euclidean metric} on
$\mathbb{R}^n$ given in standard coordinates by 
\[
    g = \delta_{ij}dx^idx^j.
\]

Citing prof. Lee, it is common to abbreviate the symmetric product of a tensor
$\alpha$ with itself by $\alpha^2$, so the Euclidean metric can also be written
as 
\[
    g = (dx^1)^2 + ...  + (dx^n)^2,
\]
so now it is way easier to understand what exactly is meant by
$ dt \otimes dt + f^2g $, which shoudl be the same as $dt^2 + f^2g$. \\
%% wiki volume form hehehehe

\textbf{Induced map} For any smooth map $F:
M \rightarrow N$ between two smooth manifolds with or without
boundary, the pullback $F^\ast: \Omega^p N \rightarrow \Omega^p M$
carries closed forms to closed forms and exact forms to exact
forms. It thus decsends to a linear map, denoted by $F^\ast: H^p N
\rightarrow H^p M$, too. \\

Digression in digression in digression: \textbf{Pullback} of $F^\ast$ is
\[
    (F^\ast \omega)_p(v_1, ..., v_n) =
        \omega_{F(p)}(dF_p(v_1), ..., dF_p(v_k)).
\] \\


\section{Explaination about induced maps etc.}
%% Mike note - this star means induced map, which is explained in the
%% previous chapter

If we have two smooth maps $F, G:
M \rightarrow N$ and we want to prove that the induced maps are equal
$F^\ast = G^\ast$. Given a closed $p-$form $\omega$ on $N$, we need to
produce a $(p-1)$-form $\eta$ no $M$ such that

\[
    G^\ast \omega - F^\ast \omega = d\eta
\]

from this, it will follow that $ G^\ast [\omega] - F^\ast [\omega] =
[d\eta] = 0$, where $[]$ is just taking homotopy equivalence class
of given form. The author suggests a way to make it more systematic,
by finding an operator $h$, which transforms closed $p$-forms on $N$
to $(p-1)$-forms on $M$ and satisfies

\[
    d(h\omega) = G^\ast \omega - F^\ast \omega.
\]

Instead of defining $h\omega$ only when $\omega$ is close, it turns
out to be far easier to define a map $h$ from the space of
\textit{all} smooth $p$-forms on $N$ to the space of smoooth
$(p-1)$-forms on $M$, which satisfies:

\[
    d(h\omega) + h(d\omega) = G^\ast \omega - F^\ast \omega ,
\]

which implies the above equality when $\omega$ is closed. (To be
completly precise, we define a family of maps, one for each $p$, which
satisfy said equalities on adequate levels.

\[
    H(\mathcal{M} \times \mathbb{R}_{\geq})_{dR}^\ast = H(\mathcal{M})_{dR}^\ast
\]






\chapter{Computation}
\addcontentsline{toc}{chapter}{Computation}

The purpose of this chapter is to present the computation of
$L_p$-cohomologies of Riemannian $f$-horns.

\section{Setting}
Here we introduce definitions and make the first observations. The setting
is largely similar to the setting presented in \cite{weber}, \cite{youssin},
\cite{cheeger}. \\

\begin{definition}[$f$-horn]
Let $\mathcal{M}$ be a Riemannian manifold. Consider a space $
\mathbb{R}_{\geq 0} \times \mathcal{M}$.  Define a Riemannian tensor
on this product by $dt^2 \oplus f^{2}(t)g $, where $g$ is the metric
on $\mathcal{M}$.  Such a space will be called an
\textbf{$f$-horn}. We will denote it by $\mathrm{c}^f \mathcal{M}$. \\
\end{definition}

\begin{remark}
This terminology is present in works of Cheeger.
\end{remark}

At first, we will focus our attention on scaling functions from family
$f_\alpha(x) = \mathrm{e}^{\alpha x}$, parametrized by $\alpha \in
\mathbb{R}$.  The intuition behind such manifolds is best presented
graphically, as in the Figure ~\ref{intro-pic-1}. \\

%% TODO -> picture % Rysuneczek z rurkom

We can make a simple observation here about differential forms associated with
$f$-horn. The tangent space in the point $(t, m)$ is:
\[
    T_{(t, m)} (\mathrm{c}^f \mathcal{M}) = \mathbb{R} \times T_m \mathcal{M}
\]
In terms of differential forms associated with $f$-horn, it means that
$\Lambda^k(\mathbb{R} \times T_m \mathcal{M}) = 
\Lambda^k(\mathbb{R})  \oplus \Lambda^k(\mathcal{M}) $.
It can be rephrased in friendlier terms in the following way.

\begin{remark}
Every $k$-form $\omega \in \Lambda^k T(\mathrm{c}^f \mathcal{M})$, and
consequently every form in the space of $p$ integrable forms $L_k^p
(\mathrm{c}^f \mathcal{M}$ can be written as $\omega = \eta + \xi
\wedge dt$, where both $\eta$ and $\xi$ do not contain $dt$.  Please
note that $\eta$ is $k$-form and $\xi$ is $k-1$ form.
\end{remark}


%% citing heavily from Weber!

\begin{remark}[Norms of forms]
We have the standard inclusion:
\[
    i_r: \mathcal{M} \rightarrow \mathrm{c}^f \mathcal{M},
\]
\[
    i_r(x) = (x, r).
\]
With this in mind, we will write $\omega_r = i_r^\ast(\omega) =
i_r^\ast(\eta)$. Further, let us
%% I can actually explain this one! And I will, to make it a bit longer.
%% it is explained in Weber's notes
 denote $|| \omega |_{\mathcal{M} \times
  \{r\}} || = f(r)^{n/p - k} ||\omega_r||$ as $|| \omega ||_r $. Moreover, if
\[
    \pi:\mathrm{c}^f \mathcal{M} \rightarrow \mathcal{M}
\]
is the projection, we establish that for $\eta \in L^\ast_p(\mathrm{c}^f\mathcal{M})$
we can write $||\eta||_r := ||\pi^\ast\eta||_r = f(r)^{n/p - k} ||\eta_r||$.
\end{remark}

%% ENDED HERE

%% Mike's note: Please note we have many different norms here, which
%% are used for things from different spaces.

As we are computing cohomologies, we should define the homology operator. Let
%% expand why in the introduction!
%%
\[
  I_r: \Omega^\ast( \mathrm{c}^f \mathcal{M} ) \rightarrow
  \Omega^{\ast-1}(\mathrm{c}^f \mathcal{I} ) 
\]
\[
    I_r(\omega)(x, t) = \int_r^t \xi(x, s) ds
\]

????????????????????????????????????????????\\

The form $I_r\omega$ is smooth for $r \in (0,1)$, but will also consider
$r=0$ in certain cases. If $r>0$ then the homotopy formula holds:
\[
    \omega - \pi^\ast(\omega_r) = d I_r\omega + I_rd\omega.
\] 
This one probably is Poincare lemma in a version suited for our situation.
This is a good place to rewrite it here using Bott \\ 
????????????????????????????????????????? \\

%%% In reference to Lemma 10.1 from prof's Weber's
\begin{lemma}
Let $k < (n+1)/p $ Then the form $\pi^\ast$ is $p$-integrable for each 
$p$-integrable form $\eta \in L^k_p(\mathcal{M})$. 
%%%% CHECK IT!!
\end{lemma}
\begin{proof}
\[
    ||\pi^\ast \eta ||^p = 
    \int_{c \mathcal{M}} |\pi^\ast \eta(x,t)|^p d \mathrm{vol}(c\mathcal{M}) = 
    \int_0^\infty ||\eta||_t^p dt = ||\eta||^p \int_0^\infty f(t)^{n-pk} dt
\]
so, if $f$ is given by $f(x) = e^\alpha$, the last integral is  
$ ||\eta||^p \int_0^\infty e^{\alpha (n-pk)} dt$. This integral is convergent
if $\alpha (n - pk) < 0$.
\end{proof}

The following lemma will be useful in further computation.
\begin{lemma}[Estimation of $\int \xi$]
  blabla bla
\end{lemma}
\begin{proof}
\begin{multline*}
    || \int_t^r \xi_x dx || \leq
     \int_t^r || \xi_x || dx  =  \\
     \int_r^t f(x)^{i - n/p} f(x)^{n/p - i} || \xi_x || dx  \leq  \\
   | \int_r^t f(x)^{i - n/p} dx|^{1/q} \cdot
   |\int_r^t f(x)^{n/p - i} || \xi_s || ds |^{1/p}
\end{multline*}
\end{proof}


%% In reference to Lemma 10.3
\begin{lemma}
   Let $k < (n+1)/p$. Then for $p$-integrable $\omega \in
   \Omega^k(\mathrm{c}^f\mathcal{M})$ the form $I_r\omega$ is
   integrable.
\end{lemma}
\begin{proof}
  For the case $k < (n+1)/p = n/p - 1/q + 1$ we get from the previous lemma:
\[
||I_r\omega|| = \int_0^\infty ||\int_r^t \xi_s ds|| dt \leq
 C \int_0^\infty f(t)^{p/q},
\]
and for $k=(n+1)/p$, we get:
\[
  ||I_r\omega|| \leq \int_0^1 | t(\mathrm{log} r - \mathrm{log} t)  |^{p/q} dt
\]
\end{proof}

\begin{remark} %% this is corollary 10.4 from Weber's
  Based on the above lemma, we get that if $k<(n+1)/p$ then a $p$-integrable
closed form is cohomologous to $\pi^\ast \omega_r$ for almost all $r$.
\[
    \omega = dI_0\omega + I_0d\omega
\]
\end{remark}
\begin{proof}
  We have $\omega - \pi^\ast(\omega_r) = dI_r\omega$ and $I_r \omega$ is 
$p$-integrable. The form $\omega_r$ is $p$-integrable for almost all $r$.
So for almost all $r$ the form $dI_r\omega$ is also $p$-integrable.
\end{proof}

\begin{lemma} %% Refering to the lemma 10.5 from Weber
  Let $k > (n+1)/p$ Then $I_0 \omega $ is $p$-integrable for 
$\omega \in L_p \Omega^k(\cfm)$ and the homotopy formula holds:
\[
    \omega = d I_0 \omega + I_0 d \omega
\]
\end{lemma}
\begin{proof}
  Again, we estimate 
\[
    ||I_0 \omega|| = \int_0^\infty ||\int_0^t \xi_s ds||_t^p dt \leq
    ...,
\]
so this norm is finite. The form $I_0 \omega$ may not be smooth. We
\end{proof}







%%%%%%%%%%%%%%%%%%%%%%%%%%%%%%%%%%%%%%%%%%%%%%%%
%%%%%%%%%%%%  Further be dragons  %%%%%%%%%%%%%%
%%%%%%%%%%%%%%%%%%%%%%%%%%%%%%%%%%%%%%%%%%%%%%%%

%%      \begin{figure}[tp]
%%        \centering
%%        \framebox{\vbox to 4cm{\vfil\hbox to
%%            7cm{\hfil\tiny.\hfil}\vfil}}
%%        \caption{Artystyczna wizja blaba w~obrazie węgierskiego artysty
%%          Josipa~A. Rozkoszy pt.~,,Blaba''}
%%      \end{figure}
%%      
%%      \chapter{Wcześniejsze implementacje blabalizatora
%%        różnicowego}\label{r:losers}
%%      
%%      \section{Podejście wprost}
%%      \begin{verbatim}
%%       )[14].
%%       ), {1234}],]. [map [cc], 1, 22]. [rho x 1]. {22; [22]},
%%              001110101010101010101010101010101111101001@
%%       [22%f4].
%%       cq. rep. else 7;
%%       ]. hlt
%%      \end{verbatim}
%%      
%%      \begin{center}
%%        \begin{tabular}{rrr}
%%          $\alpha$ & $\beta$ & $\gamma_7$ \\
%%          901384 & 13784 & 1341\\
%%          68746546 & 13498& 09165\\
%%          918324719& 1789 & 1310 \\
%%          9089 & 91032874& 1873 \\
%%          1 & 9187 & 19032874193 \\
%%          90143 & 01938 & 0193284 \\
%%          309132 & $-1349$ & $-149089088$ \\
%%          0202122 & 1234132 & 918324098 \\
%%          11234 & $-109234$ & 1934 \\
%%        \end{tabular}
%%      \end{center}

\begin{thebibliography}{99}
\addcontentsline{toc}{chapter}{Bibliography}

\bibitem[Weber]{weber} Andrzej Weber, \textit{An isomorphism from
  intersection homology to $\mathrm{L}_p$-cohomology}, Forum
  Mathematicum, de Gruyer, 1995.
  
\bibitem[Cheeger]{cheeger} Jeff Cheeger, \textit{On the Hodge theory
  of Riemannian pseudomanifolds}, Proc. of Symp. in Pure Math. vol.36,
  American Mathematical Society, Providence, 1982.

\bibitem[Bott]{bott} Raou Bott, \textit{Differetial forms in algebraic
  topology}, Springer Verlag, 1982.

\bibitem[Youssin]{youssin} Boris Youssin, \textit{$\mathrm{L}_p$
  cohomology of cones and horns } J. Differential Geometry, Volume 39,
  Number 3, 1994.
  
\bibitem[Lee]{lee} Lee, \textit{Introduction to Smooth Manifolds}

\end{thebibliography}

\end{document}
