\documentclass[licencjacka]{pracamgr}
\usepackage{polski}
\usepackage[T1]{fontenc} 
\usepackage[utf8]{inputenc} 
\usepackage{amssymb}

\author{Michał Garmulewicz}

\nralbumu{304742}

\title{$L_p$-cohomologies of Riemannian horns.}

\tytulang{An implementation of a difference blabalizer based on the theory 
  of $\sigma$ -- $\rho$ phetors}

\kierunek{Matematyka} % Praca wykonana pod kierunkiem:
% (podać tytuł/stopień imię i nazwisko opiekuna
% Instytut
% ew. Wydział ew. Uczelnia (jeżeli nie MIM UW))
\opiekun{dra hab. Andrzeja Webera\\
              Instytut Matematyki\\
        }

% miesiąc i~rok:
\date{Maj 2015}

%Podać dziedzinę wg klasyfikacji Socrates-Erasmus:
\dziedzina{ 
11.0 Matematyka, Informatyka:\\ 
11.1 Matematyka\\ 
}

%Klasyfikacja tematyczna wedlug AMS (matematyka) lub ACM (informatyka)
\klasyfikacja{14 Algebraic Geometry\\
  14F (Co)homology theory\\
  14F40 de Rham cohomology}

% Słowa kluczowe:
\keywords{blabaliza różnicowa, fetory $\sigma$-$\rho$, fooizm,
  blarbarucja, blaba, fetoryka, baleronik}

% Tu jest dobre miejsce na Twoje własne makra i~środowiska:
\newtheorem{defi}{Definicja}[section]

% koniec definicji

\begin{document}
\maketitle

%tu idzie streszczenie na strone poczatkowa
\begin{abstract}
ąęźćżźżżżżż

  W~pracy przedstawiono prototypową implementację blabalizatora
  różnicowego bazującą na teorii fetorów $\sigma$-$\rho$ profesora
  Fifaka.  Wykorzystanie teorii Fifaka daje wreszcie możliwość
  efektywnego wykonania blabalizy numerycznej.  Fakt ten stanowi
  przełom technologiczny, którego konsekwencje trudno z~góry
  przewidzieć.
\end{abstract}

\tableofcontents
%\listoffigures
%\listoftables

\chapter*{Introduction}
\addcontentsline{toc}{chapter}{Introduction}

In \cite{weber} the author considers a cone over Riemannian pseudomanifold.
The cone is given a linear metric and a computation of $L_p$ cohomology of
this space is presented. We present a slight extension of this by considering
manifolds where the metric is blabla. This can be 



\chapter*{Computation}
\addcontentsline{toc}{chapter}{Computation}

The purpose of this paper is to compute $L_p$-cohomologies of Riemannian horns.


\section{Setting}
In this section we introduce basic definitions and make the most straightforward 
observations.

%% \cite{Cheeger}

Let us consider a space $ \mathbb{R}_{\geq 0} \times \mathcal{M} $, where
$\mathcal{M} $ is Riemannian mainfold. We will define a Riemannian tensor on
this product by $dt^2 + f^{2}(t)g $, where $g$ is the metric on $\mathcal{M}$.
Such a space is called by Cheeger an \textbf{$f$-horn}. We will denote it by
$\mathrm{c}^f \mathcal{M}$

% In \cite{Weber}

At first, we will focus our attention on functions $f_1(x) = \mathrm{e}^{x}$ and
$f_2(x) = \mathrm{e}^{-x}$. The intuition behind such manifolds is best presented
graphically, as in the Figure ~\ref{intro-pic-1}.

If we consider a finite-dimensional vector space $V$ with a given metric $||
\cdot ||$ and define a new metric $||| x |||  = r \dot || x ||$. Then in the
space $(V, || \cdot||)^\ast$ dual to $(V, ||| \cdot |||)$, the normed is scaled
by the factor $\frac{1}{r}$. The bases in these spaces are $e_1, e_2, ..., e_n$ and dual
$e_1^\ast, e_2^\ast, ..., e_n^\ast$. Please note that $d\mathrm{vol} = \pm
e_1^\ast, e_2^\ast, ..., e_n^\ast $. This siplifies greatly the computation of
$L_p$ cohomology of the manifold in consideration.

Also, make a writeup here from lee about the whole volume form deal.

% Rysuneczek z rurkom

Let us now
\[
    T_{(t, m)} = \mathbb{R}_+ \times T_m \mathcal{M}
\]
% Rysuneczek kwadrata

Let us take some $ \omega \in \Lambda^k(\mathbb{R} \oplus T_m \mathcal{M}) = 
\Lambda^k(\mathbb{R})  \oplus \Lambda^k(\mathcal{M}) $.
This equality lets us state that every $k$-form can be written as $\omega = \eta
+ \xi \wedge dt$, where both $\eta$ and $\xi$ do not contain $dt$.  Please note
that $\eta$ is $k$-form and $\xi$ is $k-1$ form. \\

\scriptsize

    Mike's note: Why there is this squared thing?
    Lee, page 328.
    \textbf{Riemannian metric} is a smooth symmetric covariant 2-tensor field
    on manifold $\mathcal{M}$ that is positive definite at each point.(attaching
    a field of linear functions that takes two variables to every point of the
    manifold).

    Consulting page 328 of Lee gives us that in any smooth local coordinates
    $(x^i)$, Riemannian metric can be written as:
    \[
        g = g_{ij} dx^i \otimes dx^j = g_{ij} dx^i dx^j
    \]
    where $g_{ij}$ is a positive definite matrix of smooth functions. 

    The simplest example of Riemannian metric is \emph{Euclidean metric} on
    $\mathbb{R}^n$ given in standard coordinates by 
    \[
        g = \delta_{ij}dx^idx^j.
    \]

    Citing prof. Lee, it is common to abbreviate the symmetric product of a tensor
    $\alpha$ with itself by $\alpha^2$, so the Euclidean metric can also be written
    as 
    \[
        g = (dx^1)^2 + ...  + (dx^n)^2,
    \]
    so now it is way easier to understand what exactly is meant by
    $ dt \otimes dt + f^2g $, which shoudl be the same as $dt^2 + f^2g$. \\

\normalsize

Therefore we obtain easily $||e_1^{\ast} \wedge ... \wedge e_n^\ast || =
\frac{1}{f^k}$ and as $d\mathrm{vol} = e_1^{\ast} \wedge ... \wedge e_n^\ast $.


% ref: http://en.wikipedia.org/wiki/Volume_form

% the most important equation of the article
\[
    \int_{\mathcal{M}} ||| \omega |||^p d\mathrm{vol} = 
    \int_{\mathcal{M}}  (f^{-k}|| \omega ||)^p =  
\]

If we have the standard inclusion:
\[
    i_r: \mathcal{M} \rightarrow \mathrm{c}^f \mathcal{M}
\]
\[
    i_r(x) = (x, r)
\]

%% As in \cite{Weber}
We define $|| \omega ||_r := || \omega || = r^{n/p - k} ||\omega_r||$

Our goal is to define homotopy operator (see in Lee why ??). We do so by
defining $I_r$:
\[
  i I_r: \Omega^\ast( \mathrm{c}^f \mathcal{M} ) \rightarrow
  \Omega^{\ast-1}(\mathrm{c}^f \mathcal{M} ) 
\]
\[
    I_r(\omega)(x, t) = \int_r^t \xi(x, s) ds
\]

We now have to estimate $\int \xi$.

What's the general plan? Try to dig through prof Weber paper and make
sense of the whole estimation section, and later apply same ideas to
make your research. \\ 

\scriptsize
Mike's note. What's the point of all these operators? Will try to
explain here, using Lee, Weber, Cheeger, Hatcher.  One clue is that we
have to compute/prove something like Lee, page 444.  Say we have $F,G:
M \rightarrow N$ which are smooth maps. We want to prove that induced
maps at the homotopies are equal, $F^\ast = g^\ast$. \\

Digression in digression: \textbf{Induced map} For any smooth map $F:
M \rightarrow N$ between two smooth manifolds with or without
boundary, the pullback $F^\ast: \Omega^p N \rightarrow \Omega^p M$
carries closed forms to closed forms and exact forms to exact
forms. It thus decsends to a linear map, denoted by $F^\ast: H^p N
\rightarrow H^p M$, too. \\

Digression in digression in digression: \textbf{Pullback} of $F^\ast$ is
\[
    (F^\ast \omega)_p(v_1, ..., v_n) =
        \omega_{F(p)}(dF_p(v_1), ..., dF_p(v_k)).
\] \\

Back to the main thread of thought: If we have two smooth maps $F, G:
M \rightarrow N$ and we want to prove that the induced maps are equal
$F^\ast = G^\ast$. Given a closed $p-$form $\omega$ on $N$, we need to
produce a $(p-1)$-form $\eta$ no $M$ such that

\[
    G^\ast \omega - F^\ast \omega = d\eta
\]

from this, it will follow that $ G^\ast [\omega] - F^\ast [\omega] =
[d\eta] = 0$, where $[]$ is just taking cohomology equivalence class
of given form. The author suggests a way to make it more systematic,
by finding an operator $h$, which transforms closed $p$-forms on $N$
to $(p-1)$-forms on $M$ and satisfies

\[
    d(h\omega) = G^\ast \omega - F^\ast \omega.
\]

Instead of defining $h\omega$ only when $\omega$ is close, it turns
out to be far easier to define a map $h$ from the space of
\textit{all} smooth $p$-forms on $N$ to the space of smoooth
$(p-1)$-forms on $M$, which satisfies:

\[
    d(h\omega) + h(d\omega) = G^\ast \omega - F^\ast \omega ,
\]

which implies the above equality when $\omega$ is closed. (To be
completly precise, we define a family of maps, one for each $p$, which
satisfy said equalities on adequate levels.

\[
    H(\mathcal{M} \times \mathbb{R}_{\geq})_{dR}^\ast = H(\mathcal{M})_{dR}^\ast
\]

\normalsize



%%%%%%%%%%%%%%%%%%%%%%%%%%%%%%%%%%%%%%%%%%%%%%%%
%%%%%%%%%%%%  Further be dragons  %%%%%%%%%%%%%%
%%%%%%%%%%%%%%%%%%%%%%%%%%%%%%%%%%%%%%%%%%%%%%%%

%%      \begin{figure}[tp]
%%        \centering
%%        \framebox{\vbox to 4cm{\vfil\hbox to
%%            7cm{\hfil\tiny.\hfil}\vfil}}
%%        \caption{Artystyczna wizja blaba w~obrazie węgierskiego artysty
%%          Josipa~A. Rozkoszy pt.~,,Blaba''}
%%      \end{figure}
%%      
%%      \chapter{Wcześniejsze implementacje blabalizatora
%%        różnicowego}\label{r:losers}
%%      
%%      \section{Podejście wprost}
%%      \begin{verbatim}
%%       )[14].
%%       ), {1234}],]. [map [cc], 1, 22]. [rho x 1]. {22; [22]},
%%             dd.
%%       [11; sigma].
%%              ss.4.c.q.42.b.ll.ls.chmod.aux.rm.foo;
%%       [112.34; rho];
%%              001110101010101010101010101010101111101001@
%%       [22%f4].
%%       cq. rep. else 7;
%%       ]. hlt
%%      \end{verbatim}
%%      
%%      \begin{center}
%%        \begin{tabular}{rrr}
%%          $\alpha$ & $\beta$ & $\gamma_7$ \\
%%          901384 & 13784 & 1341\\
%%          68746546 & 13498& 09165\\
%%          918324719& 1789 & 1310 \\
%%          9089 & 91032874& 1873 \\
%%          1 & 9187 & 19032874193 \\
%%          90143 & 01938 & 0193284 \\
%%          309132 & $-1349$ & $-149089088$ \\
%%          0202122 & 1234132 & 918324098 \\
%%          11234 & $-109234$ & 1934 \\
%%        \end{tabular}
%%      \end{center}

\begin{thebibliography}{99}
\addcontentsline{toc}{chapter}{Bibliografia}


\bibitem[Hopp96]{hopp} Claude Hopper, \textit{On some $\Pi$-hedral
    surfaces in quasi-quasi space}, Omnius University Press, 1996.

\bibitem[Leuk00]{leuk} Lechoslav Leukocyt, \textit{Oval mappings ab ovo},
  Materiały Białostockiej Konferencji Hodowców Drobiu, 2000.

%%    \bibitem[Rozk93]{JR} Josip A.~Rozkosza, \textit{O pewnych własnościach
%%        pewnych funkcji}, Północnopomorski Dziennik Matematyczny 63491
%%      (1993).
%%    
%%    \bibitem[Spy59]{spyrpt} Mrowclaw Spyrpt, \textit{A matrix is a matrix
%%        is a matrix}, Mat. Zburp., 91 (1959) 28--35.
%%    
%%    \bibitem[Sri64]{srinis} Rajagopalachari Sriniswamiramanathan,
%%      \textit{Some expansions on the Flausgloten Theorem on locally
%%        congested lutches}, J. Math.  Soc., North Bombay, 13 (1964) 72--6.
%%    
%%    \bibitem[Whi25]{russell} Alfred N. Whitehead, Bertrand Russell,
%%      \textit{Principia Mathematica}, Cambridge University Press, 1925.
%%    
%%    \bibitem[Zen69]{heu} Zenon Zenon, \textit{Użyteczne heurystyki
%%        w~blabalizie}, Młody Technik, nr~11, 1969.

\end{thebibliography}

\end{document}
