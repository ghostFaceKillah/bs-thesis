\documentclass[licencjacka]{pracamgr}
\usepackage[utf8]{inputenc}
\usepackage[T1]{fontenc} 
\usepackage{amssymb}
\usepackage{amsmath}
\usepackage{amsthm}
\usepackage{stackrel}
\usepackage{polski}

\theoremstyle{definition}
\newtheorem{definition}{Definicja}[section]

\theoremstyle{definition}
\newtheorem{remark}{Uwaga}[section]

\theoremstyle{plain}
\newtheorem{lemma}{Lemat}[section]

\theoremstyle{plain}
\newtheorem{proposition}{Propozycja}[section]

\theoremstyle{plain}
\newtheorem{theorem}{Twierdzenie}[section]

\theoremstyle{plain}
\newtheorem{wniosek}{Wniosek}[section]

%% Document global definitions
\def\cfm{\ensuremath\mathrm{c}^fM}
\def\M{\ensuremath M}
\def\N{\ensuremath N}
\def\R{\ensuremath\mathbb{R}}
\newcommand\deff{\mathrel{\overset{\makebox[0pt]{\mbox{\normalfont\tiny\sffamily def}}}{:=}}}


\author{Michał Garmulewicz}

\nralbumu{304742}


\title{$\mathrm{L}_p$-kohomologie $f$-stożków Riemannowskich.}

\tytulang{$\mathrm{L}_p$-cohomologies of Riemannian $f$-horns.}

\kierunek{Matematyka} % Praca wykonana pod kierunkiem:
% (podać tytuł/stopień imię i nazwisko opiekuna
% Instytut
% ew. Wydział ew. Uczelnia (jeżeli nie MIM UW))
\opiekun{dra hab. Andrzeja Webera\\
              Instytut Matematyki\\}

% miesiąc i~rok:
\date{Maj 2015}

%Podać dziedzinę wg klasyfikacji Socrates-Erasmus:
\dziedzina{ 
11.0 Matematyka, Informatyka:\\ 
11.1 Matematyka\\ 
}

%Klasyfikacja tematyczna wedlug AMS (matematyka) lub ACM (informatyka)
\klasyfikacja{14 Algebraic Geometry\\
  14F (Co)homology theory\\
  14F40 de Rham cohomology}

% Słowa kluczowe:
\keywords{
  kohomologie de Rhama, topologia różniczkowa
}

% Tu jest dobre miejsce na Twoje własne makra i~środowiska:
\newtheorem{defi}{Definicja}[section]

% koniec definicji

\begin{document}
\maketitle

%tu idzie streszczenie na strone poczatkowa
\begin{abstract}
  W tej pracy licencjackiej opisane jest obliczenie $L_p$-kohomologii
  rozmaitości Riemannowskich.
\end{abstract}

\tableofcontents
%\listoffigures
%\listoftables

\chapter{Wstęp}

W pracy \cite{weber} prezentowane jest między innymi obliczenie
$L_p$-kohomologii stożka nad pseudorozmaitością Riemannowską. 

Na stożku tym określona jest metryka postaci
$dt \otimes dt + t^2 g$, gdzie $g$ 
jest metryką na części nieosobliwej wyjściowej pseudorozmaitości.
W dalszej części wspomnianej pracy obliczana jest $L_p$-kohomologia
wspomnianej przestrzeni.
\\

1. (bardzo krotka czesc) Algebra liniowa: tu zbiera Pan fragmenty, w ktorych
jest powiedziane jak iloczyn skalarny/norma w V indukuje iloczyn skalarny/norme
w $V^*, \Lambda^k V^*$ i co sie dzieje przy przeskalowaniu.


2. Formy rozniczkowe i regula homotopii: tu dowodzi Pan formule homotopii (*)
$Kd+dK = Id -\pi^*i^*$ (o szacowaniu normy nie ma mowy).  Wniosek:
$H^*(M)=H^*(M\times R)$, izomorfizm zadany przez $\pi^*$.


3. $L^p$-kohomologie (definicja za pomoca form gladkich), szacowanie norm
$\pi^*\omega, I_r\omega,$ (osobno dla $r=\infty$).  Trzeba jasno wypowiedziec idee
przewodnia: C$_fM$ jest dyfeomorficzne z$ M\times R$, chcemy porownac $H^*(M\times
R) z H^*(C_fM)$ uzywajac $\pi^*$ i korzystajac z formuly (*). Dlatego robimy te
oszacowania.

%% This line of research can 
%% be traced back to Cheeger \cite{cheeger}. 
%% A similar approach was presented by Youssin \cite{youssin}, where $f$-horns
%% were considered.

W tej pracy licencjackiej przedstawiam drobną modyfikację tych pojęć do
$e^{-t}$-stożków Riemannowskich, czyli przestrzeni będących produktem
rozmaitości Riemannowskiej oraz półprostej na której określono metrykę
$dt \otimes dt + (e^{-t})^2 g$. \\


\chapter{Algebra liniowa}
W rozdziale tym przytaczam definicje i wyprowadzam podstawowe zależności, które
pomogą nam w dalszych obliczeniach. \\

\section{Algebra zewnętrzna}
W tej sekcji przytaczam definicje form zewnętrznych oraz ich własności. 
 Przytoczone
definicje podane są według podręczników~\cite{lee}, rozdział 14 oraz
~\cite{kostrikin} - rozdział 6 § 3. \\

Formy zewnętrzne są dla nas istotne, ponieważ formy różniczkowe, które 
są podstawowym obiektem służącym do badania kohomologii de Rhama, 
są lokalnie elementami algebry zewnętrznej na przestrzeni stycznej
do rozmaitości w otoczeniu danego punktu. \\

Niech $V$ będzie skończenie wymiarową rzeczywistą przestrzenią wektorową.
Tensor $T:V \times ... \times V \rightarrow \mathbb{R}$ nazwiemy kowariantnym.
Bierze on jako argumenty jedynie wektory i konsekwentnie, nie bierze on jako
argumentów form.  Takie tensory mają wiele nazw: formy zewnętrzne,
multi-kowektory, czy też po prostu $k$-kowektory.  Przestrzeń wszystkich
$k$-kowektorów na przestrzeni $V$ ma wiele oznaczeń.  Jednym z bardziej
popularnych oznaczeń jest $T^k (V^\ast)$. \\

Tensor będzie nazwany \emph{alternującym}, gdy jego wartość zmieni znak w
przypadku zmienimy miejscami jego dwa wektory wejściowe.  Przestrzeń takich
tensorów, które należą do $T^k (V^\ast)$, a do tego są antysymetryczne, często
oznacza się $\Lambda^k (V^\ast)$. \\

Przedstawmy teraz kilka podstawowych operacji określonych na takich tensorach.
Dla dwóch tensorów kowariantnych $f \in T^p(V^\ast) $ oraz $g \in T^q(V^\ast) $
możemy określić produkt (tensorowy) $f \otimes g \in T^{p+q}(V^\ast) $ za
pomocą wzoru:
\[
  f \otimes g(x_1, x_2, ..., x_p, x_{p+1}, x_{p+2}, ... ,x_{p+q}) =
  f(x_1, x_2, ... , x_p) \cdot g(x_{p+1}, x_{p+2}, ... , x_{p+q}).
\] \\

Dowolny kowariantny tensor możemy
przekształcić na tensor alternujący za pomocą przekształcenia
\emph{alternatora}, nazywanego także \emph{rzutem alternującym}.
Jest on określony w następujący sposób:
\[
\text{Alt}:T^k (V^\ast) \rightarrow  \Lambda^k (V^\ast)
\]
\[
\text{Alt}(f)(x_1, x_2, x_3, ..., x_k) = \frac{1}{k!}
  \sum_{\sigma \in S_p}
     \text{sgn} f(x_{\sigma(1)}, x_{\sigma(2)}, ..., x_{\sigma(k)}).
\] \\

Z pomocą alternatora określić można iloczyn zewnętrzny antysymetryczny. 
Dla elementów $\omega \in \Lambda^p (V^\ast)$ oraz 
$\eta \in \Lambda^q (V^\ast)$, ich iloczyn zewnętrzny zadamy wzorem
\[
  \omega \wedge \eta = \frac{(p+q)!}{p!q!} \text{Alt} (\omega \otimes \eta)
\] \\

Przytoczmy teraz sposób, w jaki definiuje się bazę potęgi zewnętrznej
$\Lambda^k(V^\ast)$. Niech $v_1, v_2, ... , v_n$ będzie bazą przestrzeni dualnej
$V^\ast$, która jest dualna do bazy $w_1, w_2, ..., w_n$.
Wówczas układ
\[
  B = \{ v_{i_1} \wedge v_{i_2} \wedge ... \wedge v_{i_k} : 1 \leq i_1 < i_2 < ... <i_k \leq n \}
\]
jest bazą potęgi zewnętrznej $\Lambda^k(V^\ast)$. Warto zauważyć, że implikuje
 to $dim \Lambda^k ( V^\ast)= \binom{n}{k}$

Aby uczynić zadość tytułowi tego podrozdziału, zdefiniujmy
\emph{algebrę zewnętrzną}. Jest to suma prosta:
\[
\Lambda^\ast (V) = 
\Lambda^0(V^\ast) \oplus
\Lambda^1(V^\ast) \oplus
\Lambda^2(V^\ast) \oplus
...~.
\] Stanowi ona algebrę z działaniami dodawania oraz iloczynu zewnęnętrznego.
Fakt ten jest dokładnie uargumentowany w wielu standardowych podręcznikach,
przykładowo w~\cite{lee}, Proposition 14.11 albo \cite{kostrikin}.\\


\section{Norma indukowana na potędze zewnętrznej przestrzeni dualnej.
Skalowanie norm.}

W tym rozdziale przedstawiam relacje pomiędzy normą określoną na danej
przestrzeni wektorowej a indukowaną przez iloczyn skalarny normą na przestrzeni
dualnej oraz na potęgach zewnętrznych przestrzeni dualnej.  W szczególności, w
dalszych obliczeniach będziemy badać wyrażenia typu $r |\cdot|$, gdzie
$|\cdot|$ jest normą na skończenie wymiarowej przestrzeni liniowej.  Znaczenie
będzie miało jak zachowuje się norma na przestrzenii dualnej, gdy skalujemy
normę wyjściowej przestrzenii liniowo o czynnik $r$. \\

\emph{Komentarz: Tu jest konieczny cytat gdzie są opisane ładnie te rzeczy}.

Niech będzie dana skończenie wymiarowa rzeczywista przestrzeń liniowa $V$ z
iloczynem skalarnym $\langle \cdot, \cdot \rangle$.
Ten iloczyn skalarny, na przykład na podstawie twierdzenia Riesza o
reprezentacji \emph{Komentarz: cytowanie}, zadaje izometryczny izomorfizm
pomiędzy $V$ oraz jej przestrzenią dualną $V^\ast$.

%% Izomorfizm ten jest zadany w następujący sposób: Chcemy wektorowi $v \in V$
%% przyporządkować funkcjonał $\phi_v \in V^\ast$. Dla argumentu tego funkcjonału,
%% czyli wektora $w \in V$ ma on następujący wzór:
%% \begin{equation}
%% \phi_v (w) = \langle y, w \rangle.
%% \end{equation}
Dzięki temu, że jest on izometrią, widzimy, że iloczyn skalarny z $V$ indukuje
iloczyn skalarny na $V^\ast$.  Istnienie iloczynu skalarnego implikuje także
istnienie normy na przestrzeni dualnej. Możemy tę normę wyrazić w nieco inny
sposób, co ułatwi nam obliczenie zachowania normy ze względu na skalowanie.
\emph{Komentarz: cytowanie}. \\

Dla rzeczywistej przestrzeni wektorowej funkcjonał $\phi \in V^\ast$ ma
normę określoną wzorem:
\begin{equation}\label{norm-of-functional}
||\phi|| = \sup \left\{ |\phi(v)|: v \in V, ||v|| = 1 \right\},
\end{equation}
gdzie $||v||$ to norma pochodząca z wyjściowej przestrzeni wektorowej $V$.  \\


Iloczyn skalarny jest także przenoszony na potęgę zewnętrzną przestrzeni
dualnej. \emph{Komentarz: cytowanie!! - niby jest napisane w ćwiczeniu z
Kostrikina ale lepiej byłoby znaleźć coś, co mówi o tym wprost} Dla dwóch
$k$-form jest on zadany wzorem

\begin{equation} \langle v^1 \wedge ....\wedge v^k, w^1 \wedge ... \wedge w^k
\rangle = \text{det} \left( \langle v^i, w^j \rangle \right), 
\end{equation}

czyli jest on równy wyznacznikowi macierzy wartości iloczynu skalarnego
zaaplikowanych do poszczególnych składowych $k$-kowektorów.  \\

Możemy teraz zobserwować w jaki sposób zachowają się normy 
przestrzeni dualnej oraz potęgi zewnętrznej gdy przeskalujemy normę wyjściową
o czynnik liniowy $r$. \\

Załóżmy więc, że rozważamy rzeczywistą przestrzeń liniową $V$ z określoną
normą $|| \cdot ||$. Określmy nową normę dla $v \in V$:
\[
|| v ||_r = r || v ||.
\]
Z definicji~\ref{norm-of-functional} możemy bardzo prosto zauważyć, że
dla funkcjonału $\phi \in V^\ast$ nowa norma będzie dana następującym
wzorem:
\[
|| \phi ||_r = \frac{1}{r} ||\phi ||
\]

Zauważmy bowiem, że dla przestrzeni skończenie wymiarowej, supremum
ze wzoru~\ref{norm-of-functional} jest osiągane. 
\emph{Komentarz: cytat ze stosownego AFu}
Załóżmy, że supremum to jest osiągane dla wektora $v$. W nowej normie wektor
ten ma normę $|| v ||_r = r \cdot || v || = r \cdot 1 = r$. Ze względu
na liniowość operatora $\phi$ widzimy, że na zbiorze ${w \in V: ||w||_r = 1}$
wartość $| \phi (w) |$ będzie osiągała swoje supremum dla wielokrotności
$\alpha v$. Istotnie, $\frac{1}{r} v$ maksymalizuje tę wartość i 
jednocześnie widzimy, że 
\[
||\phi||_r = \frac{1}{r} |\phi(v)| = \frac{1}{r}|| \phi ||.
\]  \\

Skoro wiemy w jaki sposób skaluje się iloczyn skalarny na przestrzeni dualnej.
W definicji wyznacznika zakładamy bowiem, że wyznacznik jest liniowy ze względu
na mnożenie wiersza macierzy. Pomnożenie całej macierzy kwadratowej rzędu $k$
przez czynnik $r$ powoduje więc pomnożenie wyznacznika przez czynnik $r^k$.
Otrzymujemy stąd natychmiast dla $k$-formy $\omega = \omega_1 \wedge ... \wedge
\omega_k$  wzór:
\begin{equation}\label{scaling-of-norm}
|| \omega ||_r = \frac{1}{r^k} || \omega ||.
\end{equation}


\chapter{Formy rozniczkowe i regula homotopii}
W tym rozdziale opisuje formy różniczkowe, kohomologie de Rhama i udowadniam 
formułę homotopii. Formuła ta oraz jej dowód jest dla nas szczególnie
interesująca, ponieważ będzie ona kluczowa dla porównania grup
kohomologii $f$-stożka i rozmaitości $M$. \\

Poniższe definicje są przytoczone w formie opartej na podręczniku~\cite{lee}
oraz ~\cite{bott}. W szczególności, dowód formuły homotopii jest zaczerpnięty
z~\cite{bott}, rozdział I § 4. \\

Rozważamy rozmaitość $M$. Dla dowolnej przestrzeni kostycznej 
$T_p^\ast M$ rozpatrzmy jej potęgę zewnętrzną $\Lambda^k(T_p^\ast M)$.
Otrzymujemy w ten sposób przestrzeń liniową nad każdym punktem $p \in M$.
Ich sumę rozłączną oznaczymy $\Lambda^k(T^\ast M)$. Przestrzenie te,
określone nad każdym punktem z osobna, można skleić do wiązki wektorowej, na
której można zdefniować formy różniczkowe, wybierając lokalną
trywializację wiązki stycznej. \emph{Komentarz: zastanowić się co to znaczy
dokładnie} \\

\begin{definition}[Forma różniczkowa]
  $k$-formą różniczkową na rozmaitości $M$ nazwiemy gładkie przekrój wiązki
  $\Lambda^k(T^\ast M)$ nad M, czyli gładkie odwzorowanie $\omega: M \rightarrow
  \Lambda^k (T^\ast M)$, spełniające $\omega(p) \in \Lambda^k(T_p^\ast)$ dla
  każdego punktu $p \in M$.
\end{definition}

Przestrzeń $k$-form różniczkowych oznaczać będziemy przez $\Omega^k(M)$. 
Określimy na niej kluczowe dla dalszych kroków naszego rozumowania pojęcie
\emph{pochodnej zewnętrznej} formy różniczkowej. Operacja ta zwiększa stopień
formy, czyli $d: \Omega^k(M) \rightarrow \Omega^{k+1} (M)$. Pozwala to patrzeć
na nią w dalszej części rozumowania, jako na operator w ciągu przestrzeni
wektorowych $\Omega^k(M)$, tworzącym kompleks łańcuchowy. \\


Aby nadać nieco więcej intuicji definicji, przypomnimy najpierw
definicję różniczki funkcji, czyli operację 
$d: C^\infty = \Omega^o(M) \rightarrow \Omega^1(M)$. Niech $f$ będzie funkcją
gładką na $M$, a $(x^i)$ -  układem współrzędnych. Na dziedzinie tego układu
określimy różniczkę $df$ wzorem
\begin{equation}\label{exterior-derivative-for-one-forms}
df = \sum_{i=1}^n \frac{\partial f}{\partial x^i} dx^i.
\end{equation} \\

Dla form o wyższej gradacji określamy pochodną zewnętrzną w następujący
sposób. Niech 
$\omega = \sum_{j \in I} \alpha_{j_1, ..., j_k} dx^{j_1} \wedge ... \wedge dx^{j_k}$.
Różniczka $d: \Omega^k(M) \rightarrow \Omega^{k+1}(M)$ jest dana jako
\[ %% strona 363 Lee, wzór 14.20
d( \sum_{j \in I} \alpha_{j_1, ..., j_k} dx^{j_1} \wedge ... \wedge dx^{j_k}) = 
 \sum_{j \in I} \sum_{i=1}^n
 \frac{ \partial \alpha_{j_1, ..., j_k}} {\partial x^i} dx^i
                            \wedge dx^{j_1} \wedge ... \wedge dx^{j_k}).
\]
Ponadto, jeśli $\phi$ jest $p$-formą, a $\psi$ jest $q$-formą, to możemy
zapisać następujący wariant formuły Leibniza
\[
d(\phi \wedge \psi) = d\phi \ wedge \psi + (-1)^p \phi \wedge d\psi.
\]
Jest on udowodniony w~\cite{lee}. \\

Warto zwrócić uwagę, że własność~\ref{exterior-derivative-for-one-forms} wraz z
formułą Leibniza determinują postać różniczki dla form w wyższych gradacjach. \\

Przytoczymy teraz kluczową własność różniczki wraz z dowodem.
Przypomnijmy, że w teorii homologii badamy kompleksy łańcuchowe, czyli
ciągi przestrzeni
\[
   ... \xrightarrow{d} 
A^i 
   \xrightarrow{d} 
A^{i+1}
   \xrightarrow{d} 
...,
\]
że $d^2 = d \circ d = 0$. Będziemy się starali stwierdzić, że 
\begin{theorem}
Dla różniczki zewnętrznej form należących do $\left(\Omega^\ast (M), d \right)$
zachdzi
\[
d \circ d = 0.
\]
W związku z tym ciąg przestrzeni  $\left(\Omega^i (M), d \right)_i$ jest
kompleksem łańcuchowym.
\end{theorem}

\begin{proof}
Udowodnijmy najpierw na przypadku szczególnym 0-formy, czyli funkcji o
własnościach rzeczywistych. Dla tego przypadku zachodzi
\begin{align*}
d(du) & = d \left( \sum_j \frac{\partial u} {\partial x^j} dx^j \right) =
\sum_i \sum_j \frac{\partial^2 u}{\partial x^i \partial x^j } dx^i \wedge dx^j =  \\
& = \sum_{i < j} \left(
\frac{\partial^2 u}{\partial x^i \partial x^j}  -
\frac{\partial^2 u}{\partial x^j \partial x^i} 
 \right) dx^i \wedge dx^j = 0,
\end{align*}
z uwagi na to, że pochodne cząstkowe mieszane są sobie równe. 
Dla przypadku ogólnego natomiast, skorzystamy z powyższego przypadku szególnego
oraz formuły Leibniza, które w połączeniu pozwolą nam napisać
\begin{align*}
d(d \omega) & = d \left( \sum_J d \omega_J \wedge dx^{j_1} \wedge ... \wedge dx^{j_k} \right) \\
             & = \sum_J d( d\omega_J) \wedge dx^{j_1} \wedge ... \wedge dx^{j_k}  \\
& + \sum_J \sum_{i=1}^k (-1)^i d \omega_J \wedge dx^{j_1} \wedge ... \wedge d(dx^{j_i}) \wedge ... dx^{j_k} = 0.  \\.
\end{align*}
Nadużywam tu nieco notacji dla zachowania jasności dowodu. Iteracja po $J$ to iteracja
po multiindeksach, a iteracja po $i$ to normalna konwencja sumowania. Ostatnia
równość wynika wprost z definicji różniczki form $d \omega_J$. \\

\end{proof}




\section{Metryka Riemannowska i forma objętości}

\textbf{Metryką Riemannowską} nazwiemy gładkie, symetryczne kowariantne
pole 2-tensorów na rozmaitości
$M$ które jest dodatnio określone w każdym punkcie. Mówiąc bardziej
intuicyjnie, określenie metryki Riemannowskiej to doczepienie 
pola iloczynów skalarnych do rozmaitości, które zmienia się w sposób gładki.
\\

W dowolnych lokalnych gładkich współrzędnych $(x^i)$ metryka Riemannowska
może być zapisana jako
\[ %%% strona 328 Lee, przed przykładem 13.1
    g = g_{ij} dx^i \otimes dx^j = g_{ij} dx^i dx^j
\]
gdzie
$g_{ij}$
jest dodatnio określoną macierzą (której współrzędne to funkcje gładkie). Ostatnia
część równości zapisuje naszą metrykę w terminach produktu symetrycznego. \\

Biorąc pod uwagę, że w głównej części pracy rozważać będziemy
rozmaitości Riemannowskie będące produktem dwóch rozmaitości
Riemannowskich, zbadajmy w jaki sposób zadana będzie metryka na takiej
przestrzeni produktowej. Jeżeli $(M, g)$ oraz $(M', g')$ będą rozmaitościami
Riemannowskimi, to na $M \times M'$ możemy zadać metrykę produktową
 $\hat g = g \oplus g'$ w następujący sposób:
\[
\hat g
 \left( (v, v'), (w, w') \right) =
 g(v, w) + g'(v', w')
\]
dla każdego
 $(v, v'), (w, w') \in T_p M \oplus T_q M' \cong T_{(p, q)} (M \times M')$.
Gdy mamy dane jakieś konkretne współrzędne $(x_1, ... , x_n)$ dla $M$ oraz
$(y_1, ..., y_n)$ dla $M'$, to dostajemy prosto lokalne współrzędne
$(x_1, ..., x_n, y_1, ..., y_m)$ na $M \times M'$ i nietrudno sprawdzić, 
że metryka produktowa jest lokalnie reprezentowana przez macierz blokowo diagonalną
\[
  \left(\hat g_{ij} \right)  = 
  \left(
    \begin{array}{cc}
  \left( g_{ij} \right) &  0 \\
      0      & \left( g'_{ij} \right) \\
      \end{array}
  \right).
\] \\

W kolejnym rozdziale pracy będziemy mówić o całkowaniu funkcji po rozmaitości
Riemannowskiej. Co prawda nie będzie nam to potrzebne aż do kolejnego 
rozdziału, ale jest do technika ściśle związana z metryką Riemannowską, więc
przedstawię ją tutaj. \\

Aby móc całkować funkcje na rozmaitości potrzebne jest nam pojęcie
Riemannowskiej formy objętości. Szczegóły dotyczącego tego rozumowania 
są dość zawiłe i w małym stopniu mają wpływ na tę pracę. Dlatego też
przywołam tylko definicję i twierdzenie o postaci formy objętości
w lokalnych współrzędnych, zamieszacjąc jedynie odnośnik do dowodu w 
źródle. \\

\begin{remark}
Niech $(M, g)$ będzie zorientowaną rozmaitością Riemannowską wymiary
$n \geq 1$. Istnieje dokładnie jedna gładka forma objętości
$\omega_g \in \Omega^n(M)$, nazywana \textbf{Riemannowską formą objętości},
która spełnia równanie
\[
\omega_g(E_1, ...,  E_n) = 1
\]
gdzie $(E_i)$ jest lokalną, ortonormalną, dodatnio zorientowaną bazą
pól wektorowych.
\end{remark}
\begin{proof}
Pomijam, zamieszczony w oryginalnym źródle~\cite{lee}, Proposition 15.29.
\end{proof}

To, co dokładnie oznacza ta definicja i jaka jest jej motywacja jest poza
zakresem zainteresowania tej pracy. Musimy natomiast z perspektywy
dalszych obliczeń znać jaka jest lokalna postać formy objętości.

\begin{remark}\label{expression-for-volume-form}
 %% Proposition 15.31 z Lee
Niech $(M, g)$ będzie zorientowaną rozmaitością Riemannowską wymiaru $n$, z
brzegiem lub bez brzegu. W dowolnych zorientowanych gładkich współrzędnych
$(x_i)$, Riemmanowska forma objętości może być wyrażona lokalnie w następujący
sposób:
\[
    \omega_g = \sqrt{\text{det}(g_{ij})} dx^1 \wedge ... \wedge dx^n
\]
\end{remark}
Dowód tej własności także omijam, jest on dostępny w ~\cite{lee}, Proposition
15.31. \\

Do naszych dalszych obliczeń potrzebna nam będzie umiejętność całkowania
funkcji rzeczywistych pod rozmaitościach Riemannowskich. Zdefiniujmy w tym celu
stosowną całkę.
Niech $(M, g)$ będzie zorientowaną rozmaitością Riemannowską. 
Niech $\text{vol}_g$ oznacza jej formę objętości. Jeżeli mamy teraz $f$ -
funkcję o zwartym nośniku, rzeczywistą i ciągłą, określoną na $M$, to
$f \text{vol}_g$ jest $n$-formą.
Nie odwołując się do ogólnych definicji całek z różnych typów form
różniczkowych, w naszym przypadku będziemy mogli zapisać prosto, korzystając z
wcześniejszych uwag dotyczących zapisu formy objętości:
\[ %%% źródło tego wszystkiego 16.28 Lee, strona 422
  \int_M f d \text{vol}_g = \int_{\phi (U)} f(x) \sqrt{det(g_{ij})} dx^1 ... dx^n,
\]
zakładając, że rozmaitość jest cała w obrazie jednej mapy
$\phi$. Jeżeli bowiem tak by nie było, to musielibyśmy korzystać z
wielu map $\phi_1, \phi_2 ... $, które opisywałyby całą rozmaitość
biorąc pod uwagę gładki podział jedynki na rozmaitości. \\

\section{Norma formy różniczkowej}
Biorąc pod uwagę lokalną strukturę form różniczkowych jako algebry zewnętrznej
oraz uwagi z wcześniejszego rozdziału, iloczyn skalarny na rozmaitości
Riemannowskiej indukuje zarówno na przestrzeni kostycznej, jak i na 
potędze zewnętrznej przestrzeni kostycznej iloczyn skalarny, a w związku
z tym także normę. Dla porządku zapiszę jej lokalną postać. \\

Niech $(M, g)$ będzie zorientowaną, spójną 
rozmaitością Riemannowską.


Dla danych dwóch form zapisanych w lokalnych współrzędnych, ortogonalnych
względem iloczynu skalarnego (metryki Riemannowskiej)
\[
\alpha_x = \sum_{1 \leq i_1 < ... < i_k \leq n} \alpha_{i_1 ... i_k; x} e^{i^1}
\wedge ...  \wedge e^{i^k}
\]
 oraz
\[ \beta_x = \sum_{1 \leq i_1 < ... < i_k \leq n} \beta_{i_1 ... i_k; x} e^{i^1}
\wedge ...  \wedge e^{i^k},
\]
iloczyn skalarny na przestrzeni form daje się zapisać prostym wzorem
\[
    G(\alpha_x, \beta_x) = \sum_{i_1, ..., i_k} \alpha_{i_1 ... i_k; x}
                                                    \beta_{i_1 ... i_k; x}.
\] \\


Mamy więc dzięki temu iloczynowi określoną lokalną normę dla form różniczkowych.
Dla $\omega \in \Omega^k(M)$ oraz $x \in M$ napiszemy
\[
    |\omega|_x = \sqrt{ G(\omega_x, \omega_x) }.
\]
Warto podkreślić tu prosty, ale istotny wniosek, że po określeniu formy $\omega$
norma jest funkcją skalarną $| \omega |_x : M \rightarrow \mathbb{R} $. Jest to warte
odnotowania, ponieważ pomaga to w rozumieniu intuicji stojącej za definicją normy
na całości rozmaitości Riemannowskiej. Taka norma będzie po prostu całką 
z normy punktowej rozważanej formy różniczkowej. \\


\section{Przekształcenie indukowane, kohomologie de Rhama oraz formuła homotopii}
Rozważmy przypadek gładkiej funkcji $F: M \rightarrow N$, pomiędzy
dwoma rozmaitościami. Za pomocą tej
funkcji będziemy mogli określić przekształcenie, które pozwoli nam
zamieniać formy różniczkowe z rozmaitości $N$ na formy różniczkowe z
rozmaitości $M$. Z różniczką takiej funkcji możemy bowiem stowarzyszyć
przekształcenie
przeciągnięcia $F^\ast: \Omega^p N \rightarrow \Omega^p M$, działające w taki sposób:
\[
    (F^\ast \omega)_p(v_1, ..., v_n) =
        \omega_{F(p)}(dF_p(v_1), ..., dF_p(v_k)).
\] Przeciągnięcie jest też czasami nazywane cofnięciem. \\


Pokażemy teraz bardzo interesującą strukturę jaką mają formy różniczkowe na
rozmaitości, gdy rozpatrywać je jako kompleksy z działaniem różniczki.  W
poniższym rozumowaniu, przez $\Omega^p (M)$ oznaczać będziemy przestrzeń
gładkich $k$-form.  Niech $\M$ będzie rozmaitością a $p$ będzie nieujemną
liczbą całkowitą.  Ponieważ $d: \Omega^p (\M ) \rightarrow \Omega^{p+1}(\M) $
jest przekształceniem liniowym, jego jądro oraz obraz są podprzestrzeniami
liniowymi. Wprowadźmy oznaczenie:
\[
\mathcal{Z}^p ( \M ) =
d: \Omega^{p} (\M ) \rightarrow \Omega_{p+1}(\M) =
\{\text{$p$-formy zamknięte na $\M$ } \}
\]
\[
\mathcal{B}^p ( \M ) =
d: \Omega^{p} (\M ) \rightarrow \Omega_{p+1}(\M) =
\{\text{$p$-formy dokładne na $\M$ } \}.
\]

Jako konwencję przyjmuje się, że $\Omega^{p} ( \M ) $ jest zerową
przestrzenią wektorową gdy $p < 0$ lub $p > n = \text{dim} \M $. W
związku z tym zachodzi przykładowo $\mathcal{B}^0(\M)=0$ oraz
$\mathcal{Z}^n(\M)= \Omega^n(\M)$. \\

Sprawdzona wcześniej własność operatora różniczkowania $d \circ d = 0$ oznacza,
że każda forma dokładna jest zamknięta, czyli
$ \mathcal{B}^p ( \M) \subseteq \mathcal{Z}^p ( \M) $.
Stąd ma sens następująca definicja:

\begin{definition}
  \textbf{$p$-tą grupą kohomologii de Rhama} nazwiemy następującą
  ilorazową przestrzeń liniową:
  \[
  H^{p}_{dR} ( \M ) = \frac {\mathcal{Z}^p ( \M )} {\mathcal{B}^p ( \M )}
  \]
\end{definition}
%% źródło, jak sądzę jest to Lee, istotnie 443
Jest to rzeczywista przestrzeń liniowa i w związku z tym jest ona
grupą z działaniem dodawania wektorów. Pokażemy, że grupy de Rhama są
niezmiennicze ze względu na dyfeomorifzmy. Dla każdej domkniętej $p$-formy
$\omega$ na $\M$ poprzez $[\omega]$ będziemy oznaczać klasę równoważności
formy $\omega$ w $H_{dR} (M)$. Taką klasę równoważności będziemy nazywać także klasą
kohomologii formy $\omega$. Jeżeli dwie formy $\omega, \eta$ należą do tej samej klasy
kohomologii, czyli zachodzi $[\omega] = [\eta]$, to różnią się one conajwyżej o formę
dokładną. Zachodzi także następujący lemat:

\begin{lemma}(Przekształcenia indukowane kohomologii)
  Dla każdej gładkiej funkcji $F: \M \rightarrow N$ pomiędzy
  dwoma rozmaitościami gładkimi, przeciągnięcie 
  $F^\ast: \Omega^p(N) \rightarrow \Omega^p (M)$ przenosi formy dokładne
  na formy dokładne, a formy zamknięte na formy zamknięte. W ten sposób indukuje
  ono przekształcenie liniowe, w dalszym ciągu oznaczane jako $F^\ast$ z
  $H^p_{dR} (N)$ do $H^p_{dR} (M)$, które nazywane jest przekształceniem indukowanym
  kohomologii. \\
\end{lemma}
\begin{proof}
  Jeśli $\omega$ jest formą zamkniętą, to $d(F^\ast \omega) = F^\ast(d \omega) = 0$,
  więc $F^\ast \omega$ także jest zamknięte. Stąd wynika już, że przeciągnięcie
  to przenosi formy zamknięte na zamknięte, a dokłądne na dokładne. Przekształcenie
  indukowane jest zadane w prosty sposób. Dla $p$-formy zamkniętej $\omega$, niech
  \[
  F^\ast[\omega] = [F^\ast\omega].
  \]
  Wtedy jeśli $\omega' = \omega + d \eta$, to 
  \[
  F^\ast[\omega'] = [F^\ast\omega + d(F^\ast\eta)] = [F^\ast\omega],
  \]
  a więc przekształcenie jest dobrze zdefiniowane.
\end{proof} 
Odnotujmy za~\cite{Lee}, Corollary 17.3-4 następujące wnioski:
\begin{remark}
Dla każdej liczby całkowitej $p$, przypisanie $M \mapsto H_{\text{dR}}^p(M)$,
$F \mapsto F^ast$ jest funktorem kontrawariantnym z kategorii
rozmaitości gładkich do kategorii rzeczywistych przestrzeni wektorowych.
\end{remark}
\begin{remark}
  Rozmaitości gładkie, które są ze sobą dyfeomorficzne, mają izomorficzne grupy
  kohomologii de Rhama, bo z funktorialności
  $(F \circ G)^\ast = G^\ast \circ F^\ast$.
\end{remark}


Przedstawione powyżej wnioski mają daleko idące uogólnienie. Grupy de Rhama
okażą się być niezmiennikami topologicznymi.  Udowodnimy bowiem, że wspomniane
grupy są niezmiennikami homotopii.  Oznacza to, że homotopijnie równoważne
rozmaitości posiadać będą izomorficzne kohomologie de Rhama. Udowodnimy
następującą rzecz:

\begin{proposition}\label{homotopy-de-Rham}
    Homotopijnie gładko równoważne rozmaitości mają izomorficzne grupy de Rhama.
\end{proposition}

Przedstawię teraz dowód powyższej propozycji.  W tym rozumowaniu ciekawy jest
dla nas zarówno wynik, jak i technika, która wykorzystywana będzie do jego
udowodnienia.  Skorzystam później z bardzo podobnych technik do udowodnienia
najważniejszych twierdzeń pracy.  Wyprowadzimy bowiem równanie, które sprowadzi
naszą tezę do udowodnienia istnienia pewnego operatora o żądanych własnościach.  \\

Chcemy najpierw udowodnić, że homotopijne funkcje gładkie indukują to samo
przekształcenie kohomologii.  Rozważamy w tym celu dwie gładkie funkcje $F, G:
M \rightarrow N$.  Pokażemy, że ich przekształcenia indukowane są równe
$F^\ast = G^\ast$. Wyraźmy ten warunek w nieco innych słowach. \\

Gdy weźmiemy zamkniętą formę $p-$formę $\omega$ na $N$, aby
przekształcenia indukowane były równe, musimy być w stanie
wyprodukować taką $(p-1)$-formę $\eta$ na $M$, aby spełnione
\[
    G^\ast \omega - F^\ast \omega = d\eta.
\]

Z tego wyniknie bowiem, że
$ G^\ast [\omega] - F^\ast [\omega] =
[d\eta] = 0$. \\

Możemy podejść do tego problemu nieco bardziej systematycznie, 
szukając takie operatora
$h$, który jako argumenty bierze zamknięte $p$ formy na $N$
i działa tak, że spełniona jest zależność
\[
    d(h\omega) = G^\ast \omega - F^\ast \omega.
\] 
Zamiast definiować $h \omega$ tylko dla przypadku, kiedy $\omega$
jest zamknięta, określimy operator
$h$ z przestrzeni wszystkich gładkich $p$-form na $N$
do przestrzeni wszystkich gładkich $(p-1)$-form na $M$,
dla którego spełnione jest równanie
\[
    d(h\omega) + h(d\omega) = G^\ast \omega - F^\ast \omega.
\]
Gdy warunek ten jest bowiem spełniony to dla formy $\omega$, która
jest zakmnięta zajdzie także poprzedni warunek. \\

Następujące twierdzenia i wnioski przytaczam z książki \cite{bott}, Section
I.\S4, s 35. \\

Obliczenie rozpoczniemy od prostszego przypadku $M = \R^n$.
Niech $\pi: \R^n \times \R \rightarrow \R^n$  będzie rzutowaniem
na pierwszy czynnik, a $s$  będzie włożeniem zerowym (
na wartość $0$ na drugim czynniku). Podsumowując mamy:
\[
 \R^n \times \R
 \stackrel[\pi]{s}{\leftrightarrows} 
 \R^n
\]
\[
 \Omega (\R^n \times \R)
 \stackrel[\pi^\ast]{s^\ast}{\rightleftarrows} 
 \Omega(\R^n)
\]
\begin{align*}
    \pi(x, t) &= x \\
         s(x) &= (x, 0)
\end{align*}

Pokażemy, że funkcje indukują przeciwne do siebie izomorfizmy na
grupach kohomologii i stąd zachodzi $H^\ast(\R^{n+1}) \cong H^\ast(\R^{n})$. \\

Jako. że $\pi \circ s = id$  mamy prosto $s^\ast \circ \pi^\ast = id$. Jednak
$s \circ \pi \neq id$ i stąd dla operatorów
na formach zachodzi także $\pi^\ast \circ s^\ast \neq id$. Dla przykładu,
$\pi^\ast \circ s^\ast$ posyła funkcję $f(x, t)$ na $f(x, 0)$ czyli 
funkcję która jest stała wzdłuż każdego włókna. Okazuje się jednak
że w kohomologiach jednak $\pi^\ast \circ s^\ast$ jest identycznością.
Aby to pokazać, posłużymy się formułą homologii. Chcemy mianowicie
znaleźć funkcję $K$ na $\Omega(\R^n \times \R)$ taką, że spełnione jest
równanie:
\[
    id- \pi^\ast \circ s^\ast = \pm dK \pm Kd.
\]
Podkreślmy ponownie, że $dK \pm Kd$ przekształca formy domknięte
na formy dokładne i dlatego indukuje przekształcenie zerowe w kohomologii. 
Dla pełnej klarowności, dzieje się tak dlatego, że skoro $d \circ d = 0$
to z definicji forma zamknięta $\omega$ zachowuje wzór $d \omega = 0$ i
w konsekwencji $K d \omega = 0$. Dlatego też z powyższego wzoru pozostaje
tylko $d K \omega$, która jest formą dokładną. \\

Gdy taki operator $K$ istnieje, nazwany jest on \emph{operatorem homotopii},
a operator $\pi^\ast \circ s^\ast$ jest łańcuchowo homotopijne z identycznością.
Zwróćmy także uwagę, że operator homotopii obniża gradację formy o 1. \\

Każda forma na $\R^n \times \R$ da się wyrazić jako suma prosta następujących
dwóch podstawowych typów form:

\begin{enumerate}
    \item $(\pi^\ast \phi)f(x,t)$, 
    \item $(\pi^\ast \phi)f(x,t) \wedge dt$,
\end{enumerate}

gdzie $\phi$ jest formą określoną na przestrzeni podstawowej $\R^n$.
Zdefiniujemy teraz
$K: \Omega^{q} (\R^n \times \R) \rightarrow \Omega^{q-1} (\R^n \times \R)$ 
na poszególnych typach form jako:
\begin{enumerate}
    \item $(\pi^\ast \phi)f(x,t) \mapsto 0$,
    \item $(\pi^\ast \phi)f(x,t) \wedge dt \mapsto (\pi^\ast \phi) \int_0^t f$.
\end{enumerate}
Przystąpimy teraz do sprawdzenia, że $K$ jest rzeczywiście operatorem homotopii.
Dla uproszczenia dalszych wzorów zastosujemy uproszczenie notacji. Będziemy
pisać $\partial f / \partial x$ zamiast $\sum \partial f / \partial x^i dx^i$
oraz $\int g$ zamiast $\int g(x, t) dt$. \\

Korzystając z tej notacji, sprawdzamy dla $q$-formy typu 1:

\[
    \omega = (\pi^\ast \phi) \cdot f(x,t)
\]
\[
    (1 - \pi^\ast s^\ast) \omega =
       (\pi^\ast \phi) \cdot f(x,t) - \pi^\ast \phi \cdot f(x,0)
\]
\[
    (dK - Kd) \omega = - K d\omega = 
    -K \left(
    (d \pi^\ast \phi)f +
    (-1)^q \pi^\ast \phi \left(
       \frac{\partial f}{\partial x} \wedge dx +
       \frac{\partial f}{\partial t} \wedge dt
       \right)
    \right) =
\]
\[
    (-1)^{q-1}\pi^\ast \phi \int_0^t \frac{\partial f}{\partial t}
    = (-1)^{q-1} \pi^\ast \phi [f(x,t) - f(x,0)].
\]

Zestawiając powyższe równości możemy więc napisać
\[
    1 - \pi^\ast s^\ast = (-1)^{q-1}(dK - Kd)
\]
dla form typu 1. \\

Zbadajmy formułę homotopii dla $q$-form typu 2:
\begin{align*}
     \omega &= (\pi^\ast \phi)f dt \\
    d\omega &= (\pi^\ast d \phi)f dt + 
      (-1)^{q-1}(\pi^\ast \phi) \frac{\partial f}{\partial x} 
        \wedge dx \wedge dt \\
\end{align*}
\[
    (1 - \pi^\ast s^\ast) = \omega ~\text{ponieważ}~
     s^\ast(dt) = d(s^\ast) = d(0) = 0
\]
\[
    Kd\omega = (\pi^\ast d\phi) \int_0^tf +
      (-1)^{q-1} (\pi^\ast \phi)
          dx \wedge (\int_0^t \frac{\partial f}{\partial x})
\]
\[
    dK\omega = (\pi^\ast d\phi) \int_0^tf +
      (-1)^{q-1} (\pi^\ast \phi)
      \left[
          dx \wedge (\int_0^t \frac{\partial f}{\partial x})
          + f dt
      \right],
\]
i podsumowując zachodzi wzór
\[
    (dK - Kd) \omega = (-1)^{q-1} \omega.
\]

W obu przypadkach mamy więc 
\[
1 - \pi^\ast \circ s^\ast = (-1)^{q-1}(dK - Kd).
\] \\

Podsumowując, wnioskiem jest 
\begin{wniosek}
Przekształcenia $H^\ast (\R^n \times \R)
\stackrel[\pi^\ast]{s^\ast}{\rightleftarrows} H^\ast(\R^n)$ są izomorfizmami.
\end{wniosek}
Możemy także dzięki tym obserwacjom policzyć kohomologie $\R^n$.
\begin{wniosek}(Lemat Poincaré)
\[
H^\ast(\R^n) = H^\ast(punkt) = 
\begin{cases}
\R & \text{dla wymiaru 0} \\
0 & \text{dla wszyskich innych przypadków} \\
\end{cases}
\]
\end{wniosek} 

Możemy teraz uogólnić powyższy prostszy przypadek na przypadek dowolnej
rozmaitości. Rozważmy mianowicie
\[
 \M \times \R^1 \stackrel[\pi]{s}{\leftrightarrows} \M
\]
Jeśli ${U_\alpha}$ jest atlasem dla $\M$, wtedy ${U_\alpha \times \R^1}$ jest
atlasem $\M \times \R^1$. Ponownie można zauważyć, że każda forma na $\M \times
\R$ jest kombinacją liniową dwóch przedstawionych powyżej typów form.  Możemy
więc zdefiniować operator homotopii $K$ w taki sam sposób jak wcześniej.  Wtedy
można będzie przepisać wcześniejszy dowód zamieniając $\R^n$ na $\M$ i będzie
on nadal poprawny. Stąd dostajemy mocniejszy fakt, że $ H^\ast( \M \times
\R^1) \cong H^\ast( \M) $, gdzie izomorfizmy przeciwne to $\pi^\ast \circ
s^\ast$. Możemy też w końcu udowodnić Propozycję~\ref{homotopy-de-Rham}.

\begin{wniosek}
Homotopijne funkcje indukują tę samą mapę w kohomologii.
\end{wniosek}
\begin{proof}
Na potrzeby argumentu przypomnijmy definicję homotopii.  Niech $\N, \M$ będą
rozmaitościami.  Homotopią pomiędzy dwoma funkcjami $f,g: \M \rightarrow \N$
nazywamy funkcję $F: \M \times \R \rightarrow \N$ taką, że
\[
F(x,t) = 
\begin{cases}
f(x) & \text{dla}~t \geq 1 \\
g(x) & \text{dla}~t \leq 0 \\
\end{cases}
\]
\end{proof}
Równoważnie jeśli $s_0,s_1: \M \rightarrow  \M \times \R^1$ są $0$-włożeniem
$s_0(x) = (x, 0)$ i $1$-włożeniem $s_1(x) = (x,1)$, wtedy zachodzi
\begin{align*}
f &= F \circ s_1 \\
g &= F \circ s_0 \\
\end{align*}

a stąd także
\begin{align*}
f^\ast &= (F \circ s_1)^\ast = s_1^\ast \circ F^\ast \\
g^\ast &= (F \circ s_0)^\ast = s_0^\ast \circ F^\ast. \\
\end{align*}

Skoro więc zarówno $s_1^\ast$, jak i $s_0^\ast$ odwracają $\pi^\ast$, więc
są one równe w kohomologiach. Stąd zachodzi równoważny tezie wzór
\[
f^\ast = g^\ast.
\] 

Aby podkreślić dokładnie w jaki sposób implikuje to naszą tezę, przytoczę
definicję rozmaitości homotopijnie równoważnych. O rozmaitościach $\M, \N$
powiemy, że są homotopijnie równoważne, gdy istnieją funkcje
$f: \M \rightarrow \N$ oraz $g: \N \rightarrow \M$ takie, że
$f \circ g$ oraz $g \circ f$ są homotopijne do identyczności odpowiednio
na $M$ oraz $N$. \\



\section{$L_p$-kohomologie.}
W tym 
3. $L^p$-kohomologie (definicja za pomoca form gladkich), szacowanie norm
$\pi^*\omega, I_r\omega,$ (osobno dla $r=\infty$).

%% Trzeba jasno wypowiedziec idee
%% przewodnia: C$_fM$ jest dyfeomorficzne z$ M\times R$, chcemy porownac $H^*(M\times
%% R) z H^*(C_fM)$ uzywajac $\pi^*$ i korzystajac z formuly (*). Dlatego robimy te
%% oszacowania.

Niniejsza praca zajmuje się pewną modyfikacją kohomologii de Rhama -
$L_p$-kohomologiami.  Będziemy rozważać elementy tego samego kompleksu
łańcuchowego co w kohomologiach de Rhama, lecz z dodanym warunkiem
$p$-całkowalności form. Głównym obiektem naszego zainteresowania są
przestrzenie 
$L_p^k = {\omega \in \Omega^k(M): ||\omega|| <\infty}$, gdzie $|| \cdot ||$
to norma (całka) z formy, która będzie przedstawiona poniżej. 
Ograniczenie
naszych rozważań do form, które są $p$-całkowalne pozwala nam rozszerzyć
klasę przestrzeni, które badamy. Przy dobrym doborze funkcji wagowej, którą
ważymy metrykę Riemannowską na części nieosobliwej, możemy
bowiem rozważać rozmaitości z osobliwościami.\\

\section{Norma $L_p$ formy różniczkowej. $L_p$-kohomologie.}

\begin{definition}
$p$-normą dla $k$-formy $\omega$ na rozmaitości Riemannowskiej $\M$ nazwiemy
\begin{equation} \label{big-norm}
  || \omega ||^p =  \int_M |\omega|_x^p d \text{vol}_g(x).
\end{equation}
Formę należącą do tej przestrzeni będziemy nazywać formą $p$-całkowalną.
\end{definition}






\section{Uogólniona nierówność Hardy'ego}
Kluczowa w dalszych obliczeniach będzie nierówność, Hardy'ego łącząca całkowalność
funkcji z całkowalnością jej funkcji pierwotnej.

\begin{lemma}[Uogólniona nierówność Hardy'ego]\label{hardy}
    Rozważmy funkcję $f: \mathbb{R}_{+} \rightarrow \mathbb{R}$, funkcje-wagi
    $\phi, \psi: \mathbb{R}_{+} \rightarrow \mathbb{R}$ oraz $p, q \in
    \mathbb{R}$ takie, że $\frac{1}{p} + \frac{1}{q} = 1 $.  Zachodzi dla nich
\[
\int_0^\infty \left|
                \phi(x) \int_0^x f(t) dt
              \right|^p dx
\leq
C \int_0^\infty \left|
                    \psi(x)  f(x)
                \right|^p dx
\]
wtedy i tylko wtedy, gdy
\[
\sup_{x > 0}
\left[
\int_x^\infty  
   | \phi(t) |^p dt
\right]^{\frac{1}{p}}
\left[
\int_0^x
    | \psi(t) |^{-q} dt
\right]^{\frac{1}{q}} < + \infty,
\]
\end{lemma}
\begin{proof}
Dowód tego twierdzenia pomijam. Jest on dostępny w pracy~\cite{hardys}.
%% %%% http://www.encyclopediaofmath.org/index.php/Hardy_inequality
%% where you can find references to the original source. \\
\end{proof}

\begin{lemma}
Rozważmy pewną funkcję $f: \mathbb{R}_{+} \rightarrow \mathbb{R}$, gdzie $f
\geq 0$ oraz jej funkcję pierwotną $F(x) = \int_0^{x} f(t) dt$. Dla $\alpha > 0$ 
 $\int_0^\infty F(x)^pe^{- \alpha x}dx < \infty$
zachodzi wtedy i tylko wtedy, gdy
$\int_0^\infty f(x)^p e^{-\alpha x}dx < \infty$.
\end{lemma}
\begin{proof}
Wykorzystamy uogólnioną nierówność Hardy'ego.  Załóżmy $\psi(t) = \phi(t) =
e^{- \frac{t}{p} }$. Wtedy jeśli $\frac{1}{p} + \frac{1}{q} = 1 $, to
$\frac{1}{q} = \frac{p-1}{p}$ oraz $-q = \frac{p}{1-p}$.  Zbadajmy teraz, czy
spełniony jest warunek, aby zachodzić mogła nierówność Hardy'ego:
$$
\sup_{x > 0}
\left[
\int_x^\infty  
    e^{-t} dt
\right]^{\frac{1}{p}}
\left[
\int_0^x
    e^{-t \frac{1}{1-p}} dt
\right]^{\frac{p-1}{p}}
=
\sup_{x > 0}
    C
    e^{- \frac{x}{p}}
    \left(
        e^{\frac{x}{p-1}} - 1
    \right)^{\frac{p-1}{p}}
=
$$
$$
C
\sup_{x > 0}
    \left(
    e^{- \frac{x}{p-1}}
        e^{\frac{x}{p-1}} -
    e^{- \frac{x}{p-1}}
    \right)^{\frac{p-1}{p}}
=
C
\sup_{x > 0}
    \left(
        1 -
    e^{- \frac{x}{p-1}}
    \right)^{\frac{p-1}{p}}. 
$$
Wyrażenie to jest ograniczone dla $p> 1$. Po zastosowaniu uogólnionej
nierówności Hardy'ego i ewentualnym przeskalowaniu zmiennej $x$ do $\alpha x$
otrzymujemy tezę.
\end{proof}



\section{Obliczenie $L_p$ kohomologii $f$-stożka}
Celem tego rozdziału jest prezentacja obliczenia 
$L_p$-kohomologii Riemannowskiego $f$-stożka z funkcją wagową $f = e^{-t}$.
Obliczenie jest wykonane sposobem prezentowanym między innymi w
pracach \cite{cheeger}, \cite{youssin}, \cite{kirwan}, \cite{weber}.

\begin{definition}[$f$-stożek]
    Niech $\M$ będzie rozmaitością Riemannowską. Rozważmy przestrzeń
    $\mathbb{R}_{\geq 0} \times \M$. Określmy na tym produkcie tensor
    Riemannowski zadany przez wzór $dt^2 + f^{2}(t)g $, gdzie $g$ jest
    metryką na $M$.  Przestrzeń taką nazywamy \textbf{$f$-stożkiem}.
    Oznaczać ją będziemy przez symbol $\cfm$.
\end{definition}

\begin{definition}
  Niech $L_p^k M$ oznacza przestrzeń $p$-całkowalnych 
  $k$-form różnikowych z mierzalnymi  współczynnikami.
\end{definition}

Twierdzenie, które będziemy chcieli udowodnić, to
TUTEJ - zapisać twierdzenie w postaci, którą przedstawił Pan Weber


%% TODO -> picture % Rysuneczek z rurkom

Możemy teraz poczynić obserwację o formach różniczkowych określonych na 
$f$-stożku. Przestrzeń styczna do $\cfm$ w punkcie $(t, m)$ to:
\[
    T_{(t, m)} (\mathrm{c}^f M) = \mathbb{R} \times T_m M.
\]
W terminach form różniczkowych powiązanych z rozważanym $f$-stożkiem oznacza
to, że możemy napisać:

\[
\Omega^k(\mathbb{R} \times T_m \M) = 
\Omega^{k-1}(\M)  \oplus \Omega^k(\M).
\]
Spostrzeżenie to możemy wyrazić także w inny sposób: 

\begin{remark}
Każda $k$-forma $\omega \in \Omega^k T(\mathrm{c}^f M)$, 
a w konsekwencji każda forma z przestrzeni form $p$-całkowalnych  $L_k^p
(\cfm)$ może być zapisana jako$\omega = \eta + \xi \wedge dt$,
gdzie zarówno $\eta$, jak i  $\xi$ nie zawierają $dt$.  Zauważmy ponadto,
że $\eta$ jest $k$-formą, a $\xi$ jest $k-1$ formą. \\
\end{remark}

Ustalmy także dla klarowności nieco inną notację dotyczącą zapisywania
form różniczkowych względem lokalnych współrzędnych, która pozwoli nam 
lepiej zilustrować istotne dla nas elementy rozumowania. Dowolną $k$-formę $\eta$,
która w domyśle nie zawiera czynnika $dt$, zapisywać będziemy względem
lokalnych rzeczywstich współrzędnych
$(x_1, x_2, ... , x_n)$ na $\M$ jako:
\[
    \eta(t, x) = \sum_{\alpha \in I(k)} \eta_\alpha (t, x) dx^\alpha,
\]
gdzie $I(k)$ jest zbiorem wszystkich multiindeksów $\alpha = (\alpha_1, ...,
\alpha_k)$ takich, że $1 \leq \alpha_1 < ... < \alpha_i \leq n$, gdzie
\begin{equation}\label{notacja}
    dy^\alpha = dy^{\alpha_1} \wedge ... \wedge dy^{\alpha_k},
\end{equation}
a $\eta_\alpha$ jest gładką funkcją określoną na $(0, \infty) \times \M$. \\

Pomiędzy rozmaitościami $\M$ oraz $\cfm$ istnieją kanoniczne przekształcenia
projekcji oraz inkluzji. Żeby dobrze zilustrować w sposób w jaki działają one
na formy różniczkowe na poszczególnych rozmaitościach, przypomnijmy ich typy.
Inkluzja to funkcja:
\[
    i_r: \M \rightarrow \cfm \\
\]
\[
    i_r(x) = (x, r).
\]
Możemy za jej pomocą przeciągać formy z $\cfm$ do $\M$. Przeciągnięcie takie
oznaczymy jako $\omega_r = i_r^\ast(\omega) = i_r^\ast \eta $ dla formy $\omega
= \eta + \xi \wedge dt$. \\
Projekcja (rzutowanie) zadane jest jako:
\[
    \pi: \cfm \rightarrow \M
\]
\[
    \pi (x, t) = x.
\] \\


Rozważamy formę $\omega \in L^k_p (\cfm)$, gdzie
$\omega = \eta + \xi \wedge dt$.
Zauważmy, że metryka Riemannowska na $\cfm$ jest określona w taki sposób, że
stosowne normy spełniają następujące zależności:
$$
| \eta(t,x) |^2 = (\mathrm{e}^{-t})^{-2k} | \eta(t,x) |^2_{\M} +
(\mathrm{e}^{-t})^{-2(k-1)} | \xi(t,x) |^2_{\M},
$$
gdzie $|\cdot |_{\M} $ jest normą form różniczkowych indukowaną przez
metrykę Riemannowską na rozmaitości $\M$.  Czynnik $(\mathrm{e}^{-t})^{-2k}$
pojawia się ponieważ forma $\eta$ należy do $k$-tej potęgi zewnętrznej
przestrzeni kostycznej do rozmaitości $\cfm$ w punkcie $(t,x)$.  \\


Zbadamy teraz w jaki sposób zachowują się normy form przeciągniętych 
projekcją $\pi$ na $\cfm$.
Niech $\omega$ będzie $k$-formą na $\M$, którą można zapisać w lokalnych
współrzędnych $(x^1, ..., x^n)$ jako
\[
\omega(x) = \sum_{\alpha \in I(k)} \omega_\alpha (t, x) dx^\alpha.
\]
Po przeciągnięciu forma $\pi^\ast \omega$ na $\cfm$ jest dana w lokalnych 
współrzędnych $(t, x^1, ..., x^n)$ dokładnie tym samym wzorem
\[
\pi^\ast \omega(t,x) = \sum_{\alpha \in I(k)} \omega_\alpha (t, x) dx^\alpha.
\]

Zauważmy, że Riemannowska forma objętośći na $\cfm$ w punkcie $(t,x)$ różni się
od formy objętości na $\M$ w punkcie $x$ o czynnik $(e^{-t})^n$.  Policzmy więc
normę $\pi^\ast \omega$ jako elementu przestrzeni $L_k^p (\cfm)$.

\[
    ||\pi^\ast \omega ||^p = \int_{\cfm} |\omega |^p d \mathrm{vol}_{\cfm} =
    \int_0^\infty \left( e^{-t} \right)^{n-pk} \int_M |\omega|^p d
    \mathrm{vol}_M dt = 
\]

\[
    = \
    \int_0^\infty \left( e^{-t} \right)^{n-pk} || \omega ||_{M} dt = 
    \int_0^\infty || \omega ||_t^p dt,
\] 
gdzie
\[
|| \omega ||_r \deff || \omega_{|M \times \{r\} } || =
\mathrm{e}^{-r \cdot (\frac{n}{p} - k) }  ||\omega ||_{\M}.
\]
%%% Przypomnijmy, że $\omega_r = i_r^\ast (\eta)$. W tym przypadku \\

Zauważmy teraz w jaki sposób zachowywać się będzie norma formy, która
została przeciągnięta z podstawy, czyli rozmaitości $\M$. Dla 
$\eta \in L_p^\ast (\M)$ możemy napisać
\[
    || \eta ||_r \deff || \pi^\ast ||_r = 
\mathrm{e}^{-r \cdot (\frac{n}{p} - k) }  ||\eta ||_{\M}.
\] \\

W badanym przypadku rozmaitości $\cfm$ fakt, że spełniona jest formuła homotopii
wymaga szczegółowego uzasadnienia. Motywacja stojaca za formułą homotopii
została przestawiona we wcześniejszym rozdziale.\\

Zgodnie z 
TUTEJ odnośnik do 
rozdziału wcześniej ... zachodzi formuła (i istnieje stosowny operator 
$I_r$):

\[
    \omega - \pi^\ast(\omega_r) = dI_r \omega + I_r d\omega
\]


Zbadamy kiedy dla $p$-całkowalnej formy $\omega$ forma $I_r \omega$
także jest całkowalna. Fakt, że forma jest  $p$-całkowalna oznacza innymi słowy, że
\[
    || \omega ||^p = \int_{\cfm} |\omega|^p d \text{vol}_{\cfm} =
    \int_0^\infty \int_{\M} \left(
        (e^{-t})^{-pk} | \eta^{(t)}|^p_{\M} + 
        (e^{-t})^{-p(k-1)} | \xi^{(t)}|^p_{\M} 
    \right)
    (e^{-t})^{n} dt < \infty.
\]
Przypomnijmy, że czynnik $ (e^{-t})^{n} $ pochodzi od skalowania formy objętości, a
czynniki $p \cdot k$ pochodzą od tego, że forma jest w $k$-tej potędze zewnętrznej,
a norma podniesiona jest do $p$-tej potęgi. \\


$I_r \omega$ jest $(k-1)$ formą. Napiszemy analogicznie:
\[
    ||I_r \omega ||^p = 
    \int_{\cfm} |I_r \omega|^p  d\text{vol}_{\cfm} =
    \int_0^\infty \int_{\M} 
        (e^{-t})^{n-p(k-1)}
      \left| \int_r^t |\xi^{(\tau)} |^p_{\M} d\tau \right| dt.
\] \\

Wykorzystamy teraz lemat, który udowodniony został w dodatku, przyjmując
funkcję $f = |\xi^{(\tau)} |^p$, a za jej funkcję pierwotną 
$F = \int_r^t |\xi^{(\tau)}|^p d \tau $. \\
\emph{Komentarz: zacytować właściwy lemat}

Lemat możemy wykorzystać, tylko wtedy gdy 
wspołczynnik przy $e^{-t}$, nazwijmy go $\alpha$, jest większy od zera. Ten współczynnik to:
$\alpha = -p(k-1) + n$. Zbadajmy warunek:
\[
    -p(k-1) + n > 0
\]
\[
    -k + 1 + \frac{n}{p} > 0
\]
\[
    k < \frac{n}{p} + 1.
\]
Warunek pozwalający na wykorzystanie lematu jest więc spełniony dla $k \leq
\frac{n}{p}$. \\

Gdy warunek ten jest spełniony, stosujemy Lemat\ref{hardy},
wykorzystujący nierówność Hardy'ego i uzyskujemy rezultat, że $I_r \omega$ jest formą $p$-całkowalną. \\
Innymi słowy udowodniliśmy, że zachodzi
\begin{wniosek}
$I_r$ jest operatorem ciągłym w normie $L_p$ dla gradacji $k \leq \frac{n}{p}$
\end{wniosek}

Pokazaliśmy więc, że jeśli $\omega$ jest $k$-formą $p$-całkowalną, to  $I_r \omega$ jest
$p$-całkowalna i zachodzi wzór
\[
    \omega = dI_r \omega + I_r d \omega + \pi^\ast 
    \left(
        \eta^{(s)}.
    \right).
\]
Stąd jeśli $d \omega = 0$, to 
\[\label{form-in-image}
    \eta \in d(L^{i-1} \cfm) + \pi^\ast (L^i*\M),
\]
więc 
\[
    \pi^\ast: H \left( \M \right) \rightarrow H ( \cfm ),
\]
jest operatorem suriektywnym. Dzieje się tak ponieważ każda forma zamknięta
należy zgodnie z~\ref{form-in-image} z dokładnością do formy dokładnej do
obrazu $\pi^\ast$. A więc dowolna klasa abstrakcji form zamkniętych $\omega$
znajduje się w obrazie tego operatora, co jest definicją suriektywności. \\

Udowodnimy teraz iniektywność.
Jako, że zachodzi $d^2 = 0$ dla równiania~\ref{form-in-image}
możemy napisać
\begin{align*}
d \omega &= d(I_rd\omega) + d\pi^\ast( \eta^{(s)}) \\
         &= d(I_rd\omega) + \pi^\ast( \eta^{(s)}). \\
\end{align*}

Z definicji operatora $I_r$ wynika natychmiast, że jeśli
 $d \omega \in \pi^\ast(L_i(\M))$, to $Hd\omega = 0$.
Stąd $d \omega = d \pi^\ast(\eta^{(s)})$, a skoro $d$ komutuje z $\pi^\ast$,
to widzimy, że każda forma dokładna jest obrazem formy dokładnej. 
Każdy operator liniowy $\phi: V \rightarrow W$ jest różnowartościowy, gdy
$\ker \phi = 0_V$. Ma to miejsce dla operatora $\pi^\ast$, ponieważ
na formy dokładne przechodzą jedynie formy dokładne, co wynika wprost z pokazanych
powyżej własności. \\

\emph{Komentarz: Nie jestem pewien czy to dokładnie stąd wynika iniektywność} \\

Chcemy teraz pokazać, że dla $k > \frac{n}{p}$ mamy $H^\ast_p(\cfm) = 0$ dla
$k \geq \frac{n}{p}$. Niech $\phi$ będzie $p$-całkowalną $k$-formą na
$\cfm$. Dla określonego $a > 0$ możemy napisać, korzystając z nierówności
Cauchy'ego-Schwarza
\begin{align*}
\left(
    \int_0^a \int_\M |\phi^{(t)}|_M^p dt
\right) & \leq 
\left(
    \int_0^\infty \int_\M (e^{-t})^{n-pk} |\phi^{(t)}|^p dt
\right)
\left(
    \int_0^a \int_\M (e^{-t})^{pk-n} dt 
\right) \\
&= 
    ||\phi||_{\cfm}^p 
\left(
    \int_\M 1
\right)
\left(
    \frac{ (e^{-a})^{pk-n} - 1 }{n-pk} < \infty
\right) \\
\end{align*}
Z powyższego wzoru wynika, że całka $\int_0^\infty \int_\M |\phi^{(t)}|_M^p$ 
istnieje dla $k \geq \frac{n}{p}$ i także dla prawie wszystkich
\emph{Komentarz: Dlaczego prawie wszystkich? Pomyśleć}
$x \in \M$ całka
\[
\int_0^t \phi = \int_0^t \phi^{(\tau)} d\tau
\]
istnieje dla $t \in (0, \infty)$. \\

Bierzemy więc tak jak $p$-całkowalną $k$-formę $\omega = \eta + dt \wedge \xi$
i gdy $k-1 \geq \frac{n}{p}$, definiujemy operator homotopii jako
\[
I_\infty \omega = \int_t^\infty \xi.
\]

oraz pomocnicze operatory
\[
I_r \omega = \int_t^r \xi.
\]

Na podstawie takiej samej argumentacji jak powyżej, możemy dojść do wniosku, że
dla każdego $r \in (0, \infty)$ zdefiniowany operator jest ciągły dla normy
$L_p$. \\

\emph{Komentarz: rozwinąć i przemyśleć dokładnie to co jest napisane pod spodem}

Dodatkowo, jeżeli 
$I_\infty\omega$ jest gładka, to $I_\infty\omega$ przybliza sie
formami $I_r\omega$ dla $r \rightarrow \infty$.
Wtedy we wzorze~\ref{vanishing-eta} znika czynnik
    $\pi^\ast \left( \eta^{(r)} \right)$, pozostawiając nam formułę homotopii
postaci
\[
\omega = dI_\infty \omega + I_\infty d \omega.
\]
Oznacza to w szczególności, że gdy $d \omega = 0$ to $\omega = d I_\infty
\omega$. Innymi słowy oznacza to, że każda forma zamknięta jest formą dokładną,
a więc wszystkie formy danej grupie kohomologii należą do jednej klasy
abstrakcji i $H^i(\cfm) = 0$, gdy $k \geq \frac{n}{p}$. \\

Jeżeli forma
$I_\infty\omega$ nie jest gładka, to ... \emph{Jest już dużo trudniej}.



%%% Zbadać co się dzieje w tym przeciwnym przypadku
%% 
%% \section{Co pan Weber kazał mi zrobić}
%% 
%% \subsection{Pierwszy mail}
%% Ten fragment, który Pan kontempluje: Chodzi o to, że zamiast form gładkich
%% trzeba rozważać formy o mierzalnych współczynnikach z różniczką w sensie
%% "prądów" (currents), czyli funkcjonałów na formach. Być może to troche zbyt
%% obszerny temat na licencjat. Moim zdaniem wystarczy by Pan oszacował operator I
%% (odcałkowanie form) w normie $L^p$ i powiedział że, tak jak np w mojej pracy
%% pozwoli to dowieść znikanie kohomologii w jakichś gradacjach.
%% 
%% Czyli twierdzenie główne pracy by było: Operator
%% \[
%%  I_0:A^k \to A^{k+1}
%% \]
%% jest ciągły w normie $L^p$ dla k.... \\
%% 
%% Operator
%% \[
%% I_\infty:A^k \to A^{k+1}
%% \] jest ciągły w normie $L^p$ dla k.... ,
%% gdzie $A^k$ oznacza $k$-formy o mierzalnych współczynnikach. \\
%% 
%%  Jesli chodzi o ostatni rozdzial, to moze Pan heurystycznie powiediec co by
%% bylo gdyby kazda odcalkowana forma $I_\infty \omega$ byla gladka.
%% 
%% \subsection{Drugi mail}
%% 
%% Dalsze uwagi Pana Webera
%% 
%% Najwazniejsza jest czesc, gdzie jest szacowane $||I_r\omega||$ i
%% $||I_\infty\omega||$ i na to jeszcze czekamy. Po szcowaniu normy $||I_r\omega||$
%% trzeba podac sformulowanie jak wygladja kohomologie rozka i udowodnic co sie
%% da. Jesli $||I_r\omega||$ jest skonczone, to wszystko powinno dzialac. Jesli
%% $||I_\infty\omega||$ jest skonczone to przy dodatkowym zalozeniu, ze
%% $I_\infty\omega$ jest gldka moze Pan tez cos zrobic: $I_\infty\omega$ przybliza sie
%% formami $I_r\omega$ dla $r->\infty$ i w granicy dostajemy formule homotopii
%% pokazujaca, ze jesli $d\omega=0$, to $\omega=d(I_\infty\omega)$.
%% 
%% \subsection{Trzeci mail}
%% 
%% Ja: Chcę zapytać o jedną rzecz. Czy dobrze rozumiem, że jak chcemy zbiegać z r w
%% wyrażeniu $I_r$ do zera, \\
%% 
%% P. W.:
%% Zbieganie z r do 0 nie ma sensu. Przy zalozeniu ze badamy rozmaitosc z brzegiem
%% $M \times [0,\infty)$ formy dla $r=0$ sa zdefiniowane i mozna po prostu brac
%%     $I_0\omega(x,t)=\int_0^t .....$ \\
%% 
%% Ja: to chcemy definiować $I_r (x,t) = \int_r^t$, a jak
%% chcemy zbiegać z $r$ w $I_r$ to nieskończoności, to całka ma być $I_r(x,t) \int_t^r $? \\
%% 
%% mamy $ \int_t^r -> \int_t^\infty$
%% (lub  $\int_r^t -> \int_\infty^t$, to jest to samo z dokladnoscia do znaku) \\
%% 
%% 
%% \subsection{Trzeci mail}




\begin{thebibliography}{99}
\addcontentsline{toc}{chapter}{Bibliography}

\bibitem[Weber]{weber} Andrzej Weber, \textit{An isomorphism from
  intersection homology to $\mathrm{L}_p$-cohomology}, Forum
  Mathematicum, de Gruyer, 1995.
  
\bibitem[Cheeger]{cheeger} Jeff Cheeger, \textit{On the Hodge theory
  of Riemannian pseudomanifolds}, Proc. of Symp. in Pure Math. vol.36,

\bibitem[Kirwan, Woolf]{kirwan} Frances Kirwan, Jonathan Woolf, \textit{An Introduction
to Intersection Homology Theory}, Taylor \& Francis, LLC.

\bibitem[Bott,Tu]{bott} Raou Bott, \textit{Differetial forms in algebraic
  topology}, Springer Verlag, 1982.

\bibitem[Kostrikin]{kostrikin} Aleksiej I. Kostrikin, \textit{Wstęp do algebry.
Tom II: Algebra liniowa}, Wydawnictwo naukowe PWN, Warszawa 2012.

\bibitem[Youssin]{youssin} Boris Youssin, \textit{$\mathrm{L}_p$
  cohomology of cones and horns } J. Differential Geometry, Volume 39,
  Number 3, 1994.
  
\bibitem[Lee]{lee} Lee, \textit{Introduction to Smooth Manifolds}

\bibitem[Muckenhoupt]{hardys} Benjamin Muckenhoupt, \textit{Hardy's inequality
with weights}, Studia Mathematica, T. XLIV, 1972

\end{thebibliography}

\end{document}
