\documentclass[licencjacka]{pracamgr}
\usepackage[utf8]{inputenc}
\usepackage[T1]{fontenc} 
\usepackage{amssymb}
\usepackage{amsmath}
\usepackage{amsthm}
\usepackage{polski}

\theoremstyle{definition}
\newtheorem{definition}{Definicja}[section]

\theoremstyle{definition}
\newtheorem{remark}{Uwaga}[section]

\theoremstyle{plain}
\newtheorem{lemma}{Lemma}[section]

\theoremstyle{plain}
\newtheorem{theorem}{Theorem}[section]

%% Document global definitions
\def\cfm{\ensuremath\mathrm{c}^f\mathcal{M}}
\def\M{\ensuremath\mathcal{M}}
\newcommand\deff{\mathrel{\overset{\makebox[0pt]{\mbox{\normalfont\tiny\sffamily def}}}{=}}}


\author{Michał Garmulewicz}

\nralbumu{304742}


\title{$\mathrm{L}_p$-cohomologies of Riemannian $f$-horns.}

\tytulang{An implementation of a difference blabalizer based on the theory 
  of $\sigma$ -- $\rho$ phetors}

\kierunek{Matematyka} % Praca wykonana pod kierunkiem:
% (podać tytuł/stopień imię i nazwisko opiekuna
% Instytut
% ew. Wydział ew. Uczelnia (jeżeli nie MIM UW))
\opiekun{dra hab. Andrzeja Webera\\
              Instytut Matematyki\\
        }

% miesiąc i~rok:
\date{Maj 2015}

%Podać dziedzinę wg klasyfikacji Socrates-Erasmus:
\dziedzina{ 
11.0 Matematyka, Informatyka:\\ 
11.1 Matematyka\\ 
}

%Klasyfikacja tematyczna wedlug AMS (matematyka) lub ACM (informatyka)
\klasyfikacja{14 Algebraic Geometry\\
  14F (Co)homology theory\\
  14F40 de Rham cohomology}

% Słowa kluczowe:
\keywords{
  kohomologie de Rhama, topologia różniczkowa
}

% Tu jest dobre miejsce na Twoje własne makra i~środowiska:
\newtheorem{defi}{Definicja}[section]

% koniec definicji

\begin{document}
\maketitle

%tu idzie streszczenie na strone poczatkowa
\begin{abstract}
  In this thesis $L_p$-cohomologies of Riemannian $f$-horn are calculated.
\end{abstract}

\tableofcontents
%\listoffigures
%\listoftables

\chapter{Introduction}
\addcontentsline{toc}{chapter}{Introduction}

W pracy \cite{weber} prezentowane jest między innymi obliczenie
$L_p$-kohomologii stożka nad pseudorozmaitością Riemannowską.

The cone is given a metric of the form $dt \otimes dt
\oplus t^2 g$ and $g$ is the metric on the nonsingular part of the
original pseudomanifold. In the mentioned work, $L_p$ cohomology of
this space is presented.

%% This line of research can 
%% be traced back to Cheeger \cite{cheeger}. 
%% A similar approach was presented by Youssin \cite{youssin}, where $f$-horns
%% were considered.

 We present a slight modification of this notions, by considering
 manifolds where the Riemannian metric is of the for

\chapter{Preliminaries}
\section{Vector spaces and tensors}
If we consider a finite-dimensional vector space $V$ with a given metric $||
\cdot ||$ and define a new metric $|| x ||_r  = r \dot || x ||$. Then in the
space $(V, || \cdot||_r)^\ast$ dual to $(V, ||| \cdot |||)$, the normed is scaled
by the factor $\frac{1}{r}$, that is for any $\varphi \in V^\ast$ we get
$||\varphi||_r = \frac{1}{r} ||\varphi|| $. \\

We will make a simple observation that we will later use in the
computations.  We consider a Riemannian manifold $\mathrm{M}$ and
tagent $T_x\mathrm{M}$ and cotangent $T_x^\ast\mathrm{M}$ spaces in
the point $x$.  Let the bases of these spaces be $e_1, e_2, ..., e_n$
and dual $e_1^\ast, e_2^\ast, ..., e_n^\ast$. The volume form of this
manifold is $d\mathrm{vol} = \pm e_1^\ast \wedge e_2^\ast \wedge
... \wedge e_n^\ast $. \\

We now want to compute how forms from $\Lambda(\mathcal{M})$ are scaled
with respect to such a change in the norm. Suppose we are considering
space of $k$-forms on $\mathbb{M}$ at some arbitrary point. Then every
$k$-form can be locally expressed in a basis consisting of products of
covectors belonging to basis dual to the standard basis. That is every
$k$-form in the point $x$ using some local coordinates $(x_1, x_2,
... x_n)$ can be written $ \sum_{\mathrm{I} \in I } a_\mathrm{I}
dx_\mathrm{i}$, where $I$ is the set of $k$-indices of form
$\underbrace{(i_1, i_2, ..., i_k)}_{k~\mathrm{times}}$, with $i_1,
i_2, ... \in \{1, 2, ..., n \}$.  (following Einstein convention).
Let us see how basis vector is scaled:
\begin{multline*}
    ||dx_{i_1} \wedge dx_{i_2} \wedge ... \wedge dx_{i_k} ||_t =  
    ||dx_{i_1} ||_t \cdot ||  dx_{i_2} ||_t \cdot ... \cdot || dx_{i_k} ||_t =  \\
    \frac{1}{t}||dx_{i_1} || \cdot \frac{1}{t} ||  dx_{i_2} ||_t \cdot ...
     \cdot \frac{1}{t} || dx_{i_k} ||_t = 
    \frac{1}{t^k}||dx_{i_1} \wedge dx_{i_2} \wedge ... \wedge dx_{i_k} ||
\end{multline*} \\

This means that any $k$ form is scaled by $1/t^k$ when metric is scaled
by a factor of $t$.
This applies also to the volume form, so we obtain:
\[
d\mathrm{vol}_t = \frac{1}{t^n} d\mathrm{vol}
\]

\section{Riemannian metric}

\textbf{Riemannian metric} is a smooth symmetric covariant 2-tensor field
on manifold $\mathcal{M}$ that is positive definite at each point.(attaching
a field of linear functions that takes two variables to every point of the
manifold).

In any smooth local coordinates
$(x^i)$, Riemannian metric can be written as:
\[
    g = g_{ij} dx^i \otimes dx^j = g_{ij} dx^i dx^j
\]
where $g_{ij}$ is a positive definite matrix of smooth functions. 

The simplest example of Riemannian metric is \emph{Euclidean metric} on
$\mathbb{R}^n$ given in standard coordinates by 
\[
    g = \delta_{ij}dx^idx^j.
\]


%% Citing prof. Lee, it is common to abbreviate the symmetric product of a tensor
%% $\alpha$ with itself by $\alpha^2$, so the Euclidean metric can also be written
%% as 
%% \[
%%     g = (dx^1)^2 + ...  + (dx^n)^2,
%% \]
%% so now it is way easier to understand what exactly is meant by
%% $ dt \otimes dt + f^2g $, which shoudl be the same as $dt^2 + f^2g$. \\
%% wiki volume form hehehehe

\section{Differetial forms}

We will now recall definition of a norm of a differential form in each point of
the manifold. Let $(M. g)$ be an orientable, connected and complete Riemannian 
manifold. By $x$ we will mean a point of $X$ and $\Lambda^k T_x M$ is the vector
space of multilinear alternate maps:
\[
    \alpha_x:T_x^\star M \times ... \times T_x^\star M \times \rightarrow \mathbb{R}
\]

Using this, we recall that exterior form of degree $k$ is defined as a section of 
the $k$-th cotangent bundle
\[
    \Lambda^k M = \coprod_{x \in M} 
    \Lambda^k T_x^\star M \xrightarrow{\pi} M.
\]

Therefore for each point $x \in M$ we have a multilinear map $\alpha_x \in
\Lambda^k T_x^\star M$. At a given point it can be expressed in an easy form.
If $(e_1, ..., e_n)$ is a basis of $T_x M$ and $(e^1, ..., e^n)$ is the dual basis,
we can write
\[
    \alpha_x = \sum_{1 \leq i_1 < ... < i_k \leq n} a_{i_1 ... i_k} e^{i^1} \wedge ...
    \wedge e^{i^k}
\] 
where coefficients are $a_{i_1 ... i_k} = \alpha_x(e^{i^1}, ..., e^{i^2})$.

Using above, we can express given form in terms of local coordinates $x^1, ...,
x^n$ of an open subset $U$ of $M$. We have a basis $\frac{\partial}{\partial
x^1}, ..., \frac{\partial}{\partial x^n}$ of $T_x M$ for each $x \in U$ with
corresponding dual basis $dx^1, ..., dx^n$. For every point of the set $U$ one
can write
\[
    \alpha_x = \sum_{1 \leq i_1 < ... < i_k \leq n} a_{i_1 ... i_k}
       dx^{i^1} \wedge ...  \wedge dx^{i^k}.
\] 

On a Riemannian manifold one has a scalar product defined on $T_x M$ for each
point $x \in M$. Therefore, we can define a norm of every $k$-form $\alpha$.
We denote the scalar product on $T_x M$ by $<u, v>_x = g_x(u, v)$. Let us choose for
some basis $(e_1, ..., e_n)$ of $T_x M$ with corresponding dual basis $(e^1, ..., e^n)$.
We can argue that such basis can always be chosen to be orthogonal, because we can
always apply Gram-Schmidt's orthogonalisation algorithm to a basis given by local
coordinates. We can then define a map $G: \Lambda^k T_x M \times T_x M
\leftarrow \mathbb{R}$. Given two forms 
$ \alpha_x = \sum_{1 \leq i_1 < ... < i_k \leq n} \alpha_{i_1 ... i_k; x} e^{i^1}
\wedge ...  \wedge e^{i^k}$ and
$ \beta_x = \sum_{1 \leq i_1 < ... < i_k \leq n} \beta_{i_1 ... i_k; x} e^{i^1}
\wedge ...  \wedge e^{i^k}$, we set
\[
    G(\alpha_x, \beta_x) = \sum_{i_1, ..., i_k} \alpha_{i_1 ... i_k; x}
                                                    \beta_{i_1 ... i_k; x}.
\]

We can then cite lemma 1.1 from \cite{lausanne}.

\begin{lemma}
The above stated map $G: \Lambda^k T_x M \times T_x M $ is symmetric and positive
definite, and hence is a scalar product at any given point $x \in M$. It does not depend
on the choice of the particular basis $(e_1, ..., e_n)$ among the orhonormal
ones. Additionaly the basis $(e^1 \wedge ... \wedge e^n)$ is orthonormal for $G$.
\end{lemma}

We can also give eqivalent definition using other terms ...


\textbf{Induced map} For any smooth map $F:
M \rightarrow N$ between two smooth manifolds with or without
boundary, the pullback $F^\ast: \Omega^p N \rightarrow \Omega^p M$
carries closed forms to closed forms and exact forms to exact
forms. It thus decsends to a linear map, denoted by $F^\ast: H^p N
\rightarrow H^p M$, too. \\

Digression in digression in digression: \textbf{Pullback} of $F^\ast$ is
\[
    (F^\ast \omega)_p(v_1, ..., v_n) =
        \omega_{F(p)}(dF_p(v_1), ..., dF_p(v_k)).
\] \\


\section{Explaination about induced maps etc.}
%% Mike note - this star means induced map, which is explained in the
%% previous chapter

If we have two smooth maps $F, G:
M \rightarrow N$ and we want to prove that the induced maps are equal
$F^\ast = G^\ast$. Given a closed $p-$form $\omega$ on $N$, we need to
produce a $(p-1)$-form $\eta$ no $M$ such that

\[
    G^\ast \omega - F^\ast \omega = d\eta
\]

from this, it will follow that $ G^\ast [\omega] - F^\ast [\omega] =
[d\eta] = 0$, where $[]$ is just taking homotopy equivalence class
of given form. The author suggests a way to make it more systematic,
by finding an operator $h$, which transforms closed $p$-forms on $N$
to $(p-1)$-forms on $M$ and satisfies

\[
    d(h\omega) = G^\ast \omega - F^\ast \omega.
\]

Instead of defining $h\omega$ only when $\omega$ is close, it turns
out to be far easier to define a map $h$ from the space of
\textit{all} smooth $p$-forms on $N$ to the space of smoooth
$(p-1)$-forms on $M$, which satisfies:

\[
    d(h\omega) + h(d\omega) = G^\ast \omega - F^\ast \omega ,
\]

which implies the above equality when $\omega$ is closed. (To be
completly precise, we define a family of maps, one for each $p$, which
satisfy said equalities on adequate levels.

\[
    H(\mathcal{M} \times \mathbb{R}_{\geq})_{dR}^\ast = H(\mathcal{M})_{dR}^\ast
\]

\section{$L_p(\M)$ space}



\chapter{Obliczenie}
\addcontentsline{toc}{chapter}{Computation}

Celem tego rozdziału jest prezentacja obliczenia 
$L_p$-kohomologii Riemannowskiego $f$-stożka z funkcją wagową $f = e^{-t}$.
Obliczenie jest podobne do prac \cite{cheeger}, \cite{youssin}, \cite{kirwan},
\cite{weber}.

\begin{definition}[$f$-horn]
    Niech $\M$ będzie rozmaitością Riemannowską. Rozważmy przestrzeń
    $\mathbb{R}_{\geq 0} \times \mathcal{M}$. Określmy na tym produkcie tensor
    Riemannowski zadany przez wzór $dt^2 \oplus f^{2}(t)g $, gdzie $g$ jest
    metryką na $\mathcal{M}$.  Przestrzeń taką nazywamy \textbf{$f$-stożkiem}.
    Oznaczać ją będziemy przez symbol $\cfm$.
\end{definition}

\begin{definition}
  Niech $L_p^k \mathcal{M}$ oznacza przestrzeń $p$-całkowalnych 
  $k$-form różnikowych z mierzalnymi  współczynnikami.
\end{definition}


%% TODO -> picture % Rysuneczek z rurkom

Możemy teraz poczynić obserwację o formach różniczkowych określonych na 
$f$-stożku. Przestrzeń styczna do $\cfm$ w punkcie 

$(t, m)$ to:
\[
    T_{(t, m)} (\mathrm{c}^f \mathcal{M}) = \mathbb{R} \times T_m \mathcal{M}.
\]
W terminach form różniczkowych powiązanych z rozważanym $f$-stożkiem oznacza to, 
że możemy napisać:

\[
\Lambda^k(\mathbb{R} \times T_m \M) = 
\Lambda^{k-1}(\M)  \oplus \Lambda^k(\M).
\]
Spostrzeżenie to możemy wyrazić także w inny sposób: 

\begin{remark}
Każda $k$-forma $\omega \in \Lambda^k T(\mathrm{c}^f \mathcal{M})$, 
a w konsekwencji każda forma z przestrzeni form $p$-całkowalnych  $L_k^p
(\cfm)$ może być zapisana jako$\omega = \eta + \xi \wedge dt$,
gdzie zarówno $\eta$, jak i  $\xi$ nie zawierają $dt$.  Zauważmy ponadto,
że $\eta$ jest $k$-formą, a $\xi$ jest $k-1$ formą. \\
\end{remark}

Przypomnijmy także dla klarowności notację dotyczącą zapisywania form różniczkowych
względem lokalnych współrzędnych. Dowolną $k$-formę $\eta$, która w domyśle nie zawiera
czynnika $dt$, zapisywać będziemy względem lokalnych rzeczywstich współrzędnych
$(x_1, x_2, ... , x_n)$ na $\M$ jako:
\[
    \eta(t, x) = \sum_{\alpha \in I(k)} \eta_\alpha (t, x) dx^\alpha,
\]
gdzie $I(k)$ jest zbiorem wszystkich multiindeksów $\alpha = (alpha_1, ...,
\alpha_k)$ takich, że $1 \leq \alpha_1 < ... < \alpha_i \leq n$, gdzie
\begin{equation}\label{notacja}
    dy^\alpha = dy^{\alpha_1} \wedge ... \wedge dy^{\alpha_k},
\end{equation}
a $\eta_\alpha$ jest gładką funkcją określoną na $(0, \infty) \times \M$. \\

Pomiędzy rozmaitościami $\M$ oraz $\cfm$ istnieją kanoniczne przekształcenia
projekcji oraz inkluzji. Żeby dobrze zilustrować w sposób w jaki działają one
na formy różniczkowe na poszczególnych rozmaitościach, przypomnijmy ich typy.
Inkluzja to funkcja:
\[
    i_r: \M \rightarrow \cfm \\
\]
\[
    i_r(x) = (x, r).
\]
Możemy za jej pomocą przeciągać formy z $\cfm$ do $\M$. Przeciągnięcie takie
oznaczymy jako $\omega_r = i_r^\ast(\omega) = i_r^\ast \eta $ dla formy $\omega
= \eta + \xi \wedge dt$. \\
Projekcja (rzutowanie) zadane jest jako:
\[
    \pi: \cfm \rightarrow \M
\]
\[
    \pi (x, t) = x.
\] \\


Rozważamy formę $\omega \in L^k_p (\cfm)$, gdzie
$\omega = \eta + \xi \wedge dt$.
Zauważmy, że metryka Riemannowska na $\cfm$ jest określona w taki sposób, że
stosowne normy spełniają następujące zależności:
$$
| \eta(t,x) |^2 = (\mathrm{e}^{-t})^{-2k} | \eta(t,x) |^2_{\M} +
(\mathrm{e}^{-t})^{-2(k-1)} | \xi(t,x) |^2_{\M},
$$
gdzie $|\cdot |_{\M} $ jest normą form różniczkowych indukowaną przez
metrykę Riemannowską na rozmaitości $\M$.  Czynnik $(\mathrm{e}^{-t})^{-2k}$
pojawia się ponieważ forma $\eta$ należy do $k$-tej potęgi zewnętrznej
przestrzeni kostycznej do rozmaitości $\cfm$ w punkcie $(t,x)$.  Zauważmy
ponadto, że Riemannowska forma objętośći na $\cfm$ w punkcie $(t,x)$ różni się
od formy objętości na $\M$ w punkcie $x$ o czynnik $(e^{-t})^n$.  Policzmy więc
normę $\omega$ jako elementu przestrzeni $L_k^p (\cfm)$.

\[
    ||\omega ||^p = \int_{\cfm} |\omega |^p d \mathrm{vol}_{\cfm} =
    \int_0^\infty \left( e^{-t} \right)^{n-pk} \int_\mathcal{M} |\omega|^p d
    \mathrm{vol}_\mathcal{M} dt = 
\]
\[
    = \
    \int_0^\infty \left( e^{-t} \right)^{n-pk} || \omega_t ||_{\mathcal{M}} dt = 
    \int_0^\infty || \omega ||_t^p dt,
\] 
gdzie
\[
|| \omega ||_r \deff || \omega_{|\mathcal{M} \times \{r\} } || =
\mathrm{e}^{-r \cdot (\frac{n}{p} - k) }  ||\omega_r ||_{\M}.
\]
Przypomnijmy, że $\omega_r = i_r^\ast (\eta)$. \\

Zauważmy teraz w jaki sposób zachowywać się będzie norma formy, która
została przeciągnięta z podstawy, czyli rozmaitości $\M$. Dla 
$\eta \in L_p^\ast (\M)$ możemy napisać
\[
    || \eta ||_r \deff || \pi^\ast ||_r = 
\mathrm{e}^{-r \cdot (\frac{n}{p} - k) }  ||\eta ||_{\M}.
\] \\

Zazwyczaj w obliczeniach dotyczących kohomologii form argument pokazujący
że zachodzi formuła homotopii jest prosty. %% Przedstwiony on został
W badanym przypadku rozmaitości $\cfm$ fakt, że spełniona jest formuła homotopii
wymaga szczegółowego uzasadnienia. \\


Będziemy starać się dowieść, że zachodzi formuła:

\[
    \omega - \pi^\ast(\omega_r) = dI_r \omega + I_r d\omega
\]
W tym celu ponownie posłużymy się rozbiciem $k$-formy $\omega$, określonej na
$\cfm$ na
\[
    \omega = \eta + dt \wedge \xi.
\] 
Możemy teraz dla $\eta$ określić na $\cfm$ $k$-formę $\partial \eta / \partial
t$ zadaną w lokalnych współrzędnych $(x_1, x_2, ..., x_n)$ jako
\[
    \frac{\partial \eta}{\partial t} (t, x) =
    \sum_{\alpha \in I(k)} \frac{\eta_\alpha}{\partial t}(t, x) dx^\alpha.
\]

Wykorzystujemy tutaj notację z \ref{notacja}. Na tej samej podstawie możemy 
zapisać dla formy $\xi$ $(k-1)$ formę $\partial \xi / \partial t$ jako:
\[
    \frac{\partial \xi}{\partial t} (t, x) =
    \sum_{\alpha \in I(k-1)} \frac{\xi_\alpha}{\partial t}(t, x) dx^\alpha. 
\] \\

Może zdefiniować na stosownym
kompleksie form różnikowych

\[
    d_\M \omega =  
    \sum_{1 \leq j \leq m} \sum_{\alpha \in I(k)}
    \frac{\partial \eta_\alpha}{\partial x_j}(t, x) dx_j \wedge dy^\alpha
\]
\[
    + \sum_{1 \leq j \leq m} \sum_{\alpha \in I(k-1)}
    \frac{\partial \xi_\alpha}{\partial x_j}(t, x) dx_j \wedge dt \wedge dy^\alpha.
\]
Możemy wtedy określić różniczkę formy najpierw na podstawie $f$-stożka:
\[
    d_\M \omega = d_\M \eta - dt \wedge d_\M \xi,
\]
a następnie na całym badanym $f$-stożku:
\[
    d \omega = 
    d_\M \omega + dt \wedge \frac{\partial \eta}{\partial t} = 
    d_\M \eta + dt \wedge \left( 
        \frac{\partial \eta}{\partial t} - d_\M \xi
    \right). \\
\]

Możemy teraz postarać się zdefiniować operator homotopii.
Dla ustalonego $r \in (0, \infty)$ określimy operator
\[
    I_r: \Lambda^k(\cfm) \rightarrow \Lambda^{k-1}(\cfm),
\]
który w lokalnych współrzędnych $(x_1, x_2, ..., x_n)$ jest zadany wzorem:
\[
    (I_r \omega)(t, x) = \sum_{\alpha \in I(k-1)}
      \left(
          \int_s^t \xi(\tau, x) d\tau 
      \right) dy^\alpha.
\] 
Dla klarowności kolejnych wzorów wprowadzimy oznaczenie. Będziemy mianowicie pisać
\[
    (I_r \omega)(t, x) = \int_s^t \xi.
\] \\

Możemy teraz zbadać wyrażenia $d I_r \omega$ oraz $I_r d \omega$, które występują
w formule homotopii. Napiszemy:
\[
    d I_r \omega = d_\M \int_r^t \xi + dt \wedge \frac{\partial}{\partial t} \int_r^t \xi =
\]
\[
    = \int_r^t d_\M \xi + dt \wedge \xi
\]
oraz
\[
    I_r \omega = I_r
    \left(
        d_\M \eta + dt \wedge \left( \frac{\partial \eta}{\partial t} - d_\M \xi \right)
    \right)
\]
\[
    = \int_r^t 
        \left(
            \frac{\partial \eta}{\partial t} - d_M \xi
        \right)
\]
\[
    = \eta - \pi^\ast \left( \eta^{(r)} \right) - \int_r^t d_\M \xi,
\]

gdzie $\eta^{(r)} \in \Lambda^k (\M)$ jest zadane w lokalnych współrzędnych
$(x_1, x_2, ..., x_n)$ jako 
\[
    \eta^{(s)}(x) = \sum_{\alpha \in I(k)} \eta_\alpha(t, x) dx^\alpha.
\] \\

W związku z tym, zachodzi wzór:
\[
    d I_r \xi + I_r d \xi = dt \wedge \xi + \eta - \pi^\ast \left( \eta^{(r)} \right).
\] \\

Spróbujemy teraz zbadać, kiedy dla $p$-całkowalnej formy $\omega$ forma $I_r \omega$
także jest całkowalna. Fakt, że forma jest  $p$-całkowalna oznacza innymi słowy, że
\[
    || \omega ||^p = \int_{\cfm} |\omega|^p =
    \int_0^1 \int_{\M} \left(
        (e^{-t})^{-2k} | \eta^{(t)}|^p_{\M} + 
        (e^{-t})^{-2(k-1)} | \xi^{(t)}|^p_{\M} 
    \right)
    (e^{-t})^{m} dt.
\]

Jako że $I_r \omega$ jest $(k-1)$ formą, to możemy napisać
\[
    ||I_r \omega ||^p = 
    \int_{\cfm} ||I_r \omega||^p = 
    \int_0^1 \int_{\M} \left(
        (e^{-t})^{-p(k-1)}
        \int_r^t ||\xi^{(\tau)} d\tau||^p_{\M} (e^{-t})^n dt
    \right).
\] \\

Wykorzystamy teraz lemat, który udowodniony został w dodatku. \\

Lemat możemy wykorzystać, tylko wtedy gdy 
wspołczynnik przy $e^{-t}$, nazwijmy go $\alpha$, jest większy od zera. Ten współczynnik to:
$\alpha = -p(k-1) + n$. Zbadajmy warunek:
\[
    -p(k-1) + n > 0
\]
\[
    -k + 1 + \frac{n}{p} > 0
\]
\[
    k < \frac{n}{p} + 1.
\]
Warunek pozwalający na wykorzystanie lematu jest więc spełniony dla $k \leq \frac{n}{p}$. \\

Gdy warunek ten jest spełniony, stosujemy lemat ... i uzyskujemy rezultat, że $I_r \omega$
jest formą $p$-całkowalną. \\

Pokazaliśmy więc, że jeśli $\omega$ jest $k$-formą $p$-całkowalną, to  $I_r \omega$ jest
$p$-całkowalna i zachodzi wzór
\[
    \omega = dI_r \omega + I_r d \omega + \pi^\ast 
    \left(
        \eta^{(s)}.
    \right).
\]
Stąd jeśli $d \omega = 0$< to 
\[
    \eta \in d(L^{i-1} \cfm) + \pi^\ast (L^i*\M),
\]
więc 
\[
    \pi^\ast: H \left( \M \right) \rightarrow H ( \cfm )
\]
jest operatorem suriektywnym. \\

%%% Zbadać co się dzieje w tym przeciwnym przypadku


\chapter{Dowody lematów}


\begin{enumerate}
    \item $||\omega||_r = \mathrm{e}^{-r \cdot (\frac{n}{p} - k }  $
    \item Lemat Hardy'ego - Webera
    \item Formuła homotopii jest prawdziwa dla naszego stożka
\end{enumerate}




\begin{lemma}[Uogólniona nierówność Hardy'ego]
    Rozważmy funkcję $f: \mathbb{R}_{+} \rightarrow \mathbb{R}$, funkcje-wagi
    $\phi, \psi: \mathbb{R}_{+} \rightarrow \mathbb{R}$ oraz $p, q \in
    \mathbb{R}$ takie, że $\frac{1}{p} + \frac{1}{q} = 1 $.  Zachodzi dla nich
$$
\int_0^\infty \left|
                \phi(x) \int_0^x f(t) dt
              \right|^p dx
\leq
C \int_0^\infty \left|
                    \psi(x)  f(x)
                \right|^p dx
$$
wtedy i tylko wtedy, gdy
$$
\sup_{x > 0}
\left[
\int_x^\infty  
   | \phi(t) |^p dt
\right]^{\frac{1}{p}}
\left[
\int_0^x
    | \phi(t) |^{-q} dt
\right]^{\frac{1}{q}} < + \infty,
$$
\end{lemma}
\begin{proof}
    Dowód pomijam. 
    Jest on dostępny w ... . Zacytować papier od pana Webera.
%% %%% http://www.encyclopediaofmath.org/index.php/Hardy_inequality
%% where you can find references to the original source. \\
\end{proof}

\begin{lemma}
Rozważmy pewną funkcję $f: \mathbb{R}_{+} \rightarrow \mathbb{R}$, gdzie $f
\geq 0$ oraz jej funkcję pierwotną $F(x) = \int_0^{x} f(t) dt$. Dla $\alpha > 0$ warunek
$\int_0^\infty f(x)^p e^{-\alpha x}dx < \infty$ implikuje $\int_0^\infty
F(x)^pe^{- \alpha x}dx < \infty$.  \\
\end{lemma}
\begin{proof}
Wykorzystamy uogólnioną nierówność Hardy'ego.  Załóżmy $\psi(t) = \phi(t) =
e^{- \frac{t}{p} }$. Wtedy jeśli $\frac{1}{p} + \frac{1}{q} = 1 $, to
$\frac{1}{q} = \frac{p-1}{p}$ oraz $-q = \frac{p}{1-p}$.  Zbadajmy teraz, czy
spełniony jest warunek, aby zachodzić mogła nierówność Hardy'ego:
$$
\sup_{x > 0}
\left[
\int_x^\infty  
    e^{-t} dt
\right]^{\frac{1}{p}}
\left[
\int_0^x
    e^{-t \frac{1}{1-p}} dt
\right]^{\frac{p-1}{p}}
=
\sup_{x > 0}
    C
    e^{- \frac{x}{p}}
    \left(
        e^{\frac{x}{p-1}} - 1
    \right)^{\frac{p-1}{p}}
=
$$
$$
C
\sup_{x > 0}
    \left(
    e^{- \frac{x}{p-1}}
        e^{\frac{x}{p-1}} -
    e^{- \frac{x}{p-1}}
    \right)^{\frac{p-1}{p}}
=
C
\sup_{x > 0}
    \left(
        1 -
    e^{- \frac{x}{p-1}}
    \right)^{\frac{p-1}{p}}. 
$$
Wyrażenie to jest ograniczone dla $p> 1$. Po zastosowaniu uogólnionej
nierówności Hardy'ego i ewentualnym przeskalowaniu zmiennej $x$ do $\alpha x$
otrzymujemy tezę.
\end{proof}





\begin{thebibliography}{99}
\addcontentsline{toc}{chapter}{Bibliography}

\bibitem[Weber]{weber} Andrzej Weber, \textit{An isomorphism from
  intersection homology to $\mathrm{L}_p$-cohomology}, Forum
  Mathematicum, de Gruyer, 1995.
  
\bibitem[Cheeger]{cheeger} Jeff Cheeger, \textit{On the Hodge theory
  of Riemannian pseudomanifolds}, Proc. of Symp. in Pure Math. vol.36,
  American Mathematical Society, Providence, 1982.

\bibitem[Bott]{bott} Raou Bott, \textit{Differetial forms in algebraic
  topology}, Springer Verlag, 1982.

\bibitem[Youssin]{youssin} Boris Youssin, \textit{$\mathrm{L}_p$
  cohomology of cones and horns } J. Differential Geometry, Volume 39,
  Number 3, 1994.
  
\bibitem[Lee]{lee} Lee, \textit{Introduction to Smooth Manifolds}

\bibitem[Ducret]{lausanne} Stephen Ducret, \textit{$L_{q,p}$-Cohomology of Riemannian
    Manifolds and Simplical Complexes of Bounded Geometry}, Ecole Polytechnique Federale
    de Lausanne, Doctor Thesis, Lausanne, 2009

\end{thebibliography}

\end{document}
