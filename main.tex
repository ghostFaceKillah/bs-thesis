\documentclass[licencjacka]{pracamgr}
\usepackage{polski}
\usepackage[T1]{fontenc} 
\usepackage[utf8]{inputenc} 
\usepackage{amssymb}

\author{Michał Garmulewicz}

\nralbumu{304742}

\title{$L_p$-cohomologies of Riemannian horns.}

\tytulang{An implementation of a difference blabalizer based on the theory 
  of $\sigma$ -- $\rho$ phetors}

\kierunek{Matematyka}

% Praca wykonana pod kierunkiem:
% (podać tytuł/stopień imię i nazwisko opiekuna
% Instytut
% ew. Wydział ew. Uczelnia (jeżeli nie MIM UW))
\opiekun{dra hab. Andrzeja Webera\\
              Instytut Matematyki\\
        }

% miesiąc i~rok:
\date{Maj 2015}

%Podać dziedzinę wg klasyfikacji Socrates-Erasmus:
\dziedzina{ 
%11.0 Matematyka, Informatyka:\\ 
11.1 Matematyka\\ 
%11.2 Statystyka\\ 
%11.3 Informatyka\\ 
%11.4 Sztuczna inteligencja\\ 
%11.5 Nauki aktuarialne\\
%11.9 Inne nauki matematyczne i informatyczne
}

%Klasyfikacja tematyczna wedlug AMS (matematyka) lub ACM (informatyka)
\klasyfikacja{14 Algebraic Geometry\\
  14F (Co)homology theory\\
  14F40 de Rham cohomology}

% Słowa kluczowe:
\keywords{blabaliza różnicowa, fetory $\sigma$-$\rho$, fooizm,
  blarbarucja, blaba, fetoryka, baleronik}

% Tu jest dobre miejsce na Twoje własne makra i~środowiska:
\newtheorem{defi}{Definicja}[section]

% koniec definicji

\begin{document}
\maketitle

%tu idzie streszczenie na strone poczatkowa
\begin{abstract}
ąęźćżźżżżżż

  W~pracy przedstawiono prototypową implementację blabalizatora
  różnicowego bazującą na teorii fetorów $\sigma$-$\rho$ profesora
  Fifaka.  Wykorzystanie teorii Fifaka daje wreszcie możliwość
  efektywnego wykonania blabalizy numerycznej.  Fakt ten stanowi
  przełom technologiczny, którego konsekwencje trudno z~góry
  przewidzieć.
\end{abstract}

\tableofcontents
%\listoffigures
%\listoftables

\chapter*{Introduction}
\addcontentsline{toc}{chapter}{Introduction}

In \cite{weber} the author considers a cone over Riemannian pseudomanifold.
The cone is given a linear metric and a computation of $L_p$ cohomology of
this space is presented. We present a slight extension of this by considering
manifolds where the metric is blabla. This can be 



\chapter*{Computation}
\addcontentsline{toc}{chapter}{Computation}

Let us consider a manifold $ \mathbb{R}_{\geq 0} \times \mathcal{M} $, where
$\mathcal{M} $ is Riemannian mainfold. We will define a Riemannian tensor on
this product by $dt \otimes dt + f^{2}(t)g $, where $g$ is the metric on
$\mathcal{B}$
% Rysuneczek z rurkom
\[
    T_{(t, m)} = \mathbb{R}_+ \times T_m \mathcal{M}
\]
% Rysuneczek kwadrata

Let us take some $ \omega \in \Lambda^k(\mathbb{R} \oplus T_m \mathcal{M}) = 
\Lambda^k(\mathbb{R})  \oplus \Lambda^k(\mathcal{M}) $.
This equality lets us state that every $k$-form can be written as $\omega = \eta
+ \xi \wedge dt$, where both $\eta$ and $\xi$ do not contain $dt$.  Please note
that $\eta$ is $k$-form and $\xi$ is $k-1$ form.

If we consider a finite-dimensional vector space $V$ with a given metric $||
\cdot ||$ and define a new metric $||| x |||  = r \dot || x ||$. Then in the
space $(V, || \cdot||)^\ast$ dual to $(V, ||| \cdot |||)$, the normed is scaled
by the factor $\frac{1}{r}$. We now have bases $e_1, e_2, ..., e_n$ and dual
$e_1^\ast, e_2^\ast, ..., e_n^\ast$. Please note that $d\mathrm{vol} = \pm
e_1^\ast, e_2^\ast, ..., e_n^\ast $. This siplifies greatly the computation of
$L_p$ cohomology of the manifold in consideration.

Therefore we obtain easily $||e_1^{\ast} \wedge ... \wedge e_n^\ast || =
\frac{1}{f^k}$ and as $d\mathrm{vol} = e_1^{\ast} \wedge ... \wedge e_n^\ast $.
syrytyyyryayrayr 


% ref: http://en.wikipedia.org/wiki/Volume_form

% the most important equation of the article
\[
    \int_{\mathcal{M}} ||| \omega |||^p d\mathrm{vol} = 
    \int_{\mathcal{M}}  (f^{-k}|| \omega ||)^p =  
\]


%%%%%%%%%%%%%%%%%%%%%%%%%%%%%%%%%%%%%%%%%%%%%%%%
%%%%%%%%%%%%  Further be dragons  %%%%%%%%%%%%%%
%%%%%%%%%%%%%%%%%%%%%%%%%%%%%%%%%%%%%%%%%%%%%%%%

%%      \begin{figure}[tp]
%%        \centering
%%        \framebox{\vbox to 4cm{\vfil\hbox to
%%            7cm{\hfil\tiny.\hfil}\vfil}}
%%        \caption{Artystyczna wizja blaba w~obrazie węgierskiego artysty
%%          Josipa~A. Rozkoszy pt.~,,Blaba''}
%%      \end{figure}
%%      
%%      \chapter{Wcześniejsze implementacje blabalizatora
%%        różnicowego}\label{r:losers}
%%      
%%      \section{Podejście wprost}
%%      \begin{verbatim}
%%       )[14].
%%       ), {1234}],]. [map [cc], 1, 22]. [rho x 1]. {22; [22]},
%%             dd.
%%       [11; sigma].
%%              ss.4.c.q.42.b.ll.ls.chmod.aux.rm.foo;
%%       [112.34; rho];
%%              001110101010101010101010101010101111101001@
%%       [22%f4].
%%       cq. rep. else 7;
%%       ]. hlt
%%      \end{verbatim}
%%      
%%      \begin{center}
%%        \begin{tabular}{rrr}
%%          $\alpha$ & $\beta$ & $\gamma_7$ \\
%%          901384 & 13784 & 1341\\
%%          68746546 & 13498& 09165\\
%%          918324719& 1789 & 1310 \\
%%          9089 & 91032874& 1873 \\
%%          1 & 9187 & 19032874193 \\
%%          90143 & 01938 & 0193284 \\
%%          309132 & $-1349$ & $-149089088$ \\
%%          0202122 & 1234132 & 918324098 \\
%%          11234 & $-109234$ & 1934 \\
%%        \end{tabular}
%%      \end{center}

\begin{thebibliography}{99}
\addcontentsline{toc}{chapter}{Bibliografia}


\bibitem[Hopp96]{hopp} Claude Hopper, \textit{On some $\Pi$-hedral
    surfaces in quasi-quasi space}, Omnius University Press, 1996.

\bibitem[Leuk00]{leuk} Lechoslav Leukocyt, \textit{Oval mappings ab ovo},
  Materiały Białostockiej Konferencji Hodowców Drobiu, 2000.

%%    \bibitem[Rozk93]{JR} Josip A.~Rozkosza, \textit{O pewnych własnościach
%%        pewnych funkcji}, Północnopomorski Dziennik Matematyczny 63491
%%      (1993).
%%    
%%    \bibitem[Spy59]{spyrpt} Mrowclaw Spyrpt, \textit{A matrix is a matrix
%%        is a matrix}, Mat. Zburp., 91 (1959) 28--35.
%%    
%%    \bibitem[Sri64]{srinis} Rajagopalachari Sriniswamiramanathan,
%%      \textit{Some expansions on the Flausgloten Theorem on locally
%%        congested lutches}, J. Math.  Soc., North Bombay, 13 (1964) 72--6.
%%    
%%    \bibitem[Whi25]{russell} Alfred N. Whitehead, Bertrand Russell,
%%      \textit{Principia Mathematica}, Cambridge University Press, 1925.
%%    
%%    \bibitem[Zen69]{heu} Zenon Zenon, \textit{Użyteczne heurystyki
%%        w~blabalizie}, Młody Technik, nr~11, 1969.

\end{thebibliography}

\end{document}
