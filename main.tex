\documentclass[licencjacka]{pracamgr}
%% \usepackage{polski}
\usepackage[T1]{fontenc} 
\usepackage[utf8]{inputenc} 
\usepackage{amssymb}
\usepackage{amsmath}
\usepackage{amsthm}

\theoremstyle{definition}
\newtheorem{definition}{Definition}[section]

\theoremstyle{remark}
\newtheorem{remark}{Remark}[section]

\author{Michał Garmulewicz}

\nralbumu{304742}

\title{$L_p$-cohomologies of Riemannian horns.}

\tytulang{An implementation of a difference blabalizer based on the theory 
  of $\sigma$ -- $\rho$ phetors}

\kierunek{Matematyka} % Praca wykonana pod kierunkiem:
% (podać tytuł/stopień imię i nazwisko opiekuna
% Instytut
% ew. Wydział ew. Uczelnia (jeżeli nie MIM UW))
\opiekun{dra hab. Andrzeja Webera\\
              Instytut Matematyki\\
        }

% miesiąc i~rok:
\date{Maj 2015}

%Podać dziedzinę wg klasyfikacji Socrates-Erasmus:
\dziedzina{ 
11.0 Matematyka, Informatyka:\\ 
11.1 Matematyka\\ 
}

%Klasyfikacja tematyczna wedlug AMS (matematyka) lub ACM (informatyka)
\klasyfikacja{14 Algebraic Geometry\\
  14F (Co)homology theory\\
  14F40 de Rham cohomology}

% Słowa kluczowe:
\keywords{
  kohomologie de Rhama, topologia różniczkowa
}

% Tu jest dobre miejsce na Twoje własne makra i~środowiska:
\newtheorem{defi}{Definicja}[section]

% koniec definicji

\begin{document}
\maketitle

%tu idzie streszczenie na strone poczatkowa
\begin{abstract}
  In this thesis $L_p$-cohomologies of Riemannian $f$-horn are calculated.
\end{abstract}

\tableofcontents
%\listoffigures
%\listoftables

\chapter{Introduction}
\addcontentsline{toc}{chapter}{Introduction}

In \cite{weber} the author considers a cone over Riemannian
pseudomanifold.  The cone is given a metric of the form $dt \otimes dt
\oplus t^2 g$ and $g$ is the metric on the nonsingular part of the
original pseudomanifold. In the mentioned work, $L_p$ cohomology of
this space is presented.

This line of research can 
be traced back to Cheeger \cite{cheeger}. 
A similar approach was presented by Youssin \cite{youssin}, where $f$-horns
were considered.

 We present a slight modification of this notions, by considering
 manifolds where the Riemannian metric is of the for

\chapter{Preliminaries}
\section{Vector spaces and tensors}
Let us recall some basic facts about behaviour of norm when scaling tensors.
If we consider a finite-dimensional vector space $V$ with a given metric $||
\cdot ||$ and define a new metric $|| x ||_r  = r \dot || x ||$. Then in the
space $(V, || \cdot||_r)^\ast$ dual to $(V, ||| \cdot |||)$, the normed is scaled
by the factor $\frac{1}{r}$, that is for any $\varphi \in V^\ast$ we get
$||\varphi||_r = \frac{1}{r} ||\varphi|| $. \\

We will make a simple observation that we will later use in the
computations.  We consider a Riemannian manifold $\mathrm{M}$ and
tagent $T_x\mathrm{M}$ and cotangent $T_x^\ast\mathrm{M}$ spaces in
the point $x$.  Let the bases of these spaces be $e_1, e_2, ..., e_n$
and dual $e_1^\ast, e_2^\ast, ..., e_n^\ast$. The volume form of this
manifold is $d\mathrm{vol} = \pm e_1^\ast \wedge e_2^\ast \wedge
... \wedge e_n^\ast $. \\

We now want to compute how forms from $\Lambda(\mathbb{M})$ are scaled
with respect to such a change in the norm. Suppose we are considering
space of $k$-forms on $\mathbb{M}$ at some arbitrary point. Then every
$k$-form can be locally expressed in a basis consisting of products of
covectors belonging to basis dual to the standard basis. That is every
$k$-form in the point $x$ using some local coordinates $(x_1, x_2,
... x_n)$ can be written $ \sum_{\mathrm{I} \in I } a_\mathrm{I}
dx_\mathrm{i}$, where $I$ is the set of $k$-indices of form
$\underbrace{(i_1, i_2, ..., i_k)}_{k~\mathrm{times}}$, with $i_1,
i_2, ... \in \{1, 2, ..., n \}$.  (following Einstein convention).
Let us see how basis vector is scaled:
\begin{multline*}
    ||dx_{i_1} \wedge dx_{i_2} \wedge ... \wedge dx_{i_k} ||_t =  
    ||dx_{i_1} ||_t \cdot ||  dx_{i_2} ||_t \cdot ... \cdot || dx_{i_k} ||_t =  \\
    \frac{1}{t}||dx_{i_1} || \cdot \frac{1}{t} ||  dx_{i_2} ||_t \cdot ...
     \cdot \frac{1}{t} || dx_{i_k} ||_t = 
    \frac{1}{t^k}||dx_{i_1} \wedge dx_{i_2} \wedge ... \wedge dx_{i_k} ||
\end{multline*} \\

This means that any $k$ form is scaled by $1/t^k$ when metric is scaled
by a factor of $t$.
This applies also to the volume form, so we obtain:
\[
d\mathrm{vol}_t = \frac{1}{t^n} d\mathrm{vol}
\]

\section{Differetial forms}

\textbf{Riemannian metric} is a smooth symmetric covariant 2-tensor field
on manifold $\mathcal{M}$ that is positive definite at each point.(attaching
a field of linear functions that takes two variables to every point of the
manifold).

Consulting page 328 of Lee gives us that in any smooth local coordinates
$(x^i)$, Riemannian metric can be written as:
\[
    g = g_{ij} dx^i \otimes dx^j = g_{ij} dx^i dx^j
\]
where $g_{ij}$ is a positive definite matrix of smooth functions. 

The simplest example of Riemannian metric is \emph{Euclidean metric} on
$\mathbb{R}^n$ given in standard coordinates by 
\[
    g = \delta_{ij}dx^idx^j.
\]

Citing prof. Lee, it is common to abbreviate the symmetric product of a tensor
$\alpha$ with itself by $\alpha^2$, so the Euclidean metric can also be written
as 
\[
    g = (dx^1)^2 + ...  + (dx^n)^2,
\]
so now it is way easier to understand what exactly is meant by
$ dt \otimes dt + f^2g $, which shoudl be the same as $dt^2 + f^2g$. \\
%% wiki volume form hehehehe


\chapter{Computation}
\addcontentsline{toc}{chapter}{Computation}

The purpose of this chapter is to present the computation of
$L_p$-cohomologies of Riemannian $f$-horns.

\section{Setting}
Here we introduce definitions and make the first observations. The setting
is largely similar to the setting presented in \cite{weber}, \cite{youssin},
\cite{cheeger}. \\

\begin{definition}[$f$-horn]
Let $\mathcal{M}$ be a Riemannian manifold. Consider a space $
\mathbb{R}_{\geq 0} \times \mathcal{M}$.  Define a Riemannian tensor
on this product by $dt^2 \oplus f^{2}(t)g $, where $g$ is the metric
on $\mathcal{M}$.  Such a space will be called an
\textbf{$f$-horn}. We will denote it by $\mathrm{c}^f \mathcal{M}$. \\
\end{definition}

\begin{remark}
This terminology is present in works of Cheeger.
\end{remark}

At first, we will focus our attention on scaling functions from family
$f_\alpha(x) = \mathrm{e}^{\alpha x}$, parametrized by $\alpha \in
\mathbb{R}$.  The intuition behind such manifolds is best presented
graphically, as in the Figure ~\ref{intro-pic-1}. \\

%% TODO -> picture % Rysuneczek z rurkom

We can make a simple observation here about differential forms associated with
$f$-horn. The tangent space in the point $(t, m)$ is:
\[
    T_{(t, m)} (\mathrm{c}^f \mathcal{M} = \mathbb{R} \times T_m \mathcal{M}
\]
In terms of differential forms associated with $f$-horn, it means that
$\Lambda^k(\mathbb{R} \times T_m \mathcal{M}) = 
\Lambda^k(\mathbb{R})  \oplus \Lambda^k(\mathcal{M}) $.
It can be rephrased in friendlier terms in the following way.

\begin{remark}
Every $k$-form $\omega \in \Lambda^k T(\mathrm{c}^f \mathcal{M})$ can
be written as $\omega = \eta + \xi \wedge dt$, where both $\eta$ and
$\xi$ do not contain $dt$.  Please note that $\eta$ is $k$-form and
$\xi$ is $k-1$ form.
\end{remark}

%%% STOPPED HERE

%% citing heavily from Weber!
\begin{remark}[norms of forms]
If we have the standard inclusion:
\[
    i_r: \mathcal{M} \rightarrow \mathrm{c}^f \mathcal{M},
\]
\[
    i_r(x) = (x, r),
\]
then we can define $|| \omega ||_r := || \omega |_{\mathcal{M} \times
  \{r\}} || = f(r)^{n/p - k} ||\omega_r||$

Let
\[
    \pi:\mathrm{c}^f \mathcal{M} \rightarrow \mathcal{M}
\]
denote the projection. We can now establish how to calculate a norm
for $\eta \in L^\ast_p(\mathcal)$, namely 
$||\eta||_r := ||\pi^\ast\eta||_r = f(r)^{n/p - k} ||\eta_r||$.
\end{remark}


%% \scriptsize
%% Mike's note: Please note we have many different norms here, which
%% are used for things from different spaces.
%% \normalsize

Our goal is to give the homotopy operator (see in Lee why ??). We do so by
defining $I_r$:
\[
  I_r: \Omega^\ast( \mathrm{c}^f \mathcal{M} ) \rightarrow
  \Omega^{\ast-1}(\mathrm{c}^f \mathcal{I} ) 
\]
\[
    M_r(\omega)(x, t) = \int_r^t \xi(x, s) ds
\]

We now have to estimate $\int \xi$. \\

??? Why ?
cited:
The form $I_r\omega$ is smooth for $r \in (0,1)$, but will also consider
$r=0$ in certain cases. If $r>0$ then the homotopy formula holds:
\[
    \omega - \pi^\ast(\omega_r) = d I_r\omega + I_rd\omega.
\] 

In reference to Lemma 10.1 from prof's Weber's
Let $k < (n+1)/p $ Then the form $\pi^\ast$ is $p$-integrable for each 
$p$-integrable form $\eta \in L^k_p(\mathcal{M})$. TODO: Think what the real
difference between your idea and this below is:

%% Proof from prof Weber:
%% \[
%%     ||\pi^\ast \eta ||^p = 
%%     \int_{c \mathcal{M}} |\pi^\ast \eta(x,t)|^p d \mathrm{vol}(c\mathcal{M}) = 
%%     \int_0^1 ||\eta||_t^p dt = ||\eta||^p \int_0^1t^{n-pk} dt
%% \]

which would make my proof be:
\[
    ||\pi^\ast \eta ||^p = 
    \int_{c \mathcal{M}} |\pi^\ast \eta(x,t)|^p d \mathrm{vol}(c\mathcal{M}) = 
    \int_0^\infty ||\eta||_t^p dt = ||\eta||^p \int_0^\infty f(t)^{n-pk} dt
\]

so, if $f$ is given by $f(x) = e^\alpha$, the last integral is  
$ ||\eta||^p \int_0^\infty e^{\alpha (n-pk)} dt$. This integral is convergent
if $\alpha (n - pk) < 0$.\\

??? It seems now, that we want to say whether and when (depending on
$k, n, p, r$) for a $p$-integrable form $\omega \in \Omega^k(c^f\mathcal{M})$
the form $I_r\omega$ is integrable. (Lemma 10.3 from prof Weber's). \\

So due to reasons obove, we now estimate $\int \xi$. Let $\xi \in
L_p^i(c^f\mathcal{M}$ and $ r \in \mathbb{R}, r > 0$. Then
\[
    ||\int_r^t \xi_s ds || \leq \int_r^t ||\xi_s||ds =
\]
??? is that what I am supposed to do??
\[
    = \int_s^t f^{i-n/p}(s) f^{n/p-i}(s) ||\xi_s||ds
\]


The updated plan is to really understand this one here throughly:
What and where are we integrating. Why should $e^t$ be integrable 
on $\mathrm{R}_\geq$. Not looking too good.. \\

New understanding note: what prof Weber is doing here, is just calculating the 
cohomologies of this pseudoriemannian horn, by showing ... ? That's what should
be understood now. \\



\scriptsize
Mike's note:
It seems that what is happening here is that prof Weber is saying here,
is that $r>0$ standard homotopy formula holds, and now he is trying
to compute whether analogous formula in the $L_p$ space holds.


What's the general plan? Try to dig through prof's Weber's paper and
make sense of the whole estimation section, and later apply same ideas
to make your research. \\

What's the point of all these operators? Will try to
explain here, using Lee, Weber, Cheeger, Hatcher.  One clue is that we
have to compute/prove something like Lee, page 444.  Say we have $F,G:
M \rightarrow N$ which are smooth maps. We want to prove that induced
maps at the homotopies are equal, $F^\ast = g^\ast$. \\

Digression in digression: \textbf{Induced map} For any smooth map $F:
M \rightarrow N$ between two smooth manifolds with or without
boundary, the pullback $F^\ast: \Omega^p N \rightarrow \Omega^p M$
carries closed forms to closed forms and exact forms to exact
forms. It thus decsends to a linear map, denoted by $F^\ast: H^p N
\rightarrow H^p M$, too. \\

Digression in digression in digression: \textbf{Pullback} of $F^\ast$ is
\[
    (F^\ast \omega)_p(v_1, ..., v_n) =
        \omega_{F(p)}(dF_p(v_1), ..., dF_p(v_k)).
\] \\

Back to the main thread of thought: If we have two smooth maps $F, G:
M \rightarrow N$ and we want to prove that the induced maps are equal
$F^\ast = G^\ast$. Given a closed $p-$form $\omega$ on $N$, we need to
produce a $(p-1)$-form $\eta$ no $M$ such that

\[
    G^\ast \omega - F^\ast \omega = d\eta
\]

from this, it will follow that $ G^\ast [\omega] - F^\ast [\omega] =
[d\eta] = 0$, where $[]$ is just taking homotopy equivalence class
of given form. The author suggests a way to make it more systematic,
by finding an operator $h$, which transforms closed $p$-forms on $N$
to $(p-1)$-forms on $M$ and satisfies

\[
    d(h\omega) = G^\ast \omega - F^\ast \omega.
\]

Instead of defining $h\omega$ only when $\omega$ is close, it turns
out to be far easier to define a map $h$ from the space of
\textit{all} smooth $p$-forms on $N$ to the space of smoooth
$(p-1)$-forms on $M$, which satisfies:

\[
    d(h\omega) + h(d\omega) = G^\ast \omega - F^\ast \omega ,
\]

which implies the above equality when $\omega$ is closed. (To be
completly precise, we define a family of maps, one for each $p$, which
satisfy said equalities on adequate levels.

\[
    H(\mathcal{M} \times \mathbb{R}_{\geq})_{dR}^\ast = H(\mathcal{M})_{dR}^\ast
\]

\normalsize



%%%%%%%%%%%%%%%%%%%%%%%%%%%%%%%%%%%%%%%%%%%%%%%%
%%%%%%%%%%%%  Further be dragons  %%%%%%%%%%%%%%
%%%%%%%%%%%%%%%%%%%%%%%%%%%%%%%%%%%%%%%%%%%%%%%%

%%      \begin{figure}[tp]
%%        \centering
%%        \framebox{\vbox to 4cm{\vfil\hbox to
%%            7cm{\hfil\tiny.\hfil}\vfil}}
%%        \caption{Artystyczna wizja blaba w~obrazie węgierskiego artysty
%%          Josipa~A. Rozkoszy pt.~,,Blaba''}
%%      \end{figure}
%%      
%%      \chapter{Wcześniejsze implementacje blabalizatora
%%        różnicowego}\label{r:losers}
%%      
%%      \section{Podejście wprost}
%%      \begin{verbatim}
%%       )[14].
%%       ), {1234}],]. [map [cc], 1, 22]. [rho x 1]. {22; [22]},
%%             dd.
%%       [11; sigma].
%%              ss.4.c.q.42.b.ll.ls.chmod.aux.rm.foo;
%%       [112.34; rho];
%%              001110101010101010101010101010101111101001@
%%       [22%f4].
%%       cq. rep. else 7;
%%       ]. hlt
%%      \end{verbatim}
%%      
%%      \begin{center}
%%        \begin{tabular}{rrr}
%%          $\alpha$ & $\beta$ & $\gamma_7$ \\
%%          901384 & 13784 & 1341\\
%%          68746546 & 13498& 09165\\
%%          918324719& 1789 & 1310 \\
%%          9089 & 91032874& 1873 \\
%%          1 & 9187 & 19032874193 \\
%%          90143 & 01938 & 0193284 \\
%%          309132 & $-1349$ & $-149089088$ \\
%%          0202122 & 1234132 & 918324098 \\
%%          11234 & $-109234$ & 1934 \\
%%        \end{tabular}
%%      \end{center}

\begin{thebibliography}{99}
\addcontentsline{toc}{chapter}{Bibliografia}


\bibitem[Hopp96]{hopp} Claude Hopper, \textit{On some $\Pi$-hedral
    surfaces in quasi-quasi space}, Omnius University Press, 1996.

\bibitem[Leuk00]{leuk} Lechoslav Leukocyt, \textit{Oval mappings ab ovo},
  Materiały Białostockiej Konferencji Hodowców Drobiu, 2000.

%%    \bibitem[Rozk93]{JR} Josip A.~Rozkosza, \textit{O pewnych własnościach
%%        pewnych funkcji}, Północnopomorski Dziennik Matematyczny 63491
%%      (1993).
%%    
%%    \bibitem[Spy59]{spyrpt} Mrowclaw Spyrpt, \textit{A matrix is a matrix
%%        is a matrix}, Mat. Zburp., 91 (1959) 28--35.
%%    
%%    \bibitem[Sri64]{srinis} Rajagopalachari Sriniswamiramanathan,
%%      \textit{Some expansions on the Flausgloten Theorem on locally
%%        congested lutches}, J. Math.  Soc., North Bombay, 13 (1964) 72--6.
%%    
%%    \bibitem[Whi25]{russell} Alfred N. Whitehead, Bertrand Russell,
%%      \textit{Principia Mathematica}, Cambridge University Press, 1925.
%%    
%%    \bibitem[Zen69]{heu} Zenon Zenon, \textit{Użyteczne heurystyki
%%        w~blabalizie}, Młody Technik, nr~11, 1969.

\end{thebibliography}

\end{document}
