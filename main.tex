\documentclass[licencjacka]{pracamgr}
\usepackage[utf8]{inputenc}
\usepackage[T1]{fontenc} 
\usepackage{amssymb}
\usepackage{amsmath}
\usepackage{amsthm}
\usepackage{stackrel}
\usepackage{polski}

\theoremstyle{definition}
\newtheorem{definition}{Definicja}[section]

\theoremstyle{definition}
\newtheorem{remark}{Uwaga}[section]

\theoremstyle{plain}
\newtheorem{lemma}{Lemat}[section]

\theoremstyle{plain}
\newtheorem{proposition}{Propozycja}[section]

\theoremstyle{plain}
\newtheorem{theorem}{Twierdzenie}[section]

\theoremstyle{plain}
\newtheorem{example}{Przyład}[section]


\theoremstyle{plain}
\newtheorem{wniosek}{Wniosek}[section]

%% Document global definitions
\def\cfm{\ensuremath\mathrm{C}^fM}
\def\M{\ensuremath M}
\def\N{\ensuremath N}
\def\R{\ensuremath\mathbb{R}}
\newcommand\deff{\mathrel{\overset{\makebox[0pt]{\mbox{\normalfont\tiny\sffamily def}}}{:=}}}


\author{Michał Garmulewicz}

\nralbumu{304742}


\title{$\mathrm{L}_p$-kohomologie riemannowskich rożków}

\tytulang{$\mathrm{L}_p$-cohomologies of Riemannian horns.}

\kierunek{Matematyka} % Praca wykonana pod kierunkiem:
% (podać tytuł/stopień imię i nazwisko opiekuna
% Instytut
% ew. Wydział ew. Uczelnia (jeżeli nie MIM UW))
\opiekun{dra hab. Andrzeja Webera\\
              Instytut Matematyki\\}

% miesiąc i~rok:
\date{Wrzesień 2015}

%Podać dziedzinę wg klasyfikacji Socrates-Erasmus:
\dziedzina{ 
11.0 Matematyka, Informatyka:\\ 
11.1 Matematyka\\ 
}

%Klasyfikacja tematyczna wedlug AMS (matematyka) lub ACM (informatyka)
\klasyfikacja{14 Algebraic Geometry\\
  14F (Co)homology theory\\
  14F40 de Rham cohomology}

% Słowa kluczowe:
\keywords{
  kohomologie de Rhama, topologia różniczkowa
}

% Tu jest dobre miejsce na Twoje własne makra i~środowiska:
\newtheorem{defi}{Definicja}[section]

% koniec definicji

\begin{document}
\maketitle

%tu idzie streszczenie na strone poczatkowa
\begin{abstract}
    Tematem niniejszej pracy jest porównanie $L_p$-kohomologii
    rozmaitości Riemannowskiej i $f$-rożka skonstruowanego nad nią.
    Rożek ten jest zdefiniowany jako produkt 
    \[
    \cfm = \M \times \R_{+}
    \]
    z metryką Riemannowską postaci $dt \otimes dt + f(t)^2 g$, gdzie $g$ jest
    metryką na części nieosobliwej wyjściowej pseudorozmaitości. \\
    
    Praca opiera się na serii prac \cite{cheeger}, \cite{youssin} i \cite{weber}.
    W szczególności, w pracy \cite{weber} prezentowane jest między innymi
    obliczenie $L_p$-kohomologii stożka nad pseudorozmaitością Riemannowską dla
    funkcji $f(t) = t^k$. W niniejszej pracy rozważany będzie przypadek funkcji
    $f(t) = e^{-t}$. \\
\end{abstract}

\tableofcontents
%\listoffigures
%\listoftables

% #################################
% #            CHAPTER            #
% #################################

\chapter{Wstęp}

Kohomologie de Rhama są jednym ze standardowych narzędzi służących do badania
topologii rozmaitości.  Nie są one jednak dobrym narzędziem do badania
rozmaitości z osobliwoścami lub rozmaitości otwartych z metryką riemannowską.
Choć grupy kohomologii de Rhama są tam dobrze zdefiniowane dla części
nieosobliwej, to nie zawierają one informacji o metryce w otoczeniu osobliwości
oraz własnościach asymtpotycznych badanych przestrzeni.  Jednym ze sposobów na
badanie takich własności jest ograniczenie naszych rozważań do podkompleksu
klasycznego kompleksu de Rhama, a mianowicie form $p$-całkowalnych.
Zdefiniowana w ten sposób $L_p$ kohomologia w ogólności nie jest niezmiennikiem
topologicznym.  Zależy ona jednak jedynie od klasy quasi-izometrii metryki i
pozwala uzyskać nowe narzędzia w badaniu rozmaitości z osobliwościami.  Jednym
z typowych zabiegów jest rozważanie otoczeń osobliwości, które wyglądają
lokalnie jak ,,rożek'' $\M \times \R_{+}$ ze "ściśnięciem" podstawy $M$ w nowym
kierunku $t$ funkcją $f(t)$. Aby osiągnać ten efekt,  wyposażamy tę rozmaitość
produktową w metrykę Riemannowską postaci $dt^2 + f(t)^2 g$.  Uzyskujemy w ten
sposób pewien model otoczenia osobliwości, który może posłużyć jako narzędzie
do dekomponowania bardziej skomplikowanych rozmaitości na mniejsze kawałki,
których kohomologie są prostsze do policzenia.  \\

Podstawowym pytaniem, które możemy zadać w tym kontekście jest pytanie o
relację pomiędzy $L_p$ kohomologiami podstawy, czyli rozmaitości $\M$ a $L_p$
kohomologiami przestrzeni produktowych $M \times (0,1) $ czy $\M \times
\R_{+}$.  W literaturze opisanych jest wiele przykładów takich porównań.  W
jednej z podstawowych prac na ten temat, jaką jest \cite{cheeger} wykonane
jest obliczenie dla $L_2$ kohomologii skończonego stożka $ c\M = \M \times (0,
1)$, gdzie $M$ jest zamkniętą rozmaitością wymiaru $n$ z metryką stożkową
postaci $dt^2 + t^2 g$. W pracy~\cite{weber} jest ono uogólnione dla $L_p$
kohomologii, dla których otrzymano wynik
\[
H^k_{(p)} (c\M) = \begin{cases}
H^k (M) & \text{dla}~k < (n+1)/p \\
0 & \text{dla}~k \geq (n+1)/p \\
\end{cases}
\]
W pracy~\cite{youssin} uogólniono to obliczenie na szersze klasy funkcji $g$ na
tej samej przestrzeni $\M \times (0,1)$.  \\

Niniejsza praca zajmuje się pewną modyfikacją powyższych wyników.
Zostanie przedstawione obliczenie grup $L_p$ kohomologii dla rożka, czyli
przestrzeni ilorazowej $\M \times \R_{+}$ z metryką $dt^2 + (e^{-t})^2 g$.
Skrótowe sformułowanie najważniejszego twierdzenia to 
\begin{equation}
    H_{(p)}^k \left( \cfm \right) = \begin{cases}
      H^k(M) & k < \frac{n}{p} + 1, \\
      0 & k >  \frac{n}{p} + 1, \\
      \end{cases}
\end{equation}
gdzie $k$ to gradacja formy, $n$ to wymiar rozmaitości-podstawy, a parametr $p$
pochodzi od $p$-normy.  W przypadku granicznym $k = \frac{n}{p} + 1$ sytuacja
jest bardziej skomplikowana.  W szczególności, $H^k_{(p)}(\cfm)$ może być wymiaru
$\infty$.  Przypadek taki jest ilustrowany przez pierwsze $L_p$ kohomologie
półprostej, które umówione są w Przykładzie~\ref{infty-boundary-cohomologies}.
Wyniki w pracach~\cite{youssin},~\cite{weber} sugerują, że możliwe jest
nałożenie pewnych dodatkowych założeń, które pozwalają na osiągnięcie
dokładnych  wyników w przypadku granicznym, ale wątek ten nie jest podejmowany
w tej pracy.  \\

Omówmy krótko plan pracy.  W Rozdziale 2 przedstawione są podstawowe
zagadnienia z zakresu Algebry Liniowej, które są wykorzystywane w dalszych
obliczeniach. Szczególnie interesujące z perspektywy dalszego ciągu pracy
będzie zachowanie potęgi zewnętrznej przestrzeni liniowej ze względu na
skalowanie metryki.  W Rozdziale 3 prezentowana jest konstrukcja kohomologii
Riemanna. Ponadto dowodzę formułę homotopii i pokazuję jako wniosek, że
zachodzi $H^*(M) \simeq H^*(M\times \R)$, gdzie izomorfizm zadany jest przez
$\pi^*$. W Rozdziale 4 przedstawiona jest definicja $L_p$-kohomologii i
udowodnione jest wspomniane wyżej kluczowe twierdzenie pracy.

%%    W tej pracy licencjackiej przedstawiam drobną modyfikację tych pojęć do
%%    $e^{-t}$-stożków Riemannowskich, czyli przestrzeni będących produktem
%%    rozmaitości Riemannowskiej oraz półprostej na której określono metrykę
%%    $dt \otimes dt + (e^{-t})^2 g$. \\

% #################################
% #            CHAPTER            #
% #################################

\chapter{Algebra liniowa}
W rozdziale tym przytaczam definicje i wyprowadzam podstawowe zależności, które
pomogą nam w dalszych obliczeniach. \\


\section{Algebra zewnętrzna}\label{exterior-algebra-chapter}
W tej sekcji przytaczam definicje form zewnętrznych oraz ich własności. 
 Przytoczone
definicje podane są według podręczników~\cite{lee}, rozdział 14 oraz
~\cite{kostrikin} - rozdział 6 § 3. 
Formy zewnętrzne są dla nas istotne, ponieważ formy różniczkowe, które 
są podstawowym obiektem służącym do badania kohomologii de Rhama, 
są lokalnie elementami algebry zewnętrznej na przestrzeni stycznej
do rozmaitości. \\

\begin{definition}[$k$-kowektor]
Niech $V$ będzie skończenie wymiarową rzeczywistą przestrzenią wektorową.
Tensor działający jedynie na wektorach, ale nie na kowektorach, czyli tensor 
$T:V \times ... \times V \rightarrow \mathbb{R}$ nazwiemy kowariantnym.
Takie tensory mają wiele nazw: formy zewnętrzne,
multi-kowektory, czy też po prostu $k$-kowektory.  Przestrzeń wszystkich
$k$-kowektorów na przestrzeni $V$ ma wiele oznaczeń.  Jednym z bardziej
popularnych oznaczeń jest $T^k (V^\ast)$. 
\end{definition}
Tensor będzie nazwany \emph{alternującym}, gdy jego wartość zmieni znak w
przypadku zmienimy miejscami jego dwa wektory wejściowe.  Przestrzeń takich
tensorów, które należą do $T^k (V^\ast)$, a do tego są antysymetryczne, często
oznacza się $\Lambda^k (V^\ast)$.  Przedstawmy teraz kilka podstawowych
operacji określonych na takich tensorach. \\

\begin{definition}[produkt tensorowy]
Dla dwóch tensorów kowariantnych $f \in T^p(V^\ast) $ oraz $g \in T^q(V^\ast) $
możemy określić produkt (tensorowy) $f \otimes g \in T^{p+q}(V^\ast) $ za
pomocą wzoru:
\[
  f \otimes g(x_1, x_2, ..., x_p, x_{p+1}, x_{p+2}, ... ,x_{p+q}) =
  f(x_1, x_2, ... , x_p) \cdot g(x_{p+1}, x_{p+2}, ... , x_{p+q}).
\] 
\end{definition}
Dowolny kowariantny tensor możemy
przekształcić na tensor alternujący za pomocą przekształcenia
\emph{alternatora}, nazywanego także \emph{rzutem alternującym}.


\begin{definition}[Alternator]
Rzut alternujący jest określony w następujący sposób:
\[
\text{Alt}:T^k (V^\ast) \rightarrow  \Lambda^k (V^\ast)
\]
\[
\text{Alt}(f)(x_1, x_2, x_3, ..., x_k) = \frac{1}{k!}
  \sum_{\sigma \in S_p}
     \text{sgn} f(x_{\sigma(1)}, x_{\sigma(2)}, ..., x_{\sigma(k)}).
\] 
\end{definition}

Z pomocą alternatora określić można iloczyn zewnętrzny antysymetryczny. 

\begin{definition}[Iloczyn zewnętrzny]
Dla elementów $\omega \in \Lambda^p (V^\ast)$ oraz 
$\eta \in \Lambda^q (V^\ast)$ iloczyn zewnętrzny zadamy wzorem
\[
  \omega \wedge \eta = \frac{(p+q)!}{p!q!} \text{Alt} (\omega \otimes \eta)
\] \\
\end{definition}

Przytoczmy teraz sposób, w jaki definiuje się bazę potęgi zewnętrznej
$\Lambda^k(V^\ast)$. Niech $v_1, v_2, ... , v_n$ będzie bazą przestrzeni dualnej
$V^\ast$, która jest dualna do bazy $w_1, w_2, ..., w_n$.
Wówczas układ
\[
  B = \{ v_{i_1} \wedge v_{i_2} \wedge ... \wedge v_{i_k} : 1 \leq i_1 < i_2 < ... <i_k \leq n \}
\]
jest bazą potęgi zewnętrznej $\Lambda^k(V^\ast)$. Jest to udowodnione w
podręczniku~\cite[Rozdział §3.2, Twierdzenie 3]{kostrikin}.
Zdefiniujmy teraz \emph{algebrę zewnętrzną}.

\begin{definition}[algebra zewnętrzna]
Algebra zewnętrzna jest to suma prosta:
\[
\Lambda^\ast (V) = 
\Lambda^0(V^\ast) \oplus
\Lambda^1(V^\ast) \oplus
\Lambda^2(V^\ast) \oplus
...~.
\]
\end{definition}
Przestrzeń ta stanowi algebrę z działaniami dodawania oraz iloczynu
zewnętrznego.  Ponieważ działanie iloczynu zewnętrznego jest w $\Lambda^\ast
(V)$ dwulinowe, to wprowadza ono w tej przestrzeni strukturę algebry.  Algebra
zewnętrzna jest opisywana w wielu standardowych podręcznikach, przykładowo
\cite[Rozdział 6§3.2]{kostrikin},  albo ~\cite[Proposition 14.11]{lee}. \\

\section{Norma indukowana na potędze zewnętrznej przestrzeni dualnej.
Skalowanie norm.}\label{norm-scaling}

W tym rozdziale przedstawiam relacje pomiędzy normą określoną na danej
przestrzeni wektorowej a indukowaną przez iloczyn skalarny normą na przestrzeni
dualnej oraz na potędze zewnętrznej przestrzeni dualnej.  W szczególności, w
dalszych obliczeniach będziemy badać wyrażenia typu $r |\cdot|$, gdzie
$|\cdot|$ jest normą na skończenie wymiarowej przestrzeni liniowej.  Znaczenie
będzie miało jak zachowuje się norma na przestrzenii dualnej, gdy skalujemy
normę wyjściowej przestrzenii liniowo o czynnik $r$. \\

Niech będzie dana skończenie wymiarowa rzeczywista przestrzeń liniowa $V$ z
iloczynem skalarnym $\langle \cdot, \cdot \rangle$.  Ten iloczyn skalarny
zadaje izomorfizm pomiędzy $V$ oraz jej przestrzenią dualną $V^\ast$. 
%% Izomorfizm ten jest zadany w następujący sposób: Chcemy wektorowi $v \in V$
%% przyporządkować funkcjonał $\phi_v \in V^\ast$. Dla argumentu tego funkcjonału,
%% czyli wektora $w \in V$ ma on następujący wzór:
%% \begin{equation}
%% \phi_v (w) = \langle y, w \rangle.
%% \end{equation}
Ten izomorfizm indukuje iloczyn skalarny na $V^\ast$.  Iloczyn skalarny jest
zdeterminowany przez normę i w dalszych obliczeniach wygodniej jest używać
normy zamiast iloczynu skalarnego.  Przypomnijmy równoważny sposób określenia
normy, który ułatwi nam obliczenie zachowania normy ze względu na skalowanie.  

\begin{definition}[Norma przestrzeni dualnej]
Dla rzeczywistej przestrzeni wektorowej funkcjonał $\phi \in V^\ast$ ma
normę określoną wzorem:
\begin{equation}\label{norm-of-functional}
||\phi|| = \sup \left\{ |\phi(v)|: v \in V, ||v|| = 1 \right\},
\end{equation}
gdzie $||v||$ to norma pochodząca z wyjściowej przestrzeni wektorowej $V$.  \\
\end{definition}

Iloczyn skalarny jest także przenoszony na potęgę zewnętrzną przestrzeni
dualnej, co jest opisane na przykład w podręczniku~\cite[Rozdział
6§3]{kostrikin}.  Dla dwóch $k$-form jest on zadany wzorem

\begin{equation} \langle v^1 \wedge ....\wedge v^k, w^1 \wedge ... \wedge w^k
\rangle = \text{det} \left( \langle v^i, w^j \rangle \right), 
\end{equation}
czyli jest on równy wyznacznikowi macierzy wartości iloczynu skalarnego
zaaplikowanych do poszczególnych składowych $k$-kowektorów.  \\

Zaobserwujmy w jaki sposób zachowają się normy przestrzeni dualnej oraz potęgi
zewnętrznej, gdy przeskalujemy normę wyjściową
o czynnik liniowy $r$.
Załóżmy, że rozważamy rzeczywistą przestrzeń liniową $V$ z określoną
normą $|| \cdot ||$. Określmy nową normę dla $v \in V$:
\[
|| v ||_r = r || v ||.
\]
Z definicji~\ref{norm-of-functional} możemy bardzo prosto zauważyć, że
dla funkcjonału $\phi \in V^\ast$ nowa norma będzie dana 
wzorem
\[
|| \phi ||_r = \frac{1}{r} ||\phi ||.
\]
Zauważmy bowiem, że dla przestrzeni skończenie wymiarowej ze zwartości sfery
wynika, że supremum ze wzoru~\ref{norm-of-functional} jest osiągane.  Załóżmy,
że supremum to jest osiągane dla wektora $v$. W nowej normie wektor ten ma
normę $|| v ||_r = r \cdot || v || = r \cdot 1 = r$. Ze względu na liniowość
operatora $\phi$ widzimy, że na zbiorze $\{w \in V: ||w||_r = 1\}$ wartość $|
\phi (w) |$ będzie osiągała swoje supremum dla wielokrotności $\alpha v$.
Istotnie, $\frac{1}{r} v$ maksymalizuje tę wartość i jednocześnie widzimy, że 
\[
||\phi||_r = \frac{1}{r} |\phi(v)| = \frac{1}{r}|| \phi ||.
\]  \\

Dzięki powyższym obserwacjom otrzymyjemy w jaki sposób skaluje się iloczyn
skalarny na przestrzeni dualnej W definicji wyznacznika zakładamy bowiem, że
wyznacznik jest liniowy ze względu na mnożenie wiersza macierzy. Pomnożenie
całej macierzy kwadratowej wymiaru $k$ przez czynnik $r$ powoduje więc pomnożenie
wyznacznika przez czynnik $r^k$.  Otrzymujemy stąd natychmiast dla $k$-formy
$\omega = \omega_1 \wedge ... \wedge
\omega_k$  wzór:
\begin{equation}\label{scaling-of-norm}
|| \omega ||_r = \frac{1}{r^k} || \omega ||.
\end{equation}


% #################################
% #            CHAPTER            #
% #################################
\chapter{Formy różniczkowe i reguła homotopii}\label{chapter-ordinary-cohomology}
W tym rozdziale opisuję formy różniczkowe, kohomologie de Rhama i
udowadniam formułę homotopii.
Technika otaczająca formułę homotopii
będzie dla nas szczególnie interesująca. Będzie ona w dalszym
ciągu pracy poddawana modyfikacjom i posłuży ona 
do udowodnienia kluczowego twierdzenia pracy. 
Poniższe definicje są przytoczone w formie opartej na podręczniku~\cite{lee}
oraz ~\cite{bott}. W szczególności, dowód formuły homotopii jest zaczerpnięty
z~\cite[Rozdział I§4]{bott}. \\

Rozważamy rozmaitość $M$. Dla dowolnej przestrzeni kostycznej 
$T_p^\ast M$ rozpatrzmy jej potęgę zewnętrzną $\Lambda^k(T_p^\ast M)$.
Otrzymujemy w ten sposób przestrzeń liniową nad każdym punktem $p \in M$.
Ich sumę rozłączną oznaczymy $\Lambda^k(T^\ast M)$. Przestrzenie te,
określone nad każdym punktem z osobna, można skleić do wiązki wektorowej.
Przekroje tej wiązki stycznej to formy różniczkowe. \\

\begin{definition}[Forma różniczkowa]
  $k$-formą różniczkową na rozmaitości $M$ nazwiemy gładki przekrój wiązki
  $\Lambda^k(T^\ast M)$ nad M, czyli gładkie odwzorowanie $\omega: M \rightarrow
  \Lambda^k (T^\ast M)$, spełniające $\omega(p) \in \Lambda^k(T_p^\ast)$ dla
  każdego punktu $p \in M$.
\end{definition}

Przestrzeń $k$-form różniczkowych oznaczać będziemy przez $\Omega^k(M)$. 
Określimy na niej kluczowe dla dalszych kroków naszego rozumowania pojęcie
\emph{pochodnej zewnętrznej} formy różniczkowej. Operacja ta zwiększa stopień
formy, czyli $d: \Omega^k(M) \rightarrow \Omega^{k+1} (M)$. Pozwala to patrzeć
na nią w dalszej części rozumowania, jako na operator w ciągu przestrzeni
wektorowych $\Omega^k(M)$, tworzącym kompleks łańcuchowy. \\

Rozpocznijmy od najprostszego przypadku przywołując definicję różniczki
funkcji, czyli operację $d: C^\infty = \Omega^0(M) \rightarrow \Omega^1(M)$.
Niech $f$ będzie funkcją gładką na $M$, a $(x^i)$ -  układem współrzędnych. Na
dziedzinie tego układu określimy różniczkę $df$ wzorem
\begin{equation}\label{exterior-derivative-for-one-forms}
df = \sum_{i=1}^n \frac{\partial f}{\partial x^i} dx^i.
\end{equation} 
Napisy $dx^i$ oznaczają tutaj 1-formy będące różniczkami zewnętrznymi
poszczególnych składowych z układu współrzędnych $x_i$. Działanie to jest
dobrze zdefiniowane, ponieważ $x_i: \M \rightarrow \R^n$ są funkcjami, a więc
0-formami różniczkowymi.  \\

Uzupełnijmy definicję dla form o wyższej gradacji. Dla takich form określamy
pochodną zewnętrzną w następujący sposób. Niech 
$\omega = \sum_{j \in I} \alpha_{j_1, ..., j_k} dx^{j_1} \wedge ... \wedge dx^{j_k}$.
Różniczka $d: \Omega^k(M) \rightarrow \Omega^{k+1}(M)$ jest dana jako
\[ %% strona 363 Lee, wzór 14.20
d( \sum_{j \in I} \alpha_{j_1, ..., j_k} dx^{j_1} \wedge ... \wedge dx^{j_k}) = 
 \sum_{j \in I} \sum_{i=1}^n
 \frac{ \partial \alpha_{j_1, ..., j_k}} {\partial x^i} dx^i
                            \wedge dx^{j_1} \wedge ... \wedge dx^{j_k}).
\]
Ponadto, jeśli $\phi$ jest $p$-formą, a $\psi$ jest $q$-formą, to możemy
zapisać następujący wariant formuły Leibniza
\[
d(\phi \wedge \psi) = d\phi \wedge \psi + (-1)^p \phi \wedge d\psi.
\]
Jest on udowodniony w~\cite[Proposition 14.23 (b)]{lee}. \\

Warto zwrócić uwagę, że własność~\ref{exterior-derivative-for-one-forms} wraz z
formułą Leibniza oraz własnością $d \circ d = d^2 = 0$ determinują postać
różniczki dla form w wyższych gradacjach. \\

Przytoczymy teraz kluczową własność różniczki wraz z dowodem, która pozwoli nam
stwierdzić, że ciąg przestrzeni form różniczkowych kolejnych gradacji z
operatorem $d$ różniczki zewnętrznej jest kompleksem łańcuchowym, czyli ciągiem
przestrzeni
\[
   ... \xrightarrow{d} 
A^i 
   \xrightarrow{d} 
A^{i+1}
   \xrightarrow{d} 
...~.
\]

\begin{theorem}
Dla różniczki zewnętrznej form należących do $\left(\Omega^\ast (M), d \right)$
zachdzi
\[
d \circ d = 0.
\]
W związku z tym ciąg przestrzeni  $\left(\Omega^i (M), d \right)_i$, $i \in
\mathbb{N}$ jest kompleksem łańcuchowym.
\end{theorem}

\begin{proof}
Udowodnijmy najpierw na przypadku szczególnym 0-formy, czyli funkcji o
własnościach rzeczywistych. Dla tego przypadku zachodzi
\begin{align*}
d(df) & = d \left( \sum_j \frac{\partial f} {\partial x^j} dx^j \right) =
\sum_i \sum_j \frac{\partial^2 f}{\partial x^i \partial x^j } dx^i \wedge dx^j =  \\
& = \sum_{i < j} \left(
\frac{\partial^2 f}{\partial x^i \partial x^j}  -
\frac{\partial^2 f}{\partial x^j \partial x^i} 
 \right) dx^i \wedge dx^j = 0,
\end{align*}
z uwagi na to, że pochodne cząstkowe mieszane są sobie równe. 
Dla przypadku ogólnego natomiast, skorzystamy z powyższego przypadku szególnego
oraz formuły Leibniza, które w połączeniu pozwolą nam napisać
\begin{align*}
d(d \alpha) & = d \left( \sum_J d \alpha_J \wedge dx^{j_1} \wedge ... \wedge dx^{j_k} \right) \\
             & = \sum_J d( d\alpha_J) \wedge dx^{j_1} \wedge ... \wedge dx^{j_k}  \\
& + \sum_J \sum_{i=1}^k (-1)^i d \alpha_J \wedge dx^{j_1} \wedge ... \wedge d(dx^{j_i}) \wedge ... dx^{j_k} = 0.  \\
\end{align*}
Nadużywam tu nieco notacji dla zachowania jasności dowodu. Iteracja po $J$ to iteracja
po multiindeksach, a iteracja po $i$ to normalna konwencja sumowania. Ostatnia
równość wynika wprost z definicji różniczki form $d \alpha_J$. \\

\end{proof}




\section{Metryka Riemannowska i forma objętości}

\begin{definition}[Metryka Riemannowska]
Metryką Riemannowską nazwiemy gładkie, symetryczne kowariantne pole 2-tensorów
na rozmaitości $M$ które jest dodatnio określone w każdym punkcie. Mówiąc
bardziej intuicyjnie, określenie metryki Riemannowskiej to doczepienie pola
iloczynów skalarnych do rozmaitości, które zmienia się w sposób gładki.
Rozmaitość gładką wyposażaoną w metrykę Riemannowską nazwiemy rozmaitością
Riemannowską.
\end{definition}

W dowolnych lokalnych gładkich współrzędnych $(x^i)$ metryka Riemannowska
może być zapisana jako
\[ %%% strona 328 Lee, przed przykładem 13.1
    g = g_{ij} dx^i \otimes dx^j = g_{ij} dx^i dx^j
\]
gdzie
$g_{ij}$
jest dodatnio określoną macierzą (której współrzędne to funkcje gładkie). Ostatnia
część równości zapisuje naszą metrykę w terminach produktu symetrycznego. \\

Biorąc pod uwagę, że w głównej części pracy rozważać będziemy
rozmaitości Riemannowskie będące produktem dwóch rozmaitości
Riemannowskich, zbadajmy w jaki sposób zadana będzie metryka na takiej
przestrzeni produktowej. Jeżeli $(M, g)$ oraz $(M', g')$ będą rozmaitościami
Riemannowskimi, to na $M \times M'$ możemy zadać metrykę produktową
 $\hat g = g \oplus g'$ w następujący sposób:
\begin{equation}\label{metric-form}
\hat g
 \left( (v, v'), (w, w') \right) =
 g(v, w) + g'(v', w')
\end{equation}
dla każdego
 $(v, v'), (w, w') \in T_p M \oplus T_q M' \cong T_{(p, q)} (M \times M')$.
Gdy mamy dane jakieś konkretne współrzędne $(x_1, ... , x_n)$ dla $M$ oraz
$(y_1, ..., y_n)$ dla $M'$, to dostajemy prosto lokalne współrzędne $(x_1, ...,
x_n, y_1, ..., y_m)$ na $M \times M'$ i nietrudno sprawdzić, że metryka
produktowa jest lokalnie reprezentowana przez macierz blokowo-diagonalną
\[
  \left(\hat g_{ij} \right)  = 
  \left(
    \begin{array}{cc}
  \left( g_{ij} \right) &  0 \\
      0      & \left( g'_{ij} \right) \\
      \end{array}
  \right).
\] \\

W kolejnym rozdziale pracy będziemy mówić o całkowaniu funkcji po rozmaitości
Riemannowskiej.  Aby móc całkować funkcje na rozmaitości potrzebne jest nam
pojęcie Riemannowskiej formy objętości. Szczegóły dotyczącego tego rozumowania
są obszerne i w małym stopniu mają wpływ na tę pracę. Dlatego też przywołam
tylko definicję i twierdzenie o postaci formy objętości w lokalnych
współrzędnych, zamieszacjąc jedynie odnośnik do dowodu w źródle. \\

\begin{remark}
Niech $(M, g)$ będzie zorientowaną rozmaitością Riemannowską wymiaru
$n \geq 1$. Istnieje dokładnie jedna gładka forma objętości
$\omega_g \in \Omega^n(M)$, nazywana \textbf{Riemannowską formą objętości},
która spełnia równanie
\[
\omega_g(E_1, ...,  E_n) = 1
\]
gdzie $(E_i)$ jest lokalną, ortonormalną, dodatnio zorientowaną bazą
pól wektorowych.
\end{remark}
\begin{proof}
Pomijam, zamieszczony w oryginalnym źródle~\cite[Proposition 15.29]{lee}.
\end{proof}
Dokładne znaczenie tej definicji i jej motywacja jest poza
zakresem zainteresowania tej pracy. Musimy natomiast z perspektywy
dalszych obliczeń znać jaka jest lokalna postać formy objętości. \\

\begin{remark}\label{expression-for-volume-form}
 %% Proposition 15.31 z Lee
Niech $(M, g)$ będzie zorientowaną rozmaitością Riemannowską wymiaru $n$.
 W dowolnych zorientowanych gładkich współrzędnych
$(x_i)$, Riemannowska forma objętości może być wyrażona lokalnie w następujący
sposób:
\[
    \text{vol}_g = \sqrt{\text{det}(g_{ij})} dx^1 \wedge ... \wedge dx^n
\]
\end{remark}
Dowód tej własności także omijam, jest on dostępny w ~\cite[Proposition
15.31]{lee}. \\

Do naszych dalszych obliczeń potrzebna nam będzie umiejętność całkowania
funkcji rzeczywistych pod rozmaitościach Riemannowskich. Zdefiniujmy w tym celu
stosowną całkę.
Niech $(M, g)$ będzie zorientowaną rozmaitością Riemannowską. 
Niech $\text{vol}_g$ oznacza jej formę objętości. Jeżeli mamy teraz $f$ -
funkcję o zwartym nośniku, rzeczywistą i ciągłą, określoną na $M$, to
$f \text{vol}_g$ jest $n$-formą.
Nie odwołując się do ogólnych definicji całek z różnych typów form
różniczkowych, w naszym przypadku będziemy mogli zapisać prosto, korzystając z
wcześniejszych uwag dotyczących zapisu formy objętości:
\[ %%% źródło tego wszystkiego 16.28 Lee, strona 422
  \int_M f d \text{vol}_g = \int_{\phi (U)} f(x) \sqrt{det(g_{ij})} dx^1 ... dx^n,
\]
zakładając, że nośnik f jest cały w obrazie jednej mapy
$\phi$. Jeżeli bowiem tak by nie było, to musielibyśmy korzystać z
wielu map $\phi_1, \phi_2 ... $, które opisywałyby całą rozmaitość
biorąc pod uwagę gładki podział jedynki na rozmaitości. \\

\section{Norma formy różniczkowej indukowana przez iloczyn skalarny}
Biorąc pod uwagę lokalną strukturę form różniczkowych jako algebry
zewnętrznej oraz uwagi z rozdziału~\ref{exterior-algebra-chapter},
iloczyn skalarny na rozmaitości Riemannowskiej indukuje zarówno na
przestrzeni kostycznej, jak i na potędze zewnętrznej przestrzeni
kostycznej iloczyn skalarny, a w związku z tym także normę. Dla
porządku zapiszę jej lokalną postać. \\

Niech $(M, g)$ będzie zorientowaną, spójną 
rozmaitością Riemannowską.
Dla danych dwóch form zapisanych w lokalnych współrzędnych, ortogonalnych
względem iloczynu skalarnego (metryki Riemannowskiej)
\[
\alpha_x = \sum_{1 \leq i_1 < ... < i_k \leq n} \alpha_{i_1 ... i_k; x} e^{i^1}
\wedge ...  \wedge e^{i^k}
\]
 oraz
\[ \beta_x = \sum_{1 \leq i_1 < ... < i_k \leq n} \beta_{i_1 ... i_k; x} e^{i^1}
\wedge ...  \wedge e^{i^k},
\]
iloczyn skalarny na przestrzeni form daje się zapisać prostym wzorem
\[
    G(\alpha_x, \beta_x) = \sum_{i_1, ..., i_k} \alpha_{i_1 ... i_k; x}
                                                    \beta_{i_1 ... i_k; x}.
\] 
Mamy więc dzięki temu iloczynowi określoną lokalną normę dla form różniczkowych.
Dla $\omega \in \Omega^k(M)$ oraz $x \in M$ napiszemy
\[
    |\omega|_x = \sqrt{ G(\omega_x, \omega_x) }.
\]
Warto podkreślić tu prosty, ale istotny wniosek, że po określeniu formy
$\omega$ norma jest funkcją skalarną $| \omega |_x : M \rightarrow \mathbb{R}
$. Jest to warte odnotowania, ponieważ pomaga to w rozumieniu intuicji stojącej
za definicją normy na całości rozmaitości Riemannowskiej. Taka norma będzie po
prostu całką po całej rozmaitości z normy punktowej rozważanej formy
różniczkowej. \\


\section{Przekształcenie indukowane, kohomologie de Rhama oraz formuła
homotopii} Rozważmy przypadek gładkiej funkcji $F: M \rightarrow N$, pomiędzy
dwoma rozmaitościami. Za pomocą tej funkcji będziemy mogli określić
przekształcenie, które pozwoli nam zamieniać formy różniczkowe z rozmaitości
$N$ na formy różniczkowe z rozmaitości $M$. Z różniczką takiej funkcji możemy
bowiem stowarzyszyć nastęjące przekształcenie.

\begin{definition}[Przeciągnięcie]
Przeciągnięcie $F^\ast: \Omega^p(N) \rightarrow \Omega^p(M)$ to przekształcenie
określone wzorem
\[
    (F^\ast \omega)_p(v_1, ..., v_n) =
        \omega_{F(p)}(dF_p(v_1), ..., dF_p(v_k)),
\] gdzie $dF_p$ jest przekształceniem stycznym $T_p \M \rightarrow T_{f(p) N}$.
 Przeciągnięcie jest też czasami nazywane cofnięciem.
\end{definition}
\begin{remark}
Przeciągnięcie komutuje z różniczką zewnętrzną:
\[
F^\ast \circ d = d \circ F^\ast
\]
\end{remark}
\begin{proof}
Pomijam standardowy dowód tej własności. Jest on dostępny przykładowo
w~\cite[Lemma 14.16]{lee}.
\end{proof}
%% Jakby się komu chciało, to tu jest jakiś wnioseczek
%% https://unapologetic.wordpress.com/2011/07/21/pullbacks-on-cohomology/


Pokażemy teraz bardzo interesującą strukturę, jaką mają formy różniczkowe na
rozmaitości, gdy rozpatrywać je jako kompleksy z działaniem różniczki.  W
poniższym rozumowaniu, przez $\Omega^p (M)$ oznaczać będziemy przestrzeń
gładkich $k$-form.  Niech $\M$ będzie rozmaitością, a $p$ nieujemną
liczbą całkowitą.  Ponieważ $d: \Omega^p (\M ) \rightarrow \Omega^{p+1}(\M) $
jest przekształceniem liniowym, jego jądro oraz obraz są podprzestrzeniami
liniowymi. Wprowadźmy oznaczenie:
\[
\mathcal{Z}^p ( \M ) =
\text{ker} \left( d: \Omega^{p} (\M ) \rightarrow \Omega^{p+1}(\M) \right) =
\{\text{$p$-formy zamknięte na $\M$} \}
\]
\[
\mathcal{B}^p ( \M ) =
\text{im} \left( d: \Omega^{p-1} (\M ) \rightarrow \Omega^{p}(\M) \right) =
\{\text{$p$-formy dokładne na $\M$} \}.
\]

Jako konwencję przyjmuje się, że $\Omega^{p} ( \M ) $ jest zerową
przestrzenią wektorową gdy $p < 0$ lub $p > n = \text{dim} \M $. W
związku z tym zachodzi przykładowo $\mathcal{B}^0(\M)=0$ oraz
$\mathcal{Z}^n(\M)= \Omega^n(\M)$. \\

Sprawdzona wcześniej własność operatora różniczkowania $d \circ d = 0$ oznacza,
że każda forma dokładna jest zamknięta, czyli
$ \mathcal{B}^p ( \M) \subseteq \mathcal{Z}^p ( \M) $.
Stąd ma sens następująca definicja:

\begin{definition}
  \textbf{$p$-tą grupą kohomologii de Rhama} nazwiemy następującą
  ilorazową przestrzeń liniową:
  \[
  H^{p}_{dR} ( \M ) = \frac {\mathcal{Z}^p ( \M )} {\mathcal{B}^p ( \M )}
  \]
\end{definition}
%% źródło, jak sądzę jest to Lee, istotnie 443
Nazwywanie tych przestrzeni grupą jest uzasadnione.  Są one rzeczywistymi
przestrzeniami liniowymi i w związku z tym są grupami z działaniem dodawania
wektorów. Pokażemy, że grupy de Rhama są niezmiennicze ze względu na
dyfeomorifzmy. Dla każdej domkniętej $p$-formy $\omega$ na $\M$ poprzez
$[\omega]$ będziemy oznaczać klasę równoważności formy $\omega$ w $H_{dR} (M)$.
Taką klasę równoważności będziemy nazywać także klasą kohomologii formy
$\omega$. Jeżeli dwie formy $\omega, \eta$ należą do tej samej klasy
kohomologii, czyli zachodzi $[\omega] = [\eta]$, to różnią się one co najwyżej
o formę dokładną. Zachodzi także następujący lemat:

\begin{lemma}[Przekształcenia indukowane kohomologii]
  Dla każdej gładkiej funkcji {$F: \M \rightarrow N$} pomiędzy dwoma
  rozmaitościami gładkimi, przeciągnięcie $F^\ast: \Omega^p(N) \rightarrow
  \Omega^p (M)$  jest przekształceniem kompleksów:
  \[
    d \circ F^\ast = F^\ast \circ d
  \]
%% Po ludzku: przenosi formy dokładne na formy dokładne, a formy zamknięte na
%%  formy zamknięte.
\end{lemma}
\begin{wniosek}
  Powyższe przekształcenie indukuje przekształcenie liniowe, w dalszym
  ciągu oznaczane jako $F^\ast$ z $H^p_{dR} (N)$ do $H^p_{dR} (M)$, które
  nazywane jest przekształceniem indukowanym kohomologii. \\
\end{wniosek}
\begin{proof}
  Jeśli $\omega$ jest formą zamkniętą, to $d(F^\ast \omega) = F^\ast(d \omega) = 0$,
  więc $F^\ast \omega$ także jest zamknięte. Stąd wynika już, że przeciągnięcie
  to przenosi formy zamknięte na zamknięte, a dokłądne na dokładne. Przekształcenie
  indukowane jest zadane w prosty sposób. Dla $p$-formy zamkniętej $\omega$, niech
  \[
  F^\ast[\omega] = [F^\ast\omega].
  \]
  Wtedy jeśli $\omega' = \omega + d \eta$, to 
  \[
  F^\ast[\omega'] = [F^\ast\omega + d(F^\ast\eta)] = [F^\ast\omega],
  \]
  a więc przekształcenie jest dobrze zdefiniowane.
\end{proof} 
Odnotujmy za~\cite[Corollary 17.3-4]{lee}, następujące wnioski:
\begin{remark}
Dla każdej liczby całkowitej $p$, przypisanie $M \mapsto H_{\text{dR}}^p(M)$,
$F \mapsto F^\ast$ jest funktorem kontrawariantnym z kategorii
rozmaitości gładkich do kategorii rzeczywistych przestrzeni wektorowych.
\end{remark}
\begin{remark}
  Rozmaitości gładkie, które są ze sobą dyfeomorficzne, mają izomorficzne grupy
  kohomologii de Rhama, bo z funktorialności
  $(F \circ G)^\ast = G^\ast \circ F^\ast$.
\end{remark}


Przedstawione powyżej wnioski mają daleko idące uogólnienie. Grupy de Rhama
okażą się być niezmiennikami topologicznymi.  Udowodnimy bowiem, że wspomniane
grupy są niezmiennikami homotopii.  Oznacza to, że homotopijnie równoważne
rozmaitości posiadać będą izomorficzne kohomologie de Rhama. Udowodnimy
następującą rzecz:

\begin{proposition}\label{homotopy-de-Rham}
    Homotopijnie równoważne rozmaitości mają izomorficzne grupy de Rhama.
\end{proposition}

Przedstawię teraz dowód powyższej propozycji.
  W tym rozumowaniu ciekawy jest
dla nas zarówno wynik, jak i technika, która wykorzystywana będzie do jego
udowodnienia.  Skorzystam później z bardzo podobnych technik do udowodnienia
najważniejszych twierdzeń pracy.  Wyprowadzimy bowiem równanie, które sprowadzi
naszą tezę do udowodnienia istnienia pewnego operatora o żądanych własnościach.  \\

Funkcje, które zadają homotopijną równoważność niekoniecznie muszą być gładkie.
Zauważmy jednak, że biorąc ich aproksymacje, możemy ograniczyć nasze rozważanie
do funkcji gładkich. \\

Chcemy najpierw udowodnić, że homotopijne funkcje indukują to samo
przekształcenie kohomologii.  Weźmy więc dwie gładkie funkcje $F, G: M
\rightarrow N$.  Pokażemy, że ich przekształcenia indukowane są równe $F^\ast =
G^\ast$. Wyraźmy ten warunek w nieco inny sposób. \\

Gdy weźmiemy zamkniętą formę $p-$formę $\omega$ na $N$, aby
przekształcenia indukowane były równe, musimy być w stanie
wyprodukować taką $(p-1)$-formę $\eta$ na $M$, aby spełnione
\[
    G^\ast \omega - F^\ast \omega = d\eta.
\]
Z tego wyniknie bowiem, że
$ G^\ast [\omega] - F^\ast [\omega] =
[d\eta] = 0$. 
Możemy podejść do tego problemu nieco bardziej systematycznie, 
szukając takiego operatora
$I$, który jako argumenty bierze zamknięte $p$ formy na $N$
i działa tak, że spełniona jest zależność
\[
    d(I\omega) = G^\ast \omega - F^\ast \omega.
\] 
Zamiast definiować $I \omega$ tylko dla przypadku, kiedy $\omega$
jest zamknięta, określimy operator
$I$ z przestrzeni wszystkich gładkich $p$-form na $N$
do przestrzeni wszystkich gładkich $(p-1)$-form na $M$,
dla którego spełnione jest równanie
\begin{equation}\label{homotopy-formula}
    d(I\omega) + I(d\omega) = G^\ast \omega - F^\ast \omega.
\end{equation}
Gdy warunek ten jest bowiem spełniony to dla formy $\omega$, która
jest zakmnięta zajdzie także poprzedni warunek.  Zanim udowodnimy
ten wzór, udowodnimy prostszy lemat dla $M = \R^n$. \\

Następujące twierdzenia i wnioski przytaczam z książki \cite[Section
I.\S4, s 35]{bott}. \\

Niech $\pi: \R^n \times \R \rightarrow \R^n$  będzie rzutowaniem
na pierwszy czynnik, a $s$  będzie włożeniem zerowym (
na wartość $0$ na drugim czynniku). Podsumowując mamy:
\[
 \R^n \times \R
 \stackrel[\pi]{s}{\leftrightarrows} 
 \R^n
\]
\[
 \Omega (\R^n \times \R)
 \stackrel[\pi^\ast]{s^\ast}{\rightleftarrows} 
 \Omega(\R^n)
\]
\begin{align*}
    \pi(x, t) &= x \\
         s(x) &= (x, 0)
\end{align*}

\begin{lemma}
Funkcje $\pi$ oraz $s$ indukują przeciwne do siebie izomorfizmy na grupach
kohomologii de Rhama i stąd zachodzi $H^\ast(\R^{n+1}) \cong H^\ast(\R^{n})$.
\end{lemma}


Jako. że $\pi \circ s = id$  mamy prosto $s^\ast \circ \pi^\ast = id$. Jednak
$s \circ \pi \neq id$ i stąd dla operatorów
na formach zachodzi także $\pi^\ast \circ s^\ast \neq id$. Dla przykładu,
$\pi^\ast \circ s^\ast$ posyła funkcję $f(x, t)$ na $f(x, 0)$ czyli 
funkcję która jest stała wzdłuż każdego włókna. Okazuje się jednak
że w kohomologiach jednak $\pi^\ast \circ s^\ast$ jest identycznością.
Aby to pokazać, posłużymy się formułą homologii. Chcemy mianowicie
znaleźć funkcję $I$ na $\Omega(\R^n \times \R)$ taką, że spełnione jest
równanie:
\[
    id- \pi^\ast \circ s^\ast = \pm ( dI - Id)
\]
%% Podkreślmy ponownie, że $dI \pm Id$ przekształca formy domknięte
%% na formy dokładne i dlatego indukuje przekształcenie zerowe w kohomologii. 
%% Dla pełnej klarowności, dzieje się tak dlatego, że skoro $d \circ d = 0$
%% to z definicji forma zamknięta $\omega$ zachowuje wzór $d \omega = 0$ i
%% w konsekwencji $I d \omega = 0$. Dlatego też z powyższego wzoru pozostaje
%% tylko $d I \omega$, która jest formą dokładną. \\

Gdy taki operator $I$ istnieje, nazwany jest on \emph{operatorem homotopii
łańcuchowej}, a operator $\pi^\ast \circ s^\ast$ jest łańcuchowo homotopijne z
identycznością.  Zwróćmy także uwagę, że operator homotopii obniża gradację
formy o 1. \\

Każda forma na $\R^n \times \R$ da się wyrazić jako suma następujących
dwóch podstawowych typów form:

\begin{enumerate}
    \item $ f(x,t) (\pi^\ast \phi)$, 
    \item $f(x,t) (\pi^\ast \phi) \wedge dt $,
\end{enumerate}
gdzie $\phi$ jest formą określoną na przestrzeni podstawowej $\R^n$.
Zdefiniujemy teraz
$I: \Omega^{q} (\R^n \times \R) \rightarrow \Omega^{q-1} (\R^n \times \R)$ 
na poszególnych typach form jako:
\begin{enumerate}
    \item $f(x,t) (\pi^\ast \phi) \mapsto 0$,
    \item $f(x, t_0) (\pi^\ast \phi) \wedge dt \mapsto (\int_0^{t_0} f dt) (\pi^\ast \phi) $
\end{enumerate}
Przystąpimy teraz do sprawdzenia, że $I$ jest rzeczywiście operatorem homotopii.
Dla uproszczenia dalszych wzorów zastosujemy uproszczenie notacji. Będziemy
pisać $\partial f / \partial x$ zamiast $\sum \partial f / \partial x^i dx^i$
oraz $\int g$ zamiast $\int g(x, t) dt$. \\

Korzystając z tej notacji, sprawdzamy dla $q$-formy typu 1:

\[
    \omega = f(x,t)(\pi^\ast \phi)
\]
\[
    (1 - \pi^\ast s^\ast) \omega =
         f(x,t) \cdot (\pi^\ast \phi) -   f(x,0) \cdot (\pi^\ast \phi)
\]
\[
    (dI - Id) \omega = - I d\omega = 
    -I \left(
    f (d \pi^\ast \phi) +
    (-1)^q \left(
       \frac{\partial f}{\partial x} dx +
       \frac{\partial f}{\partial t} dt
       \right) \wedge
      (\pi^\ast \phi)
    \right) =
\]
\[
     (-1)^{q-1} \int_0^t \frac{\partial f}{\partial t} (\pi^\ast \phi)
    = (-1)^{q-1}  [f(x,t) - f(x,0)] (\pi^\ast \phi).
\]

Zestawiając powyższe równości możemy więc napisać
\[
    1 - \pi^\ast s^\ast = (-1)^{q-1}(d \circ I - I \circ d)
\]
dla form typu 1. \\

Zbadajmy formułę homotopii dla $q$-form typu 2:
\begin{align*}
     \omega &= f (\pi^\ast \phi) \wedge dt \\
    d\omega &= f(\pi^\ast d \phi) \wedge dt + 
      (-1)^{q-1} \frac{\partial f}{\partial x} 
       (\pi^\ast \phi) \wedge dx \wedge dt \\
\end{align*}
\[
    (1 - \pi^\ast s^\ast) = \omega ~\text{ponieważ}~
     s^\ast(dt) = d(s^\ast) = d(0) = 0
\]
\[
    Id\omega =  \left( \int_0^t f \right) (\pi^\ast d\phi) +
      (-1)^{q-1}
      (\int_0^t \frac{\partial f}{\partial x})
      (\pi^\ast \phi) \wedge dx 
\]
\[
    dI\omega = \left( \int_0^tf \right) (\pi^\ast d\phi) +
      (-1)^{q-1} (\pi^\ast \phi)  \wedge
      \left[
          (\int_0^t \frac{\partial f}{\partial x}) dx
          + f dt
      \right],
\]
i podsumowując zachodzi wzór
\[
    (d \circ I - I \circ d) \omega = (-1)^{q-1} \omega.
\]

W obu przypadkach mamy więc 
\[
1 - \pi^\ast \circ s^\ast = (-1)^{q-1}(d \circ I - I \circ d).
\] 

%% Tutaj zmiana
\begin{remark}
Redefiniując operator homotopii jako
\[
I'(\omega) = (-1)^{\text{deg}~\omega} I(\omega)
\]
możemy uzyskać formułę homotopii w bardziej popularnej i eleganckiej postaci,
która pojawia się w innych gałęziach matematyki:
\[
1 - \pi^\ast \circ s^\ast = d \circ I + I \circ d.
\]
\end{remark}

\begin{wniosek}\label{pi-is-isomorphism}
Przekształcenia $H^\ast (\R^n \times \R)
\stackrel[\pi^\ast]{s^\ast}{\rightleftarrows} H^\ast(\R^n)$ są izomorfizmami.
Równoważnie, zachodzi formuła:
\begin{equation}\label{homotopy-formula-pi}
    \omega - \pi^\ast( s^\ast \omega) = dI \omega - I d\omega
\end{equation}
\end{wniosek}


Możemy także dzięki tym obserwacjom policzyć kohomologie $\R^n$.
\begin{wniosek}(Lemat Poincaré)
\[
H^\ast(\R^n) = H^\ast(punkt) = 
\begin{cases}
\R & \text{dla wymiaru 0} \\
0 & \text{dla wszyskich innych przypadków} \\
\end{cases}
\]
\end{wniosek} 

\begin{theorem}
Grupy kohomologii $\M$ oraz $\M \times \R$ są izomorficzne:
\[
H^\ast( \M \times \R^1) \cong H^\ast( \M)
\]
\end{theorem}
\begin{proof}
Uogólnimy przedstawione wyżej rozumowanie na przypadek dowolnej
rozmaitości. Rozważmy mianowicie
\[
 \M \times \R^1 \stackrel[\pi]{s}{\leftrightarrows} \M
\]
Jeśli ${U_\alpha}$ jest atlasem dla $\M$, wtedy ${U_\alpha \times \R^1}$ jest
atlasem $\M \times \R^1$. Ponownie można zauważyć, że każda forma na $\M \times
\R$ jest kombinacją liniową dwóch przedstawionych powyżej typów form.  Możemy
więc zdefiniować operator homotopii $I$ w taki sam sposób jak wcześniej.  Wtedy
można będzie przepisać wcześniejszy dowód zamieniając $\R^n$ na $\M$ i będzie
on nadal poprawny.
 Stąd dostajemy tezę, gdzie izomorfizmy przeciwne to $\pi^\ast \circ
s^\ast$. Możemy też w końcu wywnioskować Propozycję~\ref{homotopy-de-Rham}, a co
za tym idzie formułę~\ref{homotopy-formula}.
\end{proof}

\begin{wniosek}
Homotopijne funkcje indukują tę samą mapę w kohomologii.
\end{wniosek}
\begin{proof}
Możemy biorąc przybliżenia założyć, że rozważana homotopia jest gładka.
Na potrzeby argumentu przypomnijmy definicję homotopii.  Niech $\N, \M$ będą
rozmaitościami.  Homotopią pomiędzy dwoma funkcjami $f,g: \M \rightarrow \N$
nazywamy funkcję $F: \M \times \R \rightarrow \N$ taką, że
\[
F(x,t) = 
\begin{cases}
f(x) & \text{dla}~t \geq 1 \\
g(x) & \text{dla}~t \leq 0 \\
\end{cases}
\]
\end{proof}
Równoważnie jeśli $s_0,s_1: \M \rightarrow  \M \times \R^1$ są $0$-włożeniem
$s_0(x) = (x, 0)$ i $1$-włożeniem $s_1(x) = (x,1)$, wtedy zachodzi
\begin{align*}
f &= F \circ s_1 \\
g &= F \circ s_0 \\
\end{align*}

a stąd także
\begin{align*}
f^\ast &= (F \circ s_1)^\ast = s_1^\ast \circ F^\ast \\
g^\ast &= (F \circ s_0)^\ast = s_0^\ast \circ F^\ast. \\
\end{align*}

Skoro więc zarówno $s_1^\ast$, jak i $s_0^\ast$ odwracają $\pi^\ast$, więc
są one równe w kohomologiach. Stąd zachodzi równoważny tezie wzór
\[
f^\ast = g^\ast.
\] 
Aby podkreślić dokładnie w jaki sposób implikuje to naszą tezę, przytoczę
definicję rozmaitości homotopijnie równoważnych. O rozmaitościach $\M, \N$
powiemy, że są homotopijnie równoważne, gdy istnieją funkcje
$f: \M \rightarrow \N$ oraz $g: \N \rightarrow \M$ takie, że
$f \circ g$ oraz $g \circ f$ są homotopijne do identyczności odpowiednio
na $M$ oraz $N$. Zachodzi zatem:
\[
f^\ast \circ g^\ast = id
\]
oraz 
\[
  g^\ast \circ f^\ast = id.
\] \\


% #################################
% #            CHAPTER            #
% #################################
\chapter{Kohomologie $f$-stożka}
W tym rozdziale przedstawione jest najważniejsze twierdzenie pracy, porównujące
$L_p$ kohomologie $f$-stożka z $L_p$ kohomologiami jego podstawy $M$.  Jako, że
stożek $\cfm$ jest dyfeomorficzny z $ M\times R$, chcemy porównać $H^*(M\times
\R)$ z $H^*(C_fM)$ uzywajac $\pi^*$ i korzystając z
formuly~\ref{homotopy-formula}.  Aby móc skorzystać z tego mechanizmu, musimy
kontrolować normę formy na którą podziałaliśmy operatorem homotopii tak, aby
formy występujące w formule homotopii wciąż należały do przestrzeni form
$p$-całkowalnych. W przeciwnym wypadku nie moglibyśmy bowiem powoływać się
na formułę homotopii. \\

W rozdziale tym zakładać będę, że formy na których pracuję są gładkie. Nie jest
to konieczne, jednak wymagałoby to korzystania z bardziej skomplikowanej
deficji różniczkowania - różniczki w sensie prądów. Z tego sposobu korzysta
między innymi praca~\cite{weber}.  Przedstawiane w niniejszej pracy obliczenie
jest wykonane sposobem prezentowanym między innymi w pracach \cite{cheeger},
\cite{youssin}, \cite{kirwan}, \cite{weber}. 

\begin{definition}[$f$-stożek]
    Niech $(\M, g)$ będzie rozmaitością Riemannowską. Rozważmy przestrzeń
    $\mathbb{R}_{\geq 0} \times \M$. Określmy na tym produkcie tensor
    Riemannowski zadany przez wzór $dt^2 + f^{2}(t)g $, gdzie $g$ jest
    metryką na $M$.  Przestrzeń taką nazywamy \textbf{$f$-stożkiem}.
    Oznaczać ją będziemy symbolem $\cfm$.
\end{definition}

\begin{theorem}\label{main-theorem}
  Niech $M,g$ będzie rozmaitością Riemannowską, a $\cfm$ określonym dla niej
  $f$-stożkiem dla $f = e^{-t}$. $L_p$-kohomologie tych dwóch przestrzeni są
  w następującej relacji
  \begin{equation}
    H_{(p)}^k \left( \cfm \right) = \begin{cases}
      H_{(p)}^k(M) & k < \frac{n}{p} + 1 \\
      0 & k >  \frac{n}{p} + 1 \\
      \end{cases}
  \end{equation}
\end{theorem}
Dowód tego twierdzenia oraz zmierzające do tego definicje i lematy są treścią
tego rozdziału.  Przypadek graniczny $k = \frac{n}{p} + 1$ jest w ogólności
nieregularny i nie jest badany w tej pracy. Zdarza się bowiem, że grupy
kohomologii tego rzędu są nieskończenie wymiarowe dla prostych przestrzeni,
czego ilustracją jest Przykład~\ref{infty-boundary-cohomologies}. \\

\section{Norma $L_p$ formy różniczkowej. $L_p$-kohomologie.}
%% 3. $L^p$-kohomologie (definicja za pomoca form gladkich), szacowanie norm
%% $\pi^*\omega, I_r\omega,$ (osobno dla $r=\infty$).


\begin{definition}[$L_p$-norma formy różniczkowej]
$p$-normą dla $k$-formy $\omega$ na rozmaitości Riemannowskiej $\M$ nazwiemy
\begin{equation} \label{big-norm}
|| \omega || = \left( \int_M |\omega|_x^p d \text{vol}_g(x) \right)^{\frac{1}{p}}.
\end{equation}
Formę należącą do tej przestrzeni będziemy nazywać formą $p$-całkowalną.
\end{definition}

Zajmiemy się pewną modyfikacją kohomologii de Rhama - $L_p$-kohomologiami.
Będziemy rozważać elementy tego samego kompleksu łańcuchowego co w
kohomologiach de Rhama, lecz z dodanym warunkiem $p$-całkowalności form.
Głównym obiektem naszego zainteresowania będą przestrzenie $L_p^k = \{\omega
\in \Omega^k(M): ||\omega|| <\infty, ||d \omega || < \infty \}$, gdzie $||
\cdot ||$ to norma (całka) z formy, która została przedstawiona powyżej.
Ograniczenie naszych rozważań do form, które są $p$-całkowalne pozwala
rozszerzać klasę przestrzeni, które badamy. Przy odpowiednim doborze funkcji
wagowej, którą ważymy metrykę Riemannowską na części nieosobliwej, możemy
bowiem rozważać rozmaitości z osobliwościami, na których $L_p$ kohomologie
zawierają dodatkowe informacji o osobliwości. \\

\begin{remark}
Ciąg przestrzeni $L^\ast_p$ z operatorem różniczki zewnętrznej stanowi
kompleks łańcuchowy.
\[
   ... \xrightarrow{d} 
L_p^i 
   \xrightarrow{d} 
L_p^{i+1}
   \xrightarrow{d} 
...~.
\]
\end{remark}
Prześledźmy przykład ilustrujący, że niekiedy grupy $L_p$ kohomologii mogą być
nieskończenie wymiarowe.
\begin{example}[Nieskończenie wymiarowe kohomologie w przypadku granicznym]\label{infty-boundary-cohomologies}
Dla  półprostej rzeczywistej $\R_{(+)}$ ze standardową metryką, otrzymujemy 
\[
H^0_{(2)}( \R_{+}) = 0,
\]
\[
\dim( H^1_{(2)}( \R_{+}) ) = \infty.
\]
\end{example}
\begin{proof}
Aby udowodnić pierwszą równość wystarczy zauważyć, że funkcja stała różna od
zera nie może być nigdy $L_2$ całkowalna.  Dla drugiej równości rozważmy
rodzinę form postaci $x^\alpha dx$, gdzie $\alpha \in (-\frac{3}{2},
-\frac{1}{2})$. Z jednej strony widzimy, że formy te są całkowalne. Z drugiej
strony zauważmy, że jeżeli forma takiej postaci miałaby być dokładna, $df =
x^\alpha dx $ to jej przeciwobraz w różniczce zewnętrznej miałby postać $f =
\frac{1}{1 + \alpha} x^{\alpha + 1}$ i nie byłby całkowalny. Formy tej postaci
nie są więc dokładne. Podobnie sprawdzamy, że klasy form $[x^\alpha dx]$ są
liniowo niezależne w kohomologiach.  Otrzymaliśmy więc nieskończoną rodzinę
liniowo niezależnych klas kohomologii form.
\end{proof}

\section{Uogólnione nierówności Hardy'ego}
Kluczowa w dalszych obliczeniach będzie nierówność Hardy'ego łącząca całkowalność
funkcji z całkowalnością jej funkcji pierwotnej.

\begin{theorem}[Uogólnione nierówności Hardy'ego]\label{hardy}
    Rozważmy funkcję $f: \mathbb{R}_{+} \rightarrow \mathbb{R}$, funkcje-wagi
    $\phi, \psi: \mathbb{R}_{+} \rightarrow \mathbb{R}$ oraz $p, q \in
    \mathbb{R}$ takie, że $\frac{1}{p} + \frac{1}{q} = 1 $ oraz $p > 1$.  Równości 
\begin{equation}\label{hardy-one}
\int_0^\infty \left|
                \phi(x) \int_0^x f(t) dt
              \right|^p dx
\leq
C \int_0^\infty \left|
                    \psi(x)  f(x)
                \right|^p dx
\end{equation}
\begin{equation}\label{hardy-two}
\int_0^\infty \left|
                \phi(x) \int_x^\infty f(t) dt
              \right|^p dx
\leq
C \int_0^\infty \left|
                    \psi(x)  f(x)
                \right|^p dx
\end{equation}
zachodzą wtedy i tylko wtedy, gdy odpowiednio dla~\ref{hardy-one} zachodzi
\[
\sup_{x > 0}
\left[
\int_x^\infty  
   | \phi(t) |^p dt
\right]^{\frac{1}{p}}
\left[
\int_0^x
    | \psi(t) |^{-q} dt
\right]^{\frac{1}{q}} < + \infty,
\]
a dla~\ref{hardy-two} zachodzi
\[
\sup_{x > 0}
\left[
\int_0^x
   | \phi(t) |^p dt
\right]^{\frac{1}{p}}
\left[
\int_x^\infty
    | \psi(t) |^{-q} dt
\right]^{\frac{1}{q}} < + \infty.
\]

\end{theorem}
\begin{proof}
Dowód tego twierdzenia pomijam. Jest on dostępny w pracy~\cite{hardys}.
%% %%% http://www.encyclopediaofmath.org/index.php/Hardy_inequality
%% where you can find references to the original source. \\
\end{proof}

\begin{theorem}\label{our-hardys-one}
Rozważmy pewną funkcję $f: \mathbb{R}_{+} \rightarrow \mathbb{R}$, gdzie $f
\geq 0$ oraz jej funkcję pierwotną $F(x) = \int_0^{x} f(t) dt$. Dla $\alpha > 0$ 
 $\int_0^\infty F(x)^pe^{- \alpha x}dx < \infty$
zachodzi wtedy i tylko wtedy, gdy
$\int_0^\infty f(x)^p e^{-\alpha x}dx < \infty$.
\end{theorem}
\begin{proof}
Wykorzystamy uogólnioną nierówność Hardy'ego.  Załóżmy $\psi(t) = \phi(t) =
e^{- \frac{t}{p} }$. Wtedy jeśli $\frac{1}{p} + \frac{1}{q} = 1 $, to
$\frac{1}{q} = \frac{p-1}{p}$ oraz $-q = \frac{p}{1-p}$.  Zbadajmy teraz, czy
spełniony jest warunek, aby zachodzić mogła nierówność Hardy'ego, część~\ref{hardy-one}:
$$
\sup_{x > 0}
\left[
\int_x^\infty  
    e^{-t} dt
\right]^{\frac{1}{p}}
\left[
\int_0^x
    e^{-t \frac{1}{1-p}} dt
\right]^{\frac{p-1}{p}}
=
\sup_{x > 0}
    C
    e^{- \frac{x}{p}}
    \left(
        e^{\frac{x}{p-1}} - 1
    \right)^{\frac{p-1}{p}}
=
$$
$$
C
\sup_{x > 0}
    \left(
    e^{- \frac{x}{p-1}}
        e^{\frac{x}{p-1}} -
    e^{- \frac{x}{p-1}}
    \right)^{\frac{p-1}{p}}
=
C
\sup_{x > 0}
    \left(
        1 -
    e^{- \frac{x}{p-1}}
    \right)^{\frac{p-1}{p}}. 
$$
Wyrażenie to jest ograniczone dla $p> 1$. Po zastosowaniu uogólnionej
nierówności Hardy'ego i ewentualnym przeskalowaniu zmiennej $x$ do $\alpha x$
otrzymujemy tezę.
\end{proof}

Zmodyfikujemy powyższe twierdzenie, tak aby rozważać funkcję pierwotną
$F(x) = \int_x^\infty f(t) dt $ oraz $\alpha < 0$.

\begin{theorem}
  Rozważmy pewną funkcję $f: \mathbb{R}_{+} \rightarrow \mathbb{R}$, gdzie
  $f \geq 0$
  oraz jej funkcję pierwotną $F(x) = \int_x^\infty f(t) dt$. Dla $\alpha < 0$,
 $\int_0^\infty F(x)^pe^{- \alpha x}dx < \infty$
zachodzi wtedy i tylko wtedy, gdy
$\int_0^\infty f(x)^p e^{-\alpha x}dx < \infty$.
\end{theorem}
\begin{proof}
  Dowód będzie bardzo podobny do poprzedniego. Dokonamy zmiany znaku,
  aby uwzględnić zmianę parametru $\alpha$ i 
 założymy $\psi(t) = \phi(t) =
 e^{ \frac{t}{p} }$.
 Wykorzystamy nierówność~\ref{hardy-two}. W tym celu musimy sprawdzić
 jej warunek konieczny i wystarczający:
 \begin{align*}
\sup_{x > 0}
\left[
\int_0^x e^t dt
\right]^{\frac{1}{p}}
\left[
\int_x^\infty
    e^{t \frac{1}{1-p}} dt
\right]^{\frac{p-1}{p}}
&=
\sup_{x > 0}
\left(
  e^x - 1
\right)^{\frac{1}{p}}
(p-1)^{\frac{p-1}{p}}
\left(
    e^{ - \frac{x}{p}} dt
\right) \\
& = 
\sup_{x > 0} C
\left(
  1 - e^{-x}
\right)^{\frac{1}{p}} < +\infty
 \end{align*}
 Stosujemy nierówność~\ref{hardy-two} i po przeskalowaniu $x$ przez
 dodatni czynnik $-\alpha$ otrzymujemy tezę.
\end{proof}

\section{Obliczenie $L_p$ kohomologii $f$-stożka}


Niech $L_p^k M$ oznacza przestrzeń gładkich $p$-całkowalnych 
$k$-form różnikowych z mierzalnymi  współczynnikami.
Możemy poczynić obserwację o postaci forma różniczkowych określonych na 
$f$-stożku. Przestrzeń styczna do $\cfm$ w punkcie $(t, m)$ to:
\[
    T_{(t, m)} (\mathrm{c}^f M) = \mathbb{R} \times T_m M.
\]
W terminach form różniczkowych powiązanych z rozważanym $f$-stożkiem możemy
napisać:
\[
\Lambda^k( T_{(t,m)}^\ast \cfm) = 
\Lambda^{k-1}(T_{m}^\ast\M)  \oplus \Lambda^k(T_{m}^\ast\M).
\]

\begin{wniosek}
  Każda $k$-forma $\omega \in \Lambda^k( T_{(t,m)}^\ast \cfm)$, 
a w konsekwencji każda forma z przestrzeni form $p$-całkowalnych  $L_p^k
(\cfm)$ może być zapisana jako $\omega = \eta + \xi \wedge dt$,
gdzie zarówno $\eta$, jak i  $\xi$ nie zawierają $dt$.  Zauważmy ponadto,
że $\eta$ jest $k$-formą, a $\xi$ jest $k-1$ formą. \\
\end{wniosek}

Ustalmy także dla klarowności nieco inną notację dotyczącą zapisywania
form różniczkowych względem lokalnych współrzędnych, która pozwoli nam 
lepiej zilustrować istotne dla nas elementy rozumowania. Dowolną $k$-formę $\eta$,
która w domyśle nie zawiera czynnika $dt$, zapisywać będziemy względem
lokalnych rzeczywistych współrzędnych
$(x_1, x_2, ... , x_n)$ na $\M$ jako:
\[
    \eta(t, x) = \sum_{\alpha \in I(k)} \eta_\alpha (t, x) dx^\alpha,
\]
gdzie $I(k)$ jest zbiorem wszystkich multiindeksów $\alpha = (\alpha_1, ...,
\alpha_k)$ takich, że $1 \leq \alpha_1 < ... < \alpha_i \leq n$, gdzie
\begin{equation}\label{notacja}
    dx^\alpha = dx^{\alpha_1} \wedge ... \wedge dx^{\alpha_k},
\end{equation}
a $\eta_\alpha$ jest gładką funkcją określoną na $(0, \infty) \times \M$. \\

Przypomnijmy w tym miejscu, że $\cfm$ oraz $\M \times R_+$ są
dyfeomorficzne.  Naszym planem będzie teraz zastosowanie
wniosku~\ref{pi-is-isomorphism} przy zachowaniu kontroli nad normą,
aby nie wyjść poza przestrzeń $L_p$. \\



Pomiędzy rozmaitościami $\M$ oraz $\cfm$ istnieją kanoniczne przekształcenia
projekcji oraz inkluzji. Żeby dobrze zilustrować w sposób w jaki działają one
na formy różniczkowe na poszczególnych rozmaitościach, przypomnijmy ich typy.
Inkluzja to funkcja:
\[
    s_r: \M \rightarrow \cfm \\
\]
\[
    s_r(x) = (x, r).
\]

Możemy za jej pomocą przeciągać formy z $\cfm$ do $\M$. Przeciągnięcie takie
oznaczymy jako $\omega_r = s_r^\ast(\omega) = s_r^\ast (\eta) $ dla formy $\omega
= \eta + \xi \wedge dt$. \\
Projekcja (rzutowanie) zadane jest jako:
\[
    \pi: \cfm \rightarrow \M
\]
\[
    \pi (x, t) = x.
\] \\


Rozważamy formę $\omega \in L^k_p (\cfm)$, gdzie
$\omega = \eta + \xi \wedge dt$.
Zauważmy, że metryka Riemannowska na $\cfm$ jest określona w taki sposób, że
stosowne normy spełniają następujące zależności:
\begin{equation}\label{cfm-norms-scaling}
| \omega(t,x) |^2 = (\mathrm{e}^{-t})^{-2k} | \eta(t,x) |^2_{\M} +
(\mathrm{e}^{-t})^{-2(k-1)} | \xi(t,x) |^2_{\M},
\end{equation}
gdzie $|\cdot |_{\M} $ jest normą form różniczkowych indukowaną przez
metrykę Riemannowską na rozmaitości $\M$.  Czynnik $(\mathrm{e}^{-t})^{-2k}$
pojawia się ponieważ forma $\eta$ należy do $k$-tej potęgi zewnętrznej
przestrzeni kostycznej do rozmaitości $\cfm$ w punkcie $(t,x)$. 
Wniosek ten jest zastosowaniem uwag
o postaci metryki Riemannowskiej na produkcie
rozmaitości przedstawionych w równaniu~\ref{metric-form}, lokalnej
postaci formy objętości~\ref{expression-for-volume-form} i uwag
o skalowaniu norm z Rozdziału~\ref{norm-scaling} \\

%% Niech $\omega$ będzie $k$-formą na $\M$, którą można zapisać w lokalnych
%% współrzędnych $(x^1, ..., x^n)$ jako
%% \[
%% \omega(x) = \sum_{\alpha \in I(k)} \omega_\alpha (t, x) dx^\alpha.
%% \]
%% Po przeciągnięciu forma $\pi^\ast \omega$ na $\cfm$ jest dana w lokalnych 
%% współrzędnych $(t, x^1, ..., x^n)$ dokładnie tym samym wzorem
%% \[
%% \pi^\ast \omega(t,x) = \sum_{\alpha \in I(k)} \omega_\alpha (t, x) dx^\alpha.
%% \]

Zbadamy teraz w jaki sposób zachowują się normy form przeciągniętych projekcją
$\pi$ na $\cfm$.  Zauważmy, że Riemannowska forma objętości na $\cfm$ w punkcie
$(t,x)$ różni się od formy objętości na $\M$ w punkcie $x$ o czynnik
$(e^{-t})^n$.  Z uwag o skalowaniu norm przytoczonych w
Rozdziale~\ref{norm-scaling} wynika też, że norma elementu należącego do
$k$-tej potęgi zewnętrznej pewnej przestrzeni liniowej $V$ skaluje się od
czynnik $\frac{1}{r^k}$ gdy skalujemy normę bazowej przestrzeni $V$ o czynnik
$r$.  Norma formy przy naszym skalowaniu skaluje się więc o $\frac{1}{e^{-tk}}
= e^{tk}$ Policzmy więc normę $\pi^\ast \omega$ jako elementu przestrzeni
$L_k^p (\cfm)$.

\[
    ||\pi^\ast \omega ||^p = \int_{\cfm} |\omega |^p d \mathrm{vol}_{\cfm} =
    \int_0^\infty \left( e^{-t} \right)^{n-pk} \int_M |\omega|^p d
    \mathrm{vol}_M dt = 
\]

\[
    = \
    \int_0^\infty \left( e^{-t} \right)^{n-pk} || \omega ||^p_{M} dt 
\]
%%% Przypomnijmy, że $\omega_r = i_r^\ast (\eta)$. W tym przypadku \\
%% Podobne oznaczenia i wnoski możemy 
%% została przeciągnięta z podstawy, czyli rozmaitości $\M$. Dla 
%% $\eta \in L_p^\ast (\M)$ możemy napisać
%% \[
%%     || \eta ||_r \deff || \pi^\ast ||_r = 
%% \mathrm{e}^{-r \cdot (\frac{n}{p} - k) }  ||\eta ||_{\M}.


\begin{wniosek}\label{pi-integrable}
  Całkowalność formy $\pi^\ast$ jest określona następującym warunkiem:
  \[
  ||\pi^\ast \omega ||  < \infty  \iff  k < \frac{n}{p}.
  \]
\end{wniosek}


Zgodnie z rozumowaniem identycznym do tego  z
Rozdziału~\ref{chapter-ordinary-cohomology} zachodzi formuła analogiczna
do~\ref{homotopy-formula-pi} (i istnieje stosowny operator $I_r$):
\[
    \omega - \pi^\ast(\omega_r) = dI_r \omega - I_r d\omega.
\]
Jedyne zmiany w dowodzie tej formuły to zmiany granic całkowania w dowodzie
formuły homotopii. \\

Zbadajmy kiedy dla $p$-całkowalnej formy $\omega = \eta + \xi \wedge dt$ forma
$I_r \omega$ także jest całkowalna:
\[
    ||I_r \omega ||^p = 
    \int_{\cfm} |I_r \omega|^p  d\text{vol}_{\cfm} =
    \int_0^\infty 
        (e^{-t})^{n-p(k-1)}
   \int_{\M} 
      \left| I_r \omega \right|^p d\text{vol}_{\M} dt =
\]
\[
    \int_0^\infty 
        (e^{-t})^{n-p(k-1)}
    || (I_r \omega)_t ||^p_{\M} dt 
   \leq
    \int_0^\infty 
        (e^{-t})^{n-p(k-1)}
    \left(\int_r^t || \xi_s ||_{\M} ds \right)^p~dt.
\] 
Aby oszacować tę wartość, wykorzystamy Twierdzenie~\ref{our-hardys-one},
przyjmując za funkcję z twierdzenie funkcję $f = \left( \int_{\M} |\xi_t|^p d
\text{vol}_{\M} \right)^{\frac{1}{p}} = ||\xi_t||_{\M}$.
Funkcja ta jest całkowalna w interesującym nas sensie, bowiem 
\[
||\xi||^p = 
    \int_0^\infty 
        (e^{-t})^{n-p(k-1)} ||\xi_t||^p dt < \infty
\]
Twierdzenie to możemy wykorzystać tylko wtedy, gdy wspołczynnik $\alpha$ w
wyrażeniu $(e^{-t})^\alpha$ jest większy od zera. Ten współczynnik to: $\alpha
= -p(k-1) + n$.  Zbadajmy warunek:
\[
    -p(k-1) + n > 0 \iff k < \frac{n}{p} + 1.
\]
Warunek pozwalający na wykorzystanie lematu jest więc spełniony dla $k <
\frac{n}{p} + 1$.  Gdy warunek ten jest spełniony, stosujemy wspomniane powyżej
twierdzenie wykorzystujące nierówność Hardy'ego i uzyskujemy rezultat, że $I_r
\omega$ jest formą $p$-całkowalną.  Pokazaliśmy więc, że jeśli $\omega$ jest
$k$-formą $p$-całkowalną, to  $I_r \omega$ jest $p$-całkowalna i dla $k <
\frac{n}{p} + 1$, jeśli $\omega$ jest gładka  zachodzi wzór
\[
    \omega = dI_r \omega - I_r d \omega + \pi^\ast 
    \left(
        \eta_{r}
    \right)
\]
Zwróćmy uwagę, że w tyn wzorze $\eta_r$ jest $k-1$-formą, więc
Wniosek~\ref{pi-integrable} stosujemy dla $k-1$, a nie $k$. Analogicznie do
Wniosku~\ref{pi-is-isomorphism} stwierdzamy więc, że skoro zachodzi 
formuła homotopii, to  $\pi^\ast$ zadaje izomorfizm między kohomologiami $\M$
a kohomologiami $\cfm$.
Udowodniliśmy więc pierwszą część Twierdzenia~\ref{main-theorem}. \\

Chcemy teraz pokazać, że dla $k > \frac{n}{p} + 1$ mamy $H^\ast_p(\cfm) = 0$.
Na potrzeby tego przypadku należy określić inny operator homotopii.  Dzieje się
tak ponieważ niektóre formy po zadziałaniu na nie dotychczas rozważanym
operatorem nie będą $p$-całkowalne. Na podstawie lematu~\ref{hardy-two} możemy
jednak zdefiniować operator całkowania od $t$ do $\infty$.
Bierzemy więc tak jak wcześniej $p$-całkowalną $k$-formę $\omega = \eta + dt \wedge \xi$,
 gdzie $k > \frac{n}{p} + 1$,  lecz definiujemy operator homotopii jako
\[
I_\infty \omega = \int_t^\infty \xi.
\]

oraz pomocnicze operatory
\[
I_r \omega = \int_t^r \xi,
\] gdzie $r \in (0, \infty)$.
Będziemy chcieli uzyskać zbieżność pomocniczych operatorów $I_r \to I_\infty$
Aby $I_\infty$ miał sens,  musimy jednak przekonać się, że całka $\int_a^\infty
\xi_t$ istnieje.  Niech $\xi$ będzie $p$-całkowalną $k$-formą na $\cfm$. Dla
określonego $a > 0$ możemy napisać, korzystając z nierówności Höldera
\begin{align*}
\left|
    \int_a^\infty \int_\M \xi_{t} dt
\right|
 & \leq
    \int_a^\infty \int_\M |\xi_{t}|_M dt =
    \int_a^\infty \int_\M (e^{-t})^{\frac{n}{p}-(k-1)} (e^{-t})^{k-1-\frac{n}{p}} |\xi_{t}|_M dt  \\
 & \leq 
\left(
    \int_0^\infty \int_\M (e^{-t})^{n-p(k-1)} |\xi_{t}|^p dt
\right)^\frac{1}{p}
\left(
    \int_a^\infty \int_\M (e^{-t})^{ q ( k-1-\frac{n}{p}) } dt 
\right)^\frac{1}{q} \\
& = 
    ||\xi||_{\cfm}
\left(
   \frac{ (e^{-a})^{ q ( k-1-\frac{n}{p}) }   }{ k-1-\frac{n}{p} } 
\right)^\frac{1}{q} < \infty,
\end{align*}
bo $k > \frac{n}{p} +1$ oraz $q = \frac{p}{p-1} > 0$.
Wnioskujemy więc, że skoro dla każdego $a > 0$  zachodzi
\[
\left|
    \int_a^\infty \int_\M \xi_{t} dt
\right| < \infty,
\]
to dla prawie wszystkich $x \in \M$ i dla wszystkich $t \in (0, \infty)$ całka
\[
\int_t^\infty \xi  = \int_t^\infty \xi_{\tau} d\tau
\]
istnieje. \\

Pomocnicze operatory wykorzystamy do modyfikacji lematu o formule homotopii, 
patrz Wniosek~\ref{homotopy-formula-pi}.  W szczególności w lemacie tym
badaliśmy wyrażenia typu $\int_0^t \frac{\partial f }{\partial t} = f(x, t) -
f(x, 0)$.  W przypadku nowego operatora $I_\infty$, wyrażenia te przyjmują
postać $\int_t^\infty \frac{\partial f }{\partial t}$ i niekoniecznie muszą
zezwalać na zcałkowanie różniczki, czyli na zastosowanie wariantu podstawowego
twierdzenia rachunku całkowego, bowiem granica funkcji $\lim_{x \to
\infty} f(x)$ niekoniecznie musi być skończona. Jednak dla dowolnego $r$
pomocniczy operator $I_r$ spełnia równanie homotopii, co widać wprost w
rozumowaniu porzedzającym Wniosek~\ref{homotopy-formula-pi}.  Jedyny sposób w
jaki zmienia się dowód tej formuły dla $I_r$, to wspomiane granice całkowania w
dowodzie: z $\int_0^t$ na $\int_t^r$.  Zbiegając $r \to \infty$,  w granicy
chcielibyśmy otrzymać formułę homotopii postaci:

\begin{equation}\label{homotopy-inf}
\omega = dI_\infty \omega - I_\infty d \omega,
\end{equation}
ale nie możemy tego zrobić bez dodatkowych założeń, ponieważ forma $I_\infty
\omega$ może nie być gładka.  

\begin{remark}
W formule homotopii wykorzystującej nowy operator $I_r$
\begin{equation}\label{hom-form-infty}
    \omega - \pi^\ast(\omega_r) = dI_r \omega - I_r d\omega
\end{equation}
gdy $r \to \infty$ zachodzi 
\[
 \omega_r  \to 0, 
\]
Jest tak ponieważ dla każdego $r$ formuła homotopii jest prawdziwa, a
jednocześnie prawdziwy jest wzór
\[
|| \omega ||^p = \int_0^\infty e^{\alpha t} || \omega_t ||^p_M dt
\]
 dla $\alpha \geq 0$. Jako, że $e^{\alpha t} \to \infty$ to $|| \omega_t ||^p$
musi dążyć do 0.  Formą o normie zero jest jedynie forma zerowa.
\end{remark}


\begin{wniosek}\label{asymtpotic-s-null-operator}
Dla przypadku $k > \frac{n}{p} + 1$ w formule homotopii~\ref{hom-form-infty}
składnik $\pi^\ast(\omega_r)$ dąży do zera dla $r \to \infty$.
\end{wniosek}

Dla formy dokładnej $\omega$ mamy więc formułę
\[
    \omega - \pi^\ast(\omega_r) = dI_r \omega.
\]
Gdy założymy dodatkowo, że $I_\infty\omega$ jest formą gładką, oraz że zachodzi
zbieżność
\[
d\left( I_r \omega \right) \to d (I_\infty \omega),
\]
to $\omega$ jest różniczką formy $I_\infty \omega$, i zachodzi $[\omega] = 0$, co jest
treścią drugiej częsci Twierdzenia~\ref{main-theorem}. \\

Jeżeli będziemy starali pozbyć się powyższych założeń, to sytuacja jest
bardziej skomplikowana. Twierdzenie przedstawione w pracy jest prawdziwe,
jednak jego dowód  wymaga wykorzystania różniczek w sensie ``prądów,'' ang.
\emph{currents} tak, jak przedstawiono w pracach ~\cite{cheeger},
~\cite{weber}. Jest to jednak trudniejsze zadanie, które nie jest omawiane w
tej pracy. \\



%%  Fakt, że forma jest  $p$-całkowalna oznacza, że
%% \[
%%     || \omega ||^p = \int_{\cfm} |\omega|^p d \text{vol}_{\cfm} \leq
%%     \int_0^\infty \int_{\M} \left(
%%         (e^{-t})^{-pk} | \eta_{t}|^p_{\M} + 
%%         (e^{-t})^{-p(k-1)} | \xi_{t}|^p_{\M} 
%%     \right)
%%     (e^{-t})^{n} dt < \infty.
%% \]
%% Przypomnijmy, że czynnik $ (e^{-t})^{n} $ pochodzi od skalowania formy
%% objętości, a czynniki $p \cdot k$ pochodzą od tego, że forma jest w $k$-tej
%% potędze zewnętrznej, a norma podniesiona jest do $p$-tej potęgi.  Nierówność
%% jest uzasadniona z uwagi na równianie~\ref{cfm-norms-scaling} oraz nierówność
%% Jensena dla funkcji $x \mapsto x^\frac{p}{2}$.  \\
%%  Fakt, że forma jest  $p$-całkowalna oznacza, że
%% \[
%%     || \omega ||^p = \int_{\cfm} |\omega|^p d \text{vol}_{\cfm} \leq
%%     \int_0^\infty \int_{\M} \left(
%%         (e^{-t})^{-pk} | \eta_{t}|^p_{\M} + 
%%         (e^{-t})^{-p(k-1)} | \xi_{t}|^p_{\M} 
%%     \right)
%%     (e^{-t})^{n} dt < \infty.
%% \]
%% Przypomnijmy, że czynnik $ (e^{-t})^{n} $ pochodzi od skalowania formy
%% objętości, a czynniki $p \cdot k$ pochodzą od tego, że forma jest w $k$-tej
%% potędze zewnętrznej, a norma podniesiona jest do $p$-tej potęgi.  Nierówność
%% jest uzasadniona z uwagi na równianie~\ref{cfm-norms-scaling} oraz nierówność
%% Jensena dla funkcji $x \mapsto x^\frac{p}{2}$.  \\

\begin{thebibliography}{99}
\addcontentsline{toc}{chapter}{Bibliography}

\bibitem[BoTu]{bott} Bott R., Tu, L.W.: \textit{Differetial forms in algebraic
  topology}, Springer Verlag, 1982.

\bibitem[Dai]{dai} Dai, X.: \textit{An introduction to $L^2$ cohomology},
Topology of Stratified Spaces, MSRI Publications, vol. 58, 2010.

\bibitem[Ch]{cheeger} Cheeger, J.: \textit{On the Hodge theory
  of Riemannian pseudomanifolds}, Proc. of Symp. in Pure Math. vol.36.

\bibitem[KirWo]{kirwan}  Kirwan, F., Woolf, J.: \textit{An Introduction
to Intersection Homology Theory}, Taylor \& Francis, LLC.

\bibitem[Ko]{kostrikin} Kostrikin, A.I.: \textit{Wstęp do algebry.
Tom II: Algebra liniowa}, Wydawnictwo naukowe PWN, 2012.

\bibitem[Lee]{lee} Lee, J.M., \textit{Introduction to Smooth Manifolds},
Springer, 2013.

\bibitem[Mu]{hardys} Muckenhoupt, B., \textit{Hardy's inequality
with weights}, Studia Mathematica, T. XLIV, 1972.

\bibitem[We]{weber}  Weber, A.: \textit{An isomorphism from
  intersection homology to $\mathrm{L}_p$-cohomology}, Forum
  Mathematicum, de Gruyer, 1995.

\bibitem[You]{youssin} Youssin, B., \textit{$\mathrm{L}_p$
  cohomology of cones and horns } J. Differential Geometry, Volume 39,
  Number 3, 1994.

\end{thebibliography}

\end{document}
