\documentclass[licencjacka]{pracamgr}
\usepackage[utf8]{inputenc}
\usepackage[T1]{fontenc} 
\usepackage{amssymb}
\usepackage{amsmath}
\usepackage{amsthm}
\usepackage{polski}

\theoremstyle{definition}
\newtheorem{definition}{Definicja}[section]

\theoremstyle{definition}
\newtheorem{remark}{Uwaga}[section]

\theoremstyle{plain}
\newtheorem{lemma}{Lemat}[section]

\theoremstyle{plain}
\newtheorem{theorem}{Theorem}[section]

%% Document global definitions
\def\cfm{\ensuremath\mathrm{c}^f\mathcal{M}}
\def\M{\ensuremath\mathcal{M}}
\newcommand\deff{\mathrel{\overset{\makebox[0pt]{\mbox{\normalfont\tiny\sffamily def}}}{=}}}


\author{Michał Garmulewicz}

\nralbumu{304742}


\title{$\mathrm{L}_p$-kohomologie $f$-stożków Riemannowskich.}

\tytulang{$\mathrm{L}_p$-cohomologies of Riemannian $f$-horns.}

\kierunek{Matematyka} % Praca wykonana pod kierunkiem:
% (podać tytuł/stopień imię i nazwisko opiekuna
% Instytut
% ew. Wydział ew. Uczelnia (jeżeli nie MIM UW))
\opiekun{dra hab. Andrzeja Webera\\
              Instytut Matematyki\\}

% miesiąc i~rok:
\date{Maj 2015}

%Podać dziedzinę wg klasyfikacji Socrates-Erasmus:
\dziedzina{ 
11.0 Matematyka, Informatyka:\\ 
11.1 Matematyka\\ 
}

%Klasyfikacja tematyczna wedlug AMS (matematyka) lub ACM (informatyka)
\klasyfikacja{14 Algebraic Geometry\\
  14F (Co)homology theory\\
  14F40 de Rham cohomology}

% Słowa kluczowe:
\keywords{
  kohomologie de Rhama, topologia różniczkowa
}

% Tu jest dobre miejsce na Twoje własne makra i~środowiska:
\newtheorem{defi}{Definicja}[section]

% koniec definicji

\begin{document}
\maketitle

%tu idzie streszczenie na strone poczatkowa
\begin{abstract}
  W tej pracy licencjackiej opisane jest obliczenie $L_p$-kohomologii
  rozmaitości Riemannowskich.
\end{abstract}

\tableofcontents
%\listoffigures
%\listoftables

\chapter{Wstęp}

W pracy \cite{weber} prezentowane jest między innymi obliczenie
$L_p$-kohomologii stożka nad pseudorozmaitością Riemannowską. 

Na stożku tym określona jest metryka postaci
$dt \otimes dt \oplus t^2 g$, gdzie $g$ 
jest metryką na części nieosobliwej wyjściowej pseudorozmaitości.
W dalszej części wspomnianej pracy obliczana jest $L_p$-kohomologia
wspomnianej przestrzeni.
\\

%% This line of research can 
%% be traced back to Cheeger \cite{cheeger}. 
%% A similar approach was presented by Youssin \cite{youssin}, where $f$-horns
%% were considered.

W tej pracy licencjackiej przedstawiam pewną modyfikację tych pojęć
do $e^{-t}$-stożków Riemannowskich, czyli przestrzeni będących produktem rozmaitości
Riemannowskiej oraz półprostej na której określono metrykę
$dt \otimes dt \oplus (e^{-t})^2 g$, gdzie $g$ 


\chapter{Podstawy}
W rozdziale tym przytaczam definicje i wyprowadzam prostsze zależności, które
pomogą nam w dalszych obliczeniach. \\


\section{Formy różniczkowe}
(Bardzo krótkie zagajaenie o formach - definicja i podstawowe operacje). \\

\section{Skalowanie norm przestrzeni wektorowych i tensorów}
W dalszych obliczeniach będziemy badać wyrażenia typu $r |\cdot|$, gdzie
$|\cdot|$ jest normą na skończenie wymiarowej przestrzeni liniowej.
Znaczenie będzie miało jak zachowuje się norma na przestrzenii dualnej,
gdy skalujemy normę wyjściowej przestrzenii liniowo o czynnik $r$. \\


Rozważać będziemy skończenie wymiarową przestrzeń wektorową $V$ z daną
metryką $|| \cdot ||$ i zdefiniujemy na niej nową metrykę $|| \cdot
||_r \deff r \dot || x ||$. Wtedy w przestrzeni $(V, ||
\cdot||_r)^\ast$ dualnej do $(V, ||| \cdot |||)$ norma skaluje się
przez współcznynnik $\frac{1}{r}$. Oznacza to, że dla $\varphi \in
V^\ast$ otrzymamy skalowanie $||\varphi||_r = \frac{1}{r} ||\varphi||$. \\

Poczynimy tu jeszcze jedną obserwację, która pomoże nam w dalszych
obliczeniach.  Rozważmy $n$-wymiarową rozmaitość Riemannowską
$\mathrm{M}$ oraz jej przestrzeń styczną $T_x\mathrm{M}$ i kostyczną
$T_x^\ast\mathrm{M}$ w punkcie $x$.  Niech bazami ortogonalnymi tych
przestrzeni będą $e_1, e_2, ..., e_n$ oraz dualna $e_1^\ast, e_2^\ast,
..., e_n^\ast$.  Chcemy teraz obliczyć w jaki sposób formy z
$\Lambda(\mathcal{M})$ skalują się ze względu na taką zmainę w
normie. \\

Załózmy że rozważamy przestrzeń $k$-form na $\M$ w pewnym
punkcie rozmaitości. Wtedy każda $k$ forma może być lokalnie wyrażona
w bazie składającej się z produktów kowektorów należących do bazy
dualnej do bazy standardowej. Oznacza to, że każda $k$-forma w punkcie
$x$ może być zapisana, korzystając z lokalnych współrzędnych $(x_1,
x_2, ... x_n)$ jako $ \sum_{\mathrm{I} \in I } a_\mathrm{I}
dx_\mathrm{i}$, gdzie $I$ jest zbiorem $k$-indeksów postaci
$(i_1, i_2, ..., i_k)$, dla $ 1 \leq i_1 < i_2 < ... < i_k \leq n$.  \\

Popatrzmy w jaki sposób skalują się wektory bazowe. Przy założeniu ortogonalności
zachodzi:
\[
    ||dx_{i_1} \wedge dx_{i_2} \wedge ... \wedge dx_{i_k} ||_t =  
    ||dx_{i_1} ||_t \cdot ||  dx_{i_2} ||_t \cdot ... \cdot || dx_{i_k} ||_t =  \\
\]
\[
    \frac{1}{t}||dx_{i_1} || \cdot \frac{1}{t} ||  dx_{i_2} ||_t \cdot ...
     \cdot \frac{1}{t} || dx_{i_k} ||_t = 
    \frac{1}{t^k}||dx_{i_1} \wedge dx_{i_2} \wedge ... \wedge dx_{i_k} ||
\]

Oznacza to, że każda $k$ forma skaluje się przez $1/t^k$ gdy metryka
jest skalowana przez czynnik $t$.  Obserwację tą możemy także
zaplikować do formy objętości, opisywanej w dalszych częściach  otrzymując:
\[
d\mathrm{vol}_t = \frac{1}{t^n} d\mathrm{vol}
\]

\section{Metryka Riemannowska i forma objętości}

\textbf{Metryką Riemannowską} nazwiemy gładkie, symetryczne kowariantne
pole 2-tensorów na rozmaitości
$\mathcal{M}$ które jest dodatnio określone w każdym punkcie. Mówiąc bardziej
intuicyjnie, określenie metryki Riemannowskiej to doczepienie 
pola iloczynów skalarnych do rozmaitości, które zmienia się w sposób gładki.
\\

W dowolnych lolalnych gładkich współrzędnych $(x^i)$ metryka Riemannowska
może być zapisana jako
\[
    g = g_{ij} dx^i \otimes dx^j = g_{ij} dx^i dx^j,
\]
gdzie
$g_{ij}$
jest dodatnio określoną macierzą (której współrzędne to funkcje gładkie). \\

Najprostszym przykładem metryki Riemannowskiej jest \emph{metryka Euklidesowa}
na $\mathbb{R}^n$, która zadana jest w standardowych współrzędnych jako
\[
    g = \delta_{ij}dx^idx^j.
\] \\


(Krótkie podsumowanie informacji o 
całkowaniu na rozmaitości Riemannowskiej .., forma objętości) \\


%% Citing prof. Lee, it is common to abbreviate the symmetric product of a tensor
%% $\alpha$ with itself by $\alpha^2$, so the Euclidean metric can also be written
%% as 
%% \[
%%     g = (dx^1)^2 + ...  + (dx^n)^2,
%% \]
%% so now it is way easier to understand what exactly is meant by
%% $ dt \otimes dt + f^2g $, which shoudl be the same as $dt^2 + f^2g$. \\
%% wiki volume form hehehehe

\section{Norma formy różniczkowej}

Przytoczę teraz definicję normy formy różniczkowej w poszczególnych punktach
rozmaitości. Niech $(M, g)$ będzie zorientowaną, spójną %% i zwartą ?? chyba nie
rozmaitością Riemannowską. Niech $\M$ będzie rozmaitością, a $x$ jej punktem.
$\Lambda^k T_x M$ jest przestrzenią $k$-liniowych funkcji alternujących:
\[
    \alpha_x:T_x^\star M \times ... \times T_x^\star M \times \rightarrow \mathbb{R}
\] \\

Przypomnijmy także, że forma zewnętrzna stopnia $k$ jest zdefiniowana jako 
sekcja $k$-tej wiązki kostycznej.
\[
    \Lambda^k M = \coprod_{x \in M} 
    \Lambda^k T_x^\star M \xrightarrow{\pi} M.
\] \\

Stąd w każdym punkcie $x \in M$ mamy funkcję wieloliniową $\alpha_x \in
\Lambda^k T_x^\star M$. Można ją wyrazić w prostej formie.
Jeśli $(e_1, ..., e_n)$ jest bazą $T_x M$, a  $(e^1, ..., e^n)$ jest bazą dualną,
możem napisać
\[
    \alpha_x = \sum_{1 \leq i_1 < ... < i_k \leq n} a_{i_1 ... i_k} e^{i^1} \wedge ...
    \wedge e^{i^k}
\] 
gdzie współczynniki są zadane jako
$a_{i_1 ... i_k} = \alpha_x(e^{i^1}, ..., e^{i^2})$. \\

Korzystając z powyższych spostrzeżeń, możemy wyrazić daną formę w terminach
współrzędnych lokalnych $x^1, ..., x^n$
na otwartym podzbiorze $U \subset M$.  Mamy bazę
$\frac{\partial}{\partial x^1}, ..., \frac{\partial}{\partial x^n}$ dla $T_x M$
dla każdego $x \in U$ wraz ze stowarzyszoną bazą dualną $dx^1, ..., dx^n$.
Ponadto, dla każdego punktu ze zbioru $U$ możemy napisać
\[
    \alpha_x = \sum_{1 \leq i_1 < ... < i_k \leq n} a_{i_1 ... i_k}
       dx^{i^1} \wedge ...  \wedge dx^{i^k}.
\]  \\

Na rozmaitości Riemannowskiej posiadamy iloczyn sklarany zadany na 
$T_x M$ dla każdego punktu $x \in M$.
Korzystając z niego możemy zdefiniować normę każdej $k$-formy $\alpha$.
Wprowadźmy proste oznaczenie iloczynu skalarnego na
$T_x M$ przez $<u, v>_x = g_x(u, v)$. 
Wybierzmy bazę $(e_1, ..., e_n)$ przestrzeni $T_x M$ z odpowiadającą jej bazą dualną
$(e^1, ..., e^n)$.
Można argumentować, że możemy zawsze wybrać taką bazę tak aby była ona ortogonalna.
Dzieje się tak, ponieważ możemy zawsze zastosować algorytm ortogonalizacji 
Grama-Schmidta do bazy zadanej przez różniczkowanie lokalnych współrzędnych rozmaitości. \\

Możemy teraz zdefiniować funkcję $G: \Lambda^k T_x M \times T_x M \rightarrow \mathbb{R}$.
Dla danych dwóch form
$ \alpha_x = \sum_{1 \leq i_1 < ... < i_k \leq n} \alpha_{i_1 ... i_k; x} e^{i^1}
\wedge ...  \wedge e^{i^k}$ oraz
$ \beta_x = \sum_{1 \leq i_1 < ... < i_k \leq n} \beta_{i_1 ... i_k; x} e^{i^1}
\wedge ...  \wedge e^{i^k}$,  określimy tę funkcję jako
\[
    G(\alpha_x, \beta_x) = \sum_{i_1, ..., i_k} \alpha_{i_1 ... i_k; x}
                                                    \beta_{i_1 ... i_k; x}.
\] \\

Przytoczymy teraz następujący lemat, dostępny na przykład w \cite{lausanne}.


\begin{lemma}
    Zdefiniowane powyżej przekształcenie $G: \Lambda^k T_x M \times T_x M $ 
    jest liniowe, symetryczne i dodatnio określone. Przez to jest ono iloczynem
    skalarnym dla każdego punktu $x \in M$.  Nie zależy ono od wyboru konkretnej bazy
    $(e_1, ..., e_n)$ spośród baz ortogonalnych.  Dodatkowo, baza
    $(e^1 \wedge ... \wedge e^n)$ jest ortonormalna dla $G$.
\end{lemma}

%% We can also give eqivalent definition using other terms ... ?? Może dopisać tutaj jeszcze z tego doktoratu
%% o i tamto \\


\section{Przekształcenie indukowane, kohomologie de Rhama oraz formuła homotopii}
Rozważmy przypadek gładkiej funkcji $F: M \rightarrow N$, pomiędzy dwoma rozmaitościami
(z brzegiem lub bez brzegu). Za pomocą tej funkcji będziemy 
Z różniczką takiej funkcji możemy stowarzyszyć przekształcenie
przeciągnięcia $F^\ast: \Lambda^p N \rightarrow \Lambda^p M$, działające w taki sposób:
\[
    (F^\ast \omega)_p(v_1, ..., v_n) =
        \omega_{F(p)}(dF_p(v_1), ..., dF_p(v_k)).
\] \\


W poniższym rozumowaniu, przez $\Omega^p \M$ oznaczać będziemy
przestrzeń gładkich $k$-form.  Niech $\M$ będzie rozmaitością z
brzegiem lub bez brzegu, a $p$ będzie nieujemną liczbą całkowitą.
Ponieważ $d: \Omega^p (\M ) \rightarrow \Omega_{p+1}(\M) $ jest
przekształceniem liniowym, jego jądro oraz obraz są podprzestrzeniami
liniowymi. Wprowadźmy tymaczasowe oznaczenie:
\[
\mathcal{Z}^p ( \M ) =
d: \Omega^{p} (\M ) \rightarrow \Omega_{p+1}(\M) =
\{\text{$p$-formy zamknięte na $\M$ } \}
\]
\[
\mathcal{B}^p ( \M ) =
d: \Omega^{p} (\M ) \rightarrow \Omega_{p+1}(\M) =
\{\text{$p$-formy dokładne na $\M$ } \}.
\]

Jako konwencję przyjmuje się, że $\Omega^{p} ( \M ) $ jest zerową
przestrzenią wektorową gdy $p < 0$ lub $p > n = \text{dim} \M $. W
związku z tym zachodzi przykładowo $\mathcal{B}^0(\M)=0$ oraz
$\mathcal{Z}^n(\M)= \Omega^n(\M)$. \\

Fakt, że każda forma dokładna jest zamknięta implikuje, że
$ \mathcal{B}^p ( \M) \subseteq \mathcal{Z}^p ( \M) $.
Stąd ma sens następująca definicja:

\begin{definition}
  \textbf{Grupą kohomologii de Rhama rzędu $p$} nazwiemy następującą
  ilorazową przestrzeń liniową:
  \[
  H^{p}_{dR} ( \M ) = \frac {\mathcal{Z}^p ( \M )} {\mathcal{B}^p ( \M )}
  \]
\end{definition}
Jest to rzeczywista przestrzeń liniowa, i w związku z tym jest ona
grupą z działaniem dodawania wektorów. Można także pokazać, że grupy de Rhama są
niezmiennicze ze względu na dyfeomorifzmy. Dla każdej domkniętej $p$-formy
$\omega$ na $\M$ poprzez $[\omega]$ będziemy oznaczać klasę równoważności
formy $\omega$ w $H_{dR} (M)$. Taką klasę równoważności będziemy nazywać także klasą
kohomologii formy $\omega$. Jeżeli dwie formy $\omega, \eta$ należą do tej samej klasy
kohomologii, czyli zachodzi $[\omega] = [\eta]$, to różnią się one conajwyżej o formę
dokładną. Zachodzi także następujący lemat:

\begin{lemma}(Przekształcenia indukowane kohomologii)
  Dla każdej gładkiej funkcji $F: \M \rightarrow \mathcal{N}$ pomiędzy
  dwoma rozmaitościami gładkimi, przeciągnięcie 
  $F^\ast: \Omega^p(N) \rightarrow \Omega^p (M)$ przenosi formy dokładne
  na formy dokładne, a formy zamknięte na formy zamknięte. W ten sposób indukuje
  ono przekształcenie liniowe, w dalszym ciągu oznaczane jako $F^\ast$ z
  $H^p_{dR} (N)$ do $H^p_{dR} (M)$, które nazywane jest przekształceniem indukowanym
  kohomologii. \\
\end{lemma}
\begin{proof}
  Jeśli $\omega$ jest formą zamkniętą, to $d(F^\ast \omega) = F^\ast(d \omega) = 0$,
  więc $F^\ast \omega$ także jest zamknięte. Stąd wynika już, że przeciągnięcie
  to przenosi formy zamknięte na zamknięte, a dokłądne na dokładne. Przekształcenie
  indukowane jest zadane w prosty sposób. Dla $p$-formy zamkniętej $\omega$, niech
  \[
  F^\ast[\omega] = [F^\ast\omega].
  \]
  Wtedy jeśli $\omega' = \omega + d \eta$, to 
  \[
  F^\ast[\omega'] = [F^\ast\omega d(F^\ast\eta)] = [F^\ast\omega],
  \]
  a więc przekształcenie jest dobrze zdefiniowane.
\end{proof} 
Odnotujmy następujący, ważny wniosek.

\begin{remark}
  Rozmaitości gładkie, które są ze sobą dyfeomorficzne, mają izomorficzne grupy
  kohomologii de Rhama. \\
\end{remark}


%% Mike note - this star means induced map, which is explained in the
%% previous chapter

Przedstawiony powyżej wniosek jest nieco zaskakujący - grupy de Rhama okazały
się być topologicznym niezmiennikiem. Wniosek ten ma daleko idące uogólnienie.
Można bowiem udowodnić, że wspomniane grupy są niezmiennikami homotopii. Oznacza
to, że homotopijnie równoważne rozmaitości posiadać będą izomorficzne kohomologie
de Rhama. \\

W poniższym rozumowaniu ciekawy jest dla nas zarówno wynik, jak i technika, która
wykorzystywana będzie do jego udowodnienia. Wyprowadzimy bowiem równanie, które
zada warunek na istnienie pewnego operatora. Nasza teza będzie wtedy równoznaczna
z istnieniem interesującego nas operatora. \\

Chcemy najpierw udowodnić, że homotopijne funkcje gładkie indukują to samo 
przekształcenie kohomologii.
Rozważamy w tym celu dwie gładkie funkcje $F, G: M \rightarrow N$.
Chcemy udowodnić, że ich przekształcenia indukowane są równe
$F^\ast = G^\ast$. Wyraźmy ten warunek w nieco innych słowach. \\

Gdy weźmiemy zamkniętą formę $p-$formę $\omega$ na $N$, aby
przekształcenia indukowane były równe, musimy być w stanie
wyprodukować taką $(p-1)$-formę $\eta$ na $M$, aby spełnione
\[
    G^\ast \omega - F^\ast \omega = d\eta.
\]

Z tego wyniknie bowiem, że
$ G^\ast [\omega] - F^\ast [\omega] =
[d\eta] = 0$.
Możemy podejść do tego problemu nieco bardziej systematycznie, 
szukając takie operatora
$h$, który jako argumenty bierze zamknięte $p$ formy na $N$
i działa tak, że spełniona jest zależność
\[
    d(h\omega) = G^\ast \omega - F^\ast \omega.
\] \\

Zamiast definiować $h \omega$ tylko dla przypadku, kiedy $\omega$
jest zamknięta, okazuje się, że łatwiej jest określić operator
$h$ z przestrzeni wszystkich gładkich $p$-form na $N$
do przestrzeni wszystkich gładkich $(p-1)$-form na $M$,
dla którego spełnione jest równanie
\[
    d(h\omega) + h(d\omega) = G^\ast \omega - F^\ast \omega.
\]
Gdy warunek ten jest bowiem spełniony to dla formy $\omega$, która
jest zakmnięta zajdzie także poprzedni warunek. \\

Jeśli chcemy być całkowicie dokładni, to musimy
zdefiniować rodzinę funkcji, po jednej funkcji dla każdego $p$, która
będzie spełniać stosowny warunek na danym poziomie.
\[
    H(\mathcal{M} \times \mathbb{R}_{\geq})_{dR}^\ast = H(\mathcal{M})_{dR}^\ast
\] \\


(Przytoczyć dowód istnienia operatora homotopii - najlepiej z książki Bott)

\section{$L_p$-kohomologie}
(Opisać krótko, że $L_p$ kohomologie to jest to samo tylko z dodanym warunkiem, 
żeby normy były $p$-całkowalne).


\section{Uogólniona nierówność Hardy'ego}
Kluczowa w dalszych obliczeniach będzie nierówność, Hardy'ego łącząca całkowalność
funkcji z całkowalnością jej funkcji bazowych.

\begin{lemma}[Uogólniona nierówność Hardy'ego]
    Rozważmy funkcję $f: \mathbb{R}_{+} \rightarrow \mathbb{R}$, funkcje-wagi
    $\phi, \psi: \mathbb{R}_{+} \rightarrow \mathbb{R}$ oraz $p, q \in
    \mathbb{R}$ takie, że $\frac{1}{p} + \frac{1}{q} = 1 $.  Zachodzi dla nich
$$
\int_0^\infty \left|
                \phi(x) \int_0^x f(t) dt
              \right|^p dx
\leq
C \int_0^\infty \left|
                    \psi(x)  f(x)
                \right|^p dx
$$
wtedy i tylko wtedy, gdy
$$
\sup_{x > 0}
\left[
\int_x^\infty  
   | \phi(t) |^p dt
\right]^{\frac{1}{p}}
\left[
\int_0^x
    | \phi(t) |^{-q} dt
\right]^{\frac{1}{q}} < + \infty,
$$
\end{lemma}
\begin{proof}
    Dowód pomijam. 
    Jest on dostępny w ... . Zacytować papier od pana Webera.
%% %%% http://www.encyclopediaofmath.org/index.php/Hardy_inequality
%% where you can find references to the original source. \\
\end{proof}

\begin{lemma}
Rozważmy pewną funkcję $f: \mathbb{R}_{+} \rightarrow \mathbb{R}$, gdzie $f
\geq 0$ oraz jej funkcję pierwotną $F(x) = \int_0^{x} f(t) dt$. Dla $\alpha > 0$ warunek
$\int_0^\infty f(x)^p e^{-\alpha x}dx < \infty$ implikuje $\int_0^\infty
F(x)^pe^{- \alpha x}dx < \infty$.  \\
\end{lemma}
\begin{proof}
Wykorzystamy uogólnioną nierówność Hardy'ego.  Załóżmy $\psi(t) = \phi(t) =
e^{- \frac{t}{p} }$. Wtedy jeśli $\frac{1}{p} + \frac{1}{q} = 1 $, to
$\frac{1}{q} = \frac{p-1}{p}$ oraz $-q = \frac{p}{1-p}$.  Zbadajmy teraz, czy
spełniony jest warunek, aby zachodzić mogła nierówność Hardy'ego:
$$
\sup_{x > 0}
\left[
\int_x^\infty  
    e^{-t} dt
\right]^{\frac{1}{p}}
\left[
\int_0^x
    e^{-t \frac{1}{1-p}} dt
\right]^{\frac{p-1}{p}}
=
\sup_{x > 0}
    C
    e^{- \frac{x}{p}}
    \left(
        e^{\frac{x}{p-1}} - 1
    \right)^{\frac{p-1}{p}}
=
$$
$$
C
\sup_{x > 0}
    \left(
    e^{- \frac{x}{p-1}}
        e^{\frac{x}{p-1}} -
    e^{- \frac{x}{p-1}}
    \right)^{\frac{p-1}{p}}
=
C
\sup_{x > 0}
    \left(
        1 -
    e^{- \frac{x}{p-1}}
    \right)^{\frac{p-1}{p}}. 
$$
Wyrażenie to jest ograniczone dla $p> 1$. Po zastosowaniu uogólnionej
nierówności Hardy'ego i ewentualnym przeskalowaniu zmiennej $x$ do $\alpha x$
otrzymujemy tezę.
\end{proof}


\chapter{Obliczenie}

Celem tego rozdziału jest prezentacja obliczenia 
$L_p$-kohomologii Riemannowskiego $f$-stożka z funkcją wagową $f = e^{-t}$.
Obliczenie jest wykonane sposobem prezentowanym między innymi w
pracach \cite{cheeger}, \cite{youssin}, \cite{kirwan}, \cite{weber}.

\begin{definition}[$f$-stożek]
    Niech $\M$ będzie rozmaitością Riemannowską. Rozważmy przestrzeń
    $\mathbb{R}_{\geq 0} \times \mathcal{M}$. Określmy na tym produkcie tensor
    Riemannowski zadany przez wzór $dt^2 + f^{2}(t)g $, gdzie $g$ jest
    metryką na $\mathcal{M}$.  Przestrzeń taką nazywamy \textbf{$f$-stożkiem}.
    Oznaczać ją będziemy przez symbol $\cfm$.
\end{definition}

\begin{definition}
  Niech $L_p^k \mathcal{M}$ oznacza przestrzeń $p$-całkowalnych 
  $k$-form różnikowych z mierzalnymi  współczynnikami.
\end{definition}


%% TODO -> picture % Rysuneczek z rurkom

Możemy teraz poczynić obserwację o formach różniczkowych określonych na 
$f$-stożku. Przestrzeń styczna do $\cfm$ w punkcie $(t, m)$ to:
\[
    T_{(t, m)} (\mathrm{c}^f \mathcal{M}) = \mathbb{R} \times T_m \mathcal{M}.
\]
W terminach form różniczkowych powiązanych z rozważanym $f$-stożkiem oznacza to, 
że możemy napisać:

\[
\Lambda^k(\mathbb{R} \times T_m \M) = 
\Lambda^{k-1}(\M)  \oplus \Lambda^k(\M).
\]
Spostrzeżenie to możemy wyrazić także w inny sposób: 

\begin{remark}
Każda $k$-forma $\omega \in \Lambda^k T(\mathrm{c}^f \mathcal{M})$, 
a w konsekwencji każda forma z przestrzeni form $p$-całkowalnych  $L_k^p
(\cfm)$ może być zapisana jako$\omega = \eta + \xi \wedge dt$,
gdzie zarówno $\eta$, jak i  $\xi$ nie zawierają $dt$.  Zauważmy ponadto,
że $\eta$ jest $k$-formą, a $\xi$ jest $k-1$ formą. \\
\end{remark}

Przypomnijmy także dla klarowności notację dotyczącą zapisywania form różniczkowych
względem lokalnych współrzędnych. Dowolną $k$-formę $\eta$, która w domyśle nie zawiera
czynnika $dt$, zapisywać będziemy względem lokalnych rzeczywstich współrzędnych
$(x_1, x_2, ... , x_n)$ na $\M$ jako:
\[
    \eta(t, x) = \sum_{\alpha \in I(k)} \eta_\alpha (t, x) dx^\alpha,
\]
gdzie $I(k)$ jest zbiorem wszystkich multiindeksów $\alpha = (\alpha_1, ...,
\alpha_k)$ takich, że $1 \leq \alpha_1 < ... < \alpha_i \leq n$, gdzie
\begin{equation}\label{notacja}
    dy^\alpha = dy^{\alpha_1} \wedge ... \wedge dy^{\alpha_k},
\end{equation}
a $\eta_\alpha$ jest gładką funkcją określoną na $(0, \infty) \times \M$. \\

Pomiędzy rozmaitościami $\M$ oraz $\cfm$ istnieją kanoniczne przekształcenia
projekcji oraz inkluzji. Żeby dobrze zilustrować w sposób w jaki działają one
na formy różniczkowe na poszczególnych rozmaitościach, przypomnijmy ich typy.
Inkluzja to funkcja:
\[
    i_r: \M \rightarrow \cfm \\
\]
\[
    i_r(x) = (x, r).
\]
Możemy za jej pomocą przeciągać formy z $\cfm$ do $\M$. Przeciągnięcie takie
oznaczymy jako $\omega_r = i_r^\ast(\omega) = i_r^\ast \eta $ dla formy $\omega
= \eta + \xi \wedge dt$. \\
Projekcja (rzutowanie) zadane jest jako:
\[
    \pi: \cfm \rightarrow \M
\]
\[
    \pi (x, t) = x.
\] \\


Rozważamy formę $\omega \in L^k_p (\cfm)$, gdzie
$\omega = \eta + \xi \wedge dt$.
Zauważmy, że metryka Riemannowska na $\cfm$ jest określona w taki sposób, że
stosowne normy spełniają następujące zależności:
$$
| \eta(t,x) |^2 = (\mathrm{e}^{-t})^{-2k} | \eta(t,x) |^2_{\M} +
(\mathrm{e}^{-t})^{-2(k-1)} | \xi(t,x) |^2_{\M},
$$
gdzie $|\cdot |_{\M} $ jest normą form różniczkowych indukowaną przez
metrykę Riemannowską na rozmaitości $\M$.  Czynnik $(\mathrm{e}^{-t})^{-2k}$
pojawia się ponieważ forma $\eta$ należy do $k$-tej potęgi zewnętrznej
przestrzeni kostycznej do rozmaitości $\cfm$ w punkcie $(t,x)$.  Zauważmy
ponadto, że Riemannowska forma objętośći na $\cfm$ w punkcie $(t,x)$ różni się
od formy objętości na $\M$ w punkcie $x$ o czynnik $(e^{-t})^n$.  Policzmy więc
normę $\omega$ jako elementu przestrzeni $L_k^p (\cfm)$.

\[
    ||\omega ||^p = \int_{\cfm} |\omega |^p d \mathrm{vol}_{\cfm} =
    \int_0^\infty \left( e^{-t} \right)^{n-pk} \int_\mathcal{M} |\omega|^p d
    \mathrm{vol}_\mathcal{M} dt = 
\]
\[
    = \
    \int_0^\infty \left( e^{-t} \right)^{n-pk} || \omega_t ||_{\mathcal{M}} dt = 
    \int_0^\infty || \omega ||_t^p dt,
\] 
gdzie
\[
|| \omega ||_r \deff || \omega_{|\mathcal{M} \times \{r\} } || =
\mathrm{e}^{-r \cdot (\frac{n}{p} - k) }  ||\omega_r ||_{\M}.
\]
Przypomnijmy, że $\omega_r = i_r^\ast (\eta)$. \\

Zauważmy teraz w jaki sposób zachowywać się będzie norma formy, która
została przeciągnięta z podstawy, czyli rozmaitości $\M$. Dla 
$\eta \in L_p^\ast (\M)$ możemy napisać
\[
    || \eta ||_r \deff || \pi^\ast ||_r = 
\mathrm{e}^{-r \cdot (\frac{n}{p} - k) }  ||\eta ||_{\M}.
\] \\

Zazwyczaj w obliczeniach dotyczących kohomologii form argument pokazujący
że zachodzi formuła homotopii jest prosty. %% Przedstwiony on został
W badanym przypadku rozmaitości $\cfm$ fakt, że spełniona jest formuła homotopii
wymaga szczegółowego uzasadnienia. \\


Będziemy starać się dowieść, że zachodzi formuła:

\[
    \omega - \pi^\ast(\omega_r) = dI_r \omega + I_r d\omega
\]
W tym celu ponownie posłużymy się rozbiciem $k$-formy $\omega$, określonej na
$\cfm$ na
\[
    \omega = \eta + dt \wedge \xi.
\] 
Możemy teraz dla $\eta$ określić na $\cfm$ $k$-formę $\partial \eta / \partial
t$ zadaną w lokalnych współrzędnych $(x_1, x_2, ..., x_n)$ jako
\[
    \frac{\partial \eta}{\partial t} (t, x) =
    \sum_{\alpha \in I(k)} \frac{\eta_\alpha}{\partial t}(t, x) dx^\alpha.
\]

Wykorzystujemy tutaj notację z \ref{notacja}. Na tej samej podstawie możemy 
zapisać dla formy $\xi$ $(k-1)$ formę $\partial \xi / \partial t$ jako:
\[
    \frac{\partial \xi}{\partial t} (t, x) =
    \sum_{\alpha \in I(k-1)} \frac{\xi_\alpha}{\partial t}(t, x) dx^\alpha. 
\] \\

Może zdefiniować na stosownym
kompleksie form różnikowych

\[
    d_\M \omega =  
    \sum_{1 \leq j \leq m} \sum_{\alpha \in I(k)}
    \frac{\partial \eta_\alpha}{\partial x_j}(t, x) dx_j \wedge dy^\alpha
\]
\[
    + \sum_{1 \leq j \leq m} \sum_{\alpha \in I(k-1)}
    \frac{\partial \xi_\alpha}{\partial x_j}(t, x) dx_j \wedge dt \wedge dy^\alpha.
\]
Możemy wtedy określić różniczkę formy najpierw na podstawie $f$-stożka:
\[
    d_\M \omega = d_\M \eta - dt \wedge d_\M \xi,
\]
a następnie na całym badanym $f$-stożku:
\[
    d \omega = 
    d_\M \omega + dt \wedge \frac{\partial \eta}{\partial t} = 
    d_\M \eta + dt \wedge \left( 
        \frac{\partial \eta}{\partial t} - d_\M \xi
    \right). \\
\]

Możemy teraz postarać się zdefiniować operator homotopii.
Dla ustalonego $r \in (0, \infty)$ określimy operator
\[
    I_r: \Lambda^k(\cfm) \rightarrow \Lambda^{k-1}(\cfm),
\]
który w lokalnych współrzędnych $(x_1, x_2, ..., x_n)$ jest zadany wzorem:
\[
    (I_r \omega)(t, x) = \sum_{\alpha \in I(k-1)}
      \left(
          \int_s^t \xi(\tau, x) d\tau 
      \right) dy^\alpha.
\] 
Dla klarowności kolejnych wzorów wprowadzimy oznaczenie. Będziemy mianowicie pisać
\[
    (I_r \omega)(t, x) = \int_s^t \xi.
\] \\

Możemy teraz zbadać wyrażenia $d I_r \omega$ oraz $I_r d \omega$, które występują
w formule homotopii. Napiszemy:
\[
    d I_r \omega = d_\M \int_r^t \xi + dt \wedge \frac{\partial}{\partial t} \int_r^t \xi =
\]
\[
    = \int_r^t d_\M \xi + dt \wedge \xi
\]
oraz
\[
    I_r \omega = I_r
    \left(
        d_\M \eta + dt \wedge \left( \frac{\partial \eta}{\partial t} - d_\M \xi \right)
    \right)
\]
\[
    = \int_r^t 
        \left(
            \frac{\partial \eta}{\partial t} - d_M \xi
        \right)
\]
\[
    = \eta - \pi^\ast \left( \eta^{(r)} \right) - \int_r^t d_\M \xi,
\]

gdzie $\eta^{(r)} \in \Lambda^k (\M)$ jest zadane w lokalnych współrzędnych
$(x_1, x_2, ..., x_n)$ jako 
\[
    \eta^{(s)}(x) = \sum_{\alpha \in I(k)} \eta_\alpha(t, x) dx^\alpha.
\] \\

W związku z tym, zachodzi wzór:
\[
    d I_r \xi + I_r d \xi = dt \wedge \xi + \eta - \pi^\ast \left( \eta^{(r)} \right).
\] \\

Spróbujemy teraz zbadać, kiedy dla $p$-całkowalnej formy $\omega$ forma $I_r \omega$
także jest całkowalna. Fakt, że forma jest  $p$-całkowalna oznacza innymi słowy, że
\[
    || \omega ||^p = \int_{\cfm} |\omega|^p =
    \int_0^1 \int_{\M} \left(
        (e^{-t})^{-2k} | \eta^{(t)}|^p_{\M} + 
        (e^{-t})^{-2(k-1)} | \xi^{(t)}|^p_{\M} 
    \right)
    (e^{-t})^{m} dt.
\]

Jako że $I_r \omega$ jest $(k-1)$ formą, to możemy napisać
\[
    ||I_r \omega ||^p = 
    \int_{\cfm} ||I_r \omega||^p = 
    \int_0^1 \int_{\M} \left(
        (e^{-t})^{-p(k-1)}
        \int_r^t ||\xi^{(\tau)} d\tau||^p_{\M} (e^{-t})^n dt
    \right).
\] \\

Wykorzystamy teraz lemat, który udowodniony został w dodatku. \\

Lemat możemy wykorzystać, tylko wtedy gdy 
wspołczynnik przy $e^{-t}$, nazwijmy go $\alpha$, jest większy od zera. Ten współczynnik to:
$\alpha = -p(k-1) + n$. Zbadajmy warunek:
\[
    -p(k-1) + n > 0
\]
\[
    -k + 1 + \frac{n}{p} > 0
\]
\[
    k < \frac{n}{p} + 1.
\]
Warunek pozwalający na wykorzystanie lematu jest więc spełniony dla $k \leq \frac{n}{p}$. \\

Gdy warunek ten jest spełniony, stosujemy lemat ... i uzyskujemy rezultat, że $I_r \omega$
jest formą $p$-całkowalną. \\

Pokazaliśmy więc, że jeśli $\omega$ jest $k$-formą $p$-całkowalną, to  $I_r \omega$ jest
$p$-całkowalna i zachodzi wzór
\[
    \omega = dI_r \omega + I_r d \omega + \pi^\ast 
    \left(
        \eta^{(s)}.
    \right).
\]
Stąd jeśli $d \omega = 0$< to 
\[
    \eta \in d(L^{i-1} \cfm) + \pi^\ast (L^i*\M),
\]
więc 
\[
    \pi^\ast: H \left( \M \right) \rightarrow H ( \cfm )
\]
jest operatorem suriektywnym. \\

%%% Zbadać co się dzieje w tym przeciwnym przypadku

\section{Co pan Weber kazał mi zrobić}
Ten fragment, który Pan kontempluje: Chodzi o to, że zamiast form gładkich
trzeba rozważać formy o mierzalnych współczynnikach z różniczką w sensie
"prądów" (currents), czyli funkcjonałów na formach. Być może to troche zbyt
obszerny temat na licencjat. Moim zdaniem wystarczy by Pan oszacował operator I
(odcałkowanie form) w normie $L^p$ i powiedział że, tak jak np w mojej pracy
pozwoli to dowieść znikanie kohomologii w jakichś gradacjach.

Czyli twierdzenie główne pracy by było: Operator
\[
 I_0:A^k \to A^{k+1}
\]
jest ciągły w normie $L^p$ dla k.... \\

Operator
\[
I_\infty:A^k \to A^{k+1}
\] jest ciągły w normie $L^p$ dla k.... ,
gdzie $A^k$ oznacza $k$-formy o mierzalnych współczynnikach. \\






\begin{thebibliography}{99}
\addcontentsline{toc}{chapter}{Bibliography}

\bibitem[Weber]{weber} Andrzej Weber, \textit{An isomorphism from
  intersection homology to $\mathrm{L}_p$-cohomology}, Forum
  Mathematicum, de Gruyer, 1995.
  
\bibitem[Cheeger]{cheeger} Jeff Cheeger, \textit{On the Hodge theory
  of Riemannian pseudomanifolds}, Proc. of Symp. in Pure Math. vol.36,


\bibitem[Bott]{bott} Raou Bott, \textit{Differetial forms in algebraic
  topology}, Springer Verlag, 1982.

\bibitem[Youssin]{youssin} Boris Youssin, \textit{$\mathrm{L}_p$
  cohomology of cones and horns } J. Differential Geometry, Volume 39,
  Number 3, 1994.
  
\bibitem[Lee]{lee} Lee, \textit{Introduction to Smooth Manifolds}

\bibitem[Ducret]{lausanne} Stephen Ducret, \textit{$L_{q,p}$-Cohomology of Riemannian
    Manifolds and Simplical Complexes of Bounded Geometry}, Ecole Polytechnique Federale
    de Lausanne, Doctor Thesis, Lausanne, 2009

\end{thebibliography}

\end{document}
